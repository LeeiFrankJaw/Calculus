\chapter{实数与函数}

\section{实数集的界与确界}

\begin{definition}
  \label{defn:bnd}
  设\(A\)是一个非空实数集.若存在\(M > 0\),使得对任意的\(x \in A\)都有\(\abs{\,x\,} \le M\),则称实数集\(A\)有界,称\(M\)为\(A\)的一个界;若存在\(M_1 \in \R\),使得对任意的\(x \in A\)都有\(x \le M_1\),则称实数集\(A\)有上界,称\(M_1\)为\(A\)的一个上界;下界的定义类似.
\end{definition}

% \pskip
\begin{theorem*}
  实数集有界的充分必要条件是其既有上界又有下界.

  \begin{proof}
    只证充分性.设\(M_1, M_2\)分别为实数集的上下界,取\(M = \maxb*{\,\abs*{\,M_1\,},\, \abs*{\,M_2\,}\,}\),则\(M\)就是该实数集的一个界.
  \end{proof}
\end{theorem*}

\begin{example*}
  证明正整数集\(\Z^+\)是无界集.

  \begin{proof}
    对于任意的\(M > 0\),取\(n_0 = \floor M + 1\)都有\(n_0 \in \Z^+\)且\(n_0 > M\).
  \end{proof}
\end{example*}


\begin{example*}
  证明\(\brce{n \sin \frac{n\pi}2}\)是无界集.

  \begin{proof}
    对于所有的\(M > 0\),取\(n_0 = 2 \floor M + 1\),那么就有
    \[
      \abs*{\,n_0 \sin \frac{n_0\pi}2\,} = \abs*{(-1)^{\floor M} n_0\,} = n_0 = 2 \floor M + 1 > M. \qedhere
    \]
  \end{proof}
\end{example*}

\begin{definition*}
  实数集\(A\)的最小上界称作上确界\footnote{上(下)确界的概念最早可以追溯到Roger Paman\cite{PamanBio}.},记作\(\sup A\);它的最大下界称作下确界,记作\(\inf A\).
\end{definition*}

\begin{definition}
  \label{defn:supinf}
  对于实数集\(A\)和实数\(M\),我们称\(M\)为\(A\)的上确界,如果:\enumparen{1}对于任意的\(x \in A\),都有\(x \le M \);\enumparen{2}对于任意的\(ε > 0\),都存在\(x \in A\)使得\(x > M - ε\).下确界的定义类似.
\end{definition}

\begin{example*}
  证明\(\infb[\big]{\frac1n} = 0\).

  \begin{proof}
    取\(x = 1/\paren[\big]{\floor[\big]{\frac1ε}+1}\)即可.
  \end{proof}
\end{example*}

\begin{axiom}[确界存在公理]
  \label{ax:lubglb}
  若非空实数集\(A\)有上界,则其在实数范围内上有上确界;若有下界,则在实数范围内有下确界.
\end{axiom}

\subpdfbookmark{思考}{B1.1.1.P}
\subsection*{思考}

设\(E \subset R\)为非空实数集.上确界\(\sup E\)和最大值\(\max E\)有什么和联系区别?下确界\(\inf E\)和最小值\(\min E\)呢?

\ifshowsolp
  \pskip
  如果\(E\)有最大值,则\(\sup E = \max E\).最小值的情况同理.
\fi

\ifshowex
\currentpdfbookmark{练习}{B1.1.1.E}
\subsection*{练习\label{B1.1.1.E}}

\begin{enumerate}
\item 下列集合中,有最大值的集合是\uline{\makebox[4em]{}}.
  \begin{itemize}
    \renewcommand{\labelitemi}{\faCircleThin}
  \item \(\brce{\,x \mid x \in (0, 1),\, x \in \Q\,}\)
  \item 自然数集
    \ifshowsol
    \item[\faCircle]
    \else
    \item
    \fi
    有限个数构成的集合\(\Set{\,a_1,\, a_2,\, \dots,\, a_n\,}\)
  \item \(\Set[\big]{\,x \in \R \mid x^2 - 2x -3 < 0\,}\)
  \end{itemize}

\item 集合\(\Set{\,x \in \R \mid -x^2 - x + 2 > 0\,}\)的上确界是
  \ifshowsol
    \uline{\makebox[4em]{\(1\)}}
  \else
    \uline{\makebox[4em]{}}%
  \fi.

\item \(\displaystyle \bigcup_{n=1}^\infty \paren[\Big]{1 + \frac1n, 2 - \frac1n} =\)
  \ifshowsol
    \uline{\makebox[4em]{\((1, 2)\)}}
  \else
    \uline{\makebox[4em]{}}%
  \fi.

\item \(\displaystyle \bigcap_{n=1}^\infty \parenbrkt[\Big]{2-\frac1n, 3-\frac1n} =\)
  \ifshowsol
    \uline{\makebox[4em]{\(\brce{2}\)}}
  \else
    \uline{\makebox[4em]{}}%
  \fi.

\item 设集合\(A = \brkt{-1,1},\ B = \paren{0,2}\),则\(A \setminus B =\)
  \ifshowsol
    \uline{\makebox[4em]{\(\brkt{-1,0}\)}}
  \else
    \uline{\makebox[4em]{}}%
  \fi.

\item 设集合\(A = \Set{\,x \in \R \mid x^2 < 2\,},\ B = \Set{\,x \in \Q \mid x^2 < 2\,}\),则\uline{\makebox[4em]{}}.
  \begin{itemize}
    \renewcommand{\labelitemi}{\faCircleThin}
  \item \(\sup A > \sqrt2 > \sup B\)
  \item \(\sup A = \sqrt2 > \sup B\)
  \item \(\sup A > \sqrt2 = \sup B\)
    \ifshowsol
    \item[\faCircle] \(\sup A = \sqrt2 = \sup B\)
    \else
    \item \(\sup A = \sqrt2 = \sup B\)
    \fi
  \end{itemize}

\item 若\(A, B\)为\(\R\)中的非空有界集,记
  \begin{equation*}
    M_1 = \infp*{A \cup B},\ M_2 = \minb*{\,\inf A,\, \inf B\,},\ N_1 = \supp*{A \cup B},\ N_2 = \maxb*{\,\sup A,\, \sup B\,},
  \end{equation*}
  则\uline{\hspace{8em}}.
  \begin{itemize}
    \renewcommand{\labelitemi}{\faCircleThin}
  \item \(M_1 < M_2,\ N_1 > N_2\)
  \item \(M_1 = M_2,\ N_1 > N_2\)
  \item \(M_1 < M_2,\ N_1 = N_2\)
    \ifshowsol
    \item[\faCircle] \(M_1 = M_2,\ N_1 = N_2\)
    \else
    \item \(M_1 = M_2,\ N_1 = N_2\)
    \fi
  \end{itemize}

\item 若\(A, B\)为\(\R\)中的两个不同非空有界集,且\(A \cap B\)非空,记
  \begin{equation*}
    M_1 = \infp{A \cap B},\ M_2 = \maxb{\,\inf A,\, \inf B\,},\ N_1 = \supp{A \cap B},\ N_2 = \minb{\,\sup A,\, \sup B\,},
  \end{equation*}
  则\uline{\hspace{8em}}.
  \begin{itemize}
    \renewcommand{\labelitemi}{\faCircleThin}
    \ifshowsol
    \item[\faCircle] \(M_1 \ge M_2,\ N_1 \le N_2\)
    \else
    \item \(M_1 \ge M_2,\ N_1 \le N_2\)
    \fi
  \item \(M_1 > M_2,\ N_1 < N_2\)
  \item \(M_1 < M_2,\ N_1 > N_2\)
  \item \(M_1 = M_2,\ N_1 = N_2\)
  \end{itemize}

\item \label{B1.1.1.E9}若\(a, b \in \R\),则\(\maxb{\,a, b\,}\)和\(\minb{\,a, b\,}\)分别是\uline{\makebox[4em]{}}.
  \begin{itemize}
    \renewcommand{\labelitemi}{\faCircleThin}
    \ifshowsol
    \item[\faCircle] \(\displaystyle \frac{a+b+\abs{\,a-b\,}}{2},\ \frac{a+b-\abs{\,a-b\,}}{2}\)
    \else
    \item \(\displaystyle \frac{a+b+\abs{\,a-b\,}}{2},\ \frac{a+b-\abs{\,a-b\,}}{2}\)
    \fi
  \item \(\displaystyle \frac{a-b+\abs{\,a-b\,}}{2},\ \frac{a+b-\abs{\,a-b\,}}{2}\)
  \item \(\displaystyle \frac{a+b-\abs{\,a-b\,}}{2},\ \frac{a-b+\abs{\,a-b\,}}{2}\)
  \item \(\displaystyle \frac{a+b+\abs{\,a-b\,}}{2},\ \frac{a-b+\abs{\,a-b\,}}{2}\)
  \end{itemize}

\item 设\(A,\ B\)为\(\R^+\)中的非空有界集,定义数集\(P = \Set{\,x \mid x = a + b,\, a \in A,\, b \in B\,},\, Q = \Set{\,x \mid x = a - b,\, a \in A,\, b \in B\,}\),则下列四个等式
  \begin{align*}
    \sup P &= \sup A + \sup B & \inf P &= \inf A + \inf B \\
    \sup Q &= \sup A - \sup B & \inf Q &= \inf A - \inf B
  \end{align*}
  正确的等式个数是
  \ifshowsol
    \uline{\makebox[4em]{\(2\)}}%
  \else
    \uline{\makebox[4em]{}}%
  \fi
  个.

  \ifshowsol
    前两个等式是正确的,后两个改成\(\sup Q = \sup A - \inf B\)和\(\inf Q = \inf A - \sup B\)就正确了.
  \fi
\end{enumerate}
\fi

\section{函数的概念}

\begin{definition*}
  设\(D\)是一个非空实数集,\(\,f\)是\(D\)上的一个对应关系.若对任意的\(x \in D\),都存在唯一的实数\(y\)通过\(\,f\)与\(x\)对应,则称\(\,f\)是定义在\(D\)上的一个函数,记作\(y = f(x),\ x \in D\).
\end{definition*}

函数的记号是欧拉的,现代定义最早是由狄利克雷提出的.

\begin{definition*}
  设函数\(\,f\mkern2mu\)在\(D\)上有定义,称点集\(\Set{\,(x,y) \,\mid\, y = f(x),\, x \in D\,}\)为\(\,f\mkern2mu\)的图形.
\end{definition*}

\begin{definition}[狄利克雷函数]
  \label{defn:dirichlet}
  \[
    \Fn D(x) =
    \begin{cases}
      1, & x \in \Q, \\
      0, & x \in \R \setminus \Q.
    \end{cases}
  \]
\end{definition}

\begin{definition}[单位阶跃函数]
  \label{defn:heaviside}
  \[
    \Fn H(x) =
    \begin{cases}
      1, & x \ge 0, \\
      0, & x < 0.
    \end{cases}
  \]
\end{definition}

\begin{definition*}[正负号函数\footnote{通行的译名为“符号函数”,有歧义,不从.另参见\url{https://www.termonline.cn/wordDetail?subject=f58bc05926ad11ee9fd4b068e6519520&base=1}}]
  \[
    \sgn(x) =
    \begin{cases}
      1, & x > 0, \\
      0, & x = 0, \\
      -1, & x < 0.
    \end{cases}
  \]
\end{definition*}

\begin{definition*}[分段函数]
  在定义域的不同范围上,用不同的数学表达式定义的函数称作分段函数.
\end{definition*}

\begin{definition*}[隐函数]
  无法或者不方便显式地来定义,而是使用代数关系(一般无法写出解析解形式)来确定的函数.
\end{definition*}

\begin{example*}
  对于任意\(k \le 0\),找到使得等式\(2^x = kx + 2\)成立的\(x\),这样建立的从\(k\)到\(x\)的关系就是一个隐函数.
\end{example*}

\begin{example*}
  易知函数\(y = x + \frac12 \sin x\)是单调递增函数,那么由这个函数确定的反函数\(x = x(y)\)也是一个隐函数\footnote{反函数和单调函数的定义见于第~\ref{sec:funcops}节和第~\ref{sec:funcprops}节.}.
\end{example*}

\begin{example*}
  等式\(x^2 + y^2 = 1\)是函数\(y = \sqrt{1 - x^2}\)和函数\(y = -\sqrt{1 - x^2}\)的隐式表示.
\end{example*}

\subpdfbookmark{思考}{B1.1.2.P}
\subsection*{思考}

\begin{enumerate}
\item 取整函数的定义为
  \[
    y = \floor x = n,\ x \in \brktparen{n, n+1},\ n = 0, \pm1, \pm2, \dots
  \]
  试画出取整函数的图像.

  \ifshowsolp
    \begin{figure}[H]
      \centering
      \tikzsetnextfilename{B1.1.2.P.1}
      \begin{tikzpicture}[scale=0.67,font=\scriptsize]
        \draw[->] (-4,0) -- (5,0) node[anchor=west] {\(x\)};
        \draw[->] (0,-4) -- (0,5) node[anchor=south] {\(y\)};
        \draw[thick] plot[jump mark left,mark=*]
        coordinates{(-3,-3) (-2,-2) (-1,-1) (0,0) (1,1) (2,2) (3,3) (4,4)};
        \draw[-] (-3,-0.1)  node[anchor=north,yshift=1pt]                 {\(-3\)} -- (-3,0.1);
        \draw[-] (-2,-0.1)  node[anchor=north,yshift=1pt]                 {\(-2\)} -- (-2,0.1);
        \draw[-] (-1,-0.1)  node[anchor=north,yshift=1pt]                 {\(-1\)} -- (-1,0.1);
        \draw[-] (0,-0.1)   node[anchor=north east,yshift=1pt,xshift=1pt] {\(0\)}  -- (0,0.1);
        \draw[-] (1,-0.1)   node[anchor=north,yshift=1pt]                 {\(1\)}  -- (1,0.1);
        \draw[-] (2,-0.1)   node[anchor=north,yshift=1pt]                 {\(2\)}  -- (2,0.1);
        \draw[-] (3,-0.1)   node[anchor=north,yshift=1pt]                 {\(3\)}  -- (3,0.1);
        \draw[-] (4,-0.1)   node[anchor=north,yshift=1pt]                 {\(4\)}  -- (4,0.1);
        \draw[-] (-0.05,-3) node[anchor=east,xshift=3pt]                  {\(-3\)} -- (0.05,-3);
        \draw[-] (-0.05,-2) node[anchor=east,xshift=3pt]                  {\(-2\)} -- (0.05,-2);
        \draw[-] (-0.05,-1) node[anchor=north east,yshift=2pt,xshift=3pt] {\(-1\)} -- (0.05,-1);
        \draw[-] (-0.05,1)  node[anchor=east,xshift=3pt]                  {\(1\)}  -- (0.05,1);
        \draw[-] (-0.05,2)  node[anchor=east,xshift=3pt]                  {\(2\)}  -- (0.05,2);
        \draw[-] (-0.05,3)  node[anchor=east,xshift=3pt]                  {\(3\)}  -- (0.05,3);
        \draw[-] (-0.05,4)  node[anchor=east,xshift=3pt]                  {\(4\)}  -- (0.05,4);
      \end{tikzpicture}
    \end{figure}
  \fi

\item 能否画出狄利克雷函数的图像?

  \ifshowsolp
    不能.
  \fi
\end{enumerate}

\ifshowex
\currentpdfbookmark{练习}{B1.1.2.E}
\subsection*{练习}

\begin{enumerate}
\item 函数\(y = \sqrt{x - \sqrt{x}} + \sqrt{2^x - 4}\)的定义域为
  \ifshowsol
    \uline{\makebox[5em]{\(\brktparen{2, +\infty}\)}}.
  \else
    \uline{\makebox[5em]{}}.
  \fi

\item 函数\(y = \frac1{\floor{x+1}}\)的定义域为
  \ifshowsol
    \uline{\makebox[10em]{\(\paren{-\infty, -1} \cup \brktparen{0, +\infty}\)}}.
  \else
    \uline{\makebox[10em]{}}.
  \fi

\item 函数\(y = \abs{\,x\,} + \abs{\,x-1\,} - \abs{\,4-2x\,}\)的最大值和最小值分别是
  \ifshowsol
    \uline{\makebox[3em]{\(3\)}}和\uline{\makebox[3em]{\(-3\)}}.
  \else
    \uline{\makebox[3em]{}}和\uline{\makebox[3em]{}}.
  \fi

\item 设非负实数\(x, y\)满足方程\((1-x)\,y^2 = x^3\)且\(t > 0\).若\(x = \frac{t^2}{1+t^2}\),则\(y\)能看作\(t\)的一个函数.用\(t\)表示\(y\),有\(y =\)
  \ifshowsol
    \uline{\makebox[6em]{\(t^3/\paren{1+t^2}\)}}.
  \else
    \uline{\makebox[6em]{}}.
  \fi

\item 有函数
  \[
    f(x) =
    \begin{cases}
      \frac1{2^x}, & x \ge 4, \\
      \,f(x+1), & x < 4.
    \end{cases}
  \]
  求\(\,f(\log_2 3) =\)
  \ifshowsol
    \uline{\makebox[4em]{\(1/{24}\)}}.
  \else
    \uline{\makebox[4em]{}}.
  \fi

\item 函数
  \[
    f(x) =
    \begin{cases}
      x^2 + 2x - 3, & x \le 0, \\
      -2 + \ln x, & x > 0
    \end{cases}
  \]
  的零点个数为
  \ifshowsol
    \uline{\makebox[3em]{\(2\)}}.
  \else
    \uline{\makebox[3em]{}}.
  \fi

\item 函数
  \[
    f(x) =
    \begin{cases}
      4x - 4, & x \le 1, \\
      x^2 - 4x + 3, & x > 1
    \end{cases}
  \]
  的图像与函数图像
  \[
    g(x) = \log_2 x
  \]
  的图像交点个数是
  \ifshowsol
    \uline{\makebox[3em]{\(3\)}}.
  \else
    \uline{\makebox[3em]{}}.
  \fi

\item 设函数
  \[
    f(x) =
    \begin{cases}
      2^{-x} - 1, & x \le 0, \\
      x^{1/2}, & x > 0.
    \end{cases}
  \]
  不等式\(\,f(x) > 1\)的解集是
  \ifshowsol
    \uline{\makebox[10em]{\(\paren{-\infty, -1} \cup \paren{1, +\infty}\)}}.
  \else
    \uline{\makebox[10em]{}}.
  \fi

\item 函数\(\displaystyle y = \frac{\ln(1+3x)}{x^2-3x+2}\)的定义域为
  \ifshowsol
    \uline{\makebox[13em]{\(\Set{\,x \mid x > -1/3,\, x \ne 1,\, x \ne 2\,}\)}}.
  \else
    \uline{\makebox[13em]{}}.
  \fi

\item 函数\(y = \arcsin(2x+1)\)的定义域为
  \ifshowsol
    \uline{\makebox[5em]{\(\brkt{-1,0}\)}}.
  \else
    \uline{\makebox[5em]{}}.
  \fi

\item 函数\(y = a(x-2) + b + 3\)的图像可由函数\(y = ax + b\)的图像\uline{\makebox[4em]{}}得到.
  \begin{itemize}
    \renewcommand{\labelitemi}{\faCircleThin}
  \item 向左平移\(2\)个单位,向上平移\(3\)个单位
  \item 向左平移\(2\)个单位,向下平移\(3\)个单位
    \ifshowsol
    \item[\faCircle] 向右平移\(2\)个单位,向上平移\(3\)个单位
    \else
    \item 向右平移\(2\)个单位,向上平移\(3\)个单位
    \fi
  \item 向右平移\(2\)个单位,向下平移\(3\)个单位
  \end{itemize}

\item 下列各组函数中,两个函数相等的一组是\uline{\makebox[4em]{}}.
  \begin{itemize}
    \renewcommand{\labelitemi}{\faCircleThin}
  \item \(\,f(x) = \sqrt{x^2},\ g(x) = x\)
    \ifshowsol
    \item[\faCircle] \(\,f(x) = 3\ln x,\ g(x) = \ln x^3\)
    \else
    \item \(\,f(x) = 3\ln x,\ g(x) = \ln x^3\)
    \fi
  \item \(\,f(x) = \tan x,\ g(x) = \frac1{\cot x}\)
  \item \(\,f(x) = x - 3,\ g(x) = \frac{x^2-2x-3}{x+1}\)
  \end{itemize}
\end{enumerate}
\fi

\section{函数的运算\label{sec:funcops}}

\begin{definition}
  \label{defn:func4ops}
  设函数\(\,f\mkern2mu\)和\(\mkern1mu g\)的定义域分别为\(D_1\)和\(D_2\),且\(D = D_1 \cap D_2\)非空.函数的四则运算为
  \begin{enumerate}
    \renewcommand{\labelenumi}{\enumparen{\arabic{enumi}}}
  \item \(\,f \pm g\colon x \mapsto f(x) \pm g(x),\ x \in D\);\hfill(加减法)
  \item \(\,fg\colon x \mapsto f(x) \, g(x),\ x \in D\);\hfill(乘法)
  \item \(\,f/g\colon x \mapsto f(x) / g(x),\ x \in D \tand g(x) \ne 0\).\hfill(除法)
  \end{enumerate}
\end{definition}

\begin{example*}
  设\(\,f(x) = x + \sqrt x,\ g(x) = x - \sqrt x\).
  经过函数的加法运算,有\((\,f+g)(x) = 2x,\ x \in \brktparen{0, +\infty}\).
\end{example*}

\begin{definition*}
  设\(g\colon D_g \to Z_g,\ f\colon D_f \to Z_f\)且\(Z_g \subset D_f\),那么对于任意的\(x \in D_g\),函数值\(\,f\paren{g(x)}\)都是有意义的.函数\(\,f\mkern2mu\)和\(\mkern1mu g\)的复合函数为
  \[
    \begin{split}
      f \circ g\colon & D_g \to Z_f \\
      & x \mapsto f\paren{g(x)}.
    \end{split}
  \]

  \begin{remark}
    我们经常把条件放宽,只要\(Z_g \cap D_f \ne \emptyset\),就可以做复合,这个函数和上面的定义中的规则一样,只是定义域可能会变窄,即
    \[
      \begin{split}
        f \circ g\colon & g^{-1}(Z_g \cap D_f) \to Z_f \\
        & x \mapsto f\paren{g(x)}.
      \end{split}
    \]
    注意到\(g^{-1}(Z_g \cap D_f) \subset D_g\).
  \end{remark}
\end{definition*}

\begin{example*}
  设函数\(y = \sqrt u\)和\(u = 1+x^2\).它们可以做复合运算得到\(y = \sqrt{1+x^2}\).
\end{example*}

\begin{example*}
  设函数\(y = \arcsin u\)和\(u = 2+x^2\).它们无法做这样的复合运算\(y = \arcsin(2+x^2)\).
\end{example*}

\begin{definition}
  \label{defn:funcinv}
  设函数\(\,f\colon D \to Z_f\)和二元关系\(g \subset Z_f \times D\).若对任意的\(y \in Z_f\)都能通过\(\mkern1mu g\)找到唯一的二元组\((y,x) \in g\)使得\(y = f(x)\),则称\(\mkern1mu g\)是\(\,f\mkern2mu\)的反函数,记作\(\,f^{-1}\).
\end{definition}

\begin{remark}
  易知\(\,f(\,f^{-1}(x)) = f^{-1}(\,f(x)) = x\).
\end{remark}

\begin{theorem*}
  函数\(y = f(x)\)与其反函数\(y = f^{-1}(x)\)的图像关于直线\(y = x\)对称.
\end{theorem*}

\begin{theorem*}
  函数一一对应是其反函数存在的充分必要条件.
\end{theorem*}

\begin{example*}
  求\(y = \sinh x = (e^x - e^{-x})/2\)的反函数.

  \begin{remark}
    通过变形将\(x\)用\(y\)表出,有
    \begin{align*}
      y &= \frac{e^x - e^{-x}}{2} && \reason{原式} \\
      2y &= e^x - e^{-x} && \reason{同乘以\(2\)} \\
      0 &= \paren{e^{x}}^2- 2y\,e^x - 1 && \reason{同乘以\(e^x\)并移项} \\
      e^x &= y + \sqrt{\smash{y^2} + 1} && \reason{关于\(e^x\)解方程} \\
      x &= \lnp[\big]{y + \sqrt{\smash{y^2} + 1}}. && \reason{取对数}
    \end{align*}
    所以反函数是\(y = \arcsinh x = \lnp[\big]{x + \sqrt{\smash{x^2} + 1}}\).
  \end{remark}
\end{example*}

\begin{definition*}
  反三角函数
  \begin{enumerate}
    \renewcommand{\labelenumi}{\enumparen{\arabic{enumi}}}
  \item 反正弦函数\(y = \arcsin x\),定义域\(\brkt{-1, 1}\),值域\(\brkt{-\frac\pi2, \frac\pi2}\);
  \item 反余弦函数\(y = \arccos x\),定义域\(\brkt{-1, 1}\),值域\(\brkt{0, \pi}\);
  \item 反正切函数\(y = \arctan x\),定义域\(\paren{-\infty, +\infty}\),值域\(\paren{-\frac\pi2, \frac\pi2}\);
  \item 反余切函数\(y = \arccot x\),定义域\(\paren{-\infty, +\infty}\),值域\(\paren{0, \pi}\).
  \end{enumerate}
  其中,我们限制值域的区间叫作反三角函数的主值区间.对于反余切函数,另一种常见的习惯是将主值定义成\(\parenbrkt{-\frac\pi2,\frac\pi2}\).
\end{definition*}

\subpdfbookmark{思考}{B1.1.3.P}
\subsection*{思考}

\begin{enumerate}
\item 复合函数\(\,f \circ g\)与\(g \circ f\)是否相等?

  \ifshowsolp
    不一定相等.
  \fi

\item 什么样的函数存在反函数?如何确定反函数的定义域和值域?

  \ifshowsolp
    一一对应的函数一定存在反函数.如果不是一一对应,那么至少也应该是单射的.这时,\(\ran f\)就是\(\,f^{-1}\)的定义域,\(\,f\mkern2mu\)的定义域就是反函数的值域.如果连单射都不是,我们就要向反三角函数那样找到一个主分支,来构造反函数.
  \fi
\end{enumerate}

\ifshowex
\currentpdfbookmark{练习}{B1.1.3.E}
\subsection*{练习}

\begin{enumerate}
\item 设
  \[
    f(x) =
    \begin{cases}
      2e^{x-1}, & x < 2, \\
      \log_3(x^2-1), & x \ge 2,
    \end{cases}
  \]
  则\(\,f(\,f(2)) =\)
  \ifshowsol
    \uline{\makebox[3em]{\(2\)}}.
  \else
    \uline{\makebox[3em]{}}.
  \fi

\item 设\(\,f(x) = \cos x,\ f(\varphi(x)) = 1 - x^2\),则\(\varphi(x)\)的定义域是
  \ifshowsol
    {\setlength{\ULdepth}{.8ex}%
      \uline{\makebox[6em]{\(\brkt[\big]{-\sqrt2, \sqrt2}\)}}}.
  \else
    \uline{\makebox[6em]{}}.
  \fi

\item 下列各对函数\(y = f(u), u = g(x)\)中,可以复合成复合函数\(y = f(g(x))\)的是
  \uline{\hfill}.
  \begin{itemize}
    \renewcommand{\labelitemi}{\faCircleThin}
    \ifshowsol
    \item[\faCircle] \(\,f(u) = \sqrt{u^2 + 1},\ g(x) = e^x\)
    \else
    \item \(\,f(u) = \sqrt{u^2 + 1},\ g(x) = e^x\)
    \fi
  \item \(\,f(u) = \arccos(1+2u),\ g(x) = 1 + x^2\)
  \item \(\,f(u) = \sqrt{u+1},\ g(x) = \sin x - 3\)
  \item \(\,f(u) = \ln^2 u,\ g(x) = \arcsin x\)
  \end{itemize}

\item 函数\(y = e^{3x-1},\ x \in \brktparen{0, +\infty}\)的反函数为
  \ifshowsol
    {\setlength{\ULdepth}{.85ex}%
      \uline{\makebox[15em]{\(y = (1 + \ln x)/3,\ x \in \brktparen{e^{-1}, +\infty}\)}}}.
  \else
    \uline{\makebox[15em]{}}.
  \fi

\item 求函数
  \[
    y =
    \begin{cases}
      1 + e^{-x}, & x \le 0, \\
      2 - 2x, & 0 < x < 1, \\
      2x - (1+x^2), & x \ge 1
    \end{cases}
  \]
  的反函数.

  \ifshowsol
    反函数为
    \[
      y =
      \begin{cases}
        -\lnp{x-1}, & x \ge 2, \\
        1 - \frac12 x, & 0 < x < 2, \\
        1 + \sqrt{-x}, & x \le 0.
      \end{cases}
    \]
  \fi

\item 已知\(\,f(\sin x) = 2 \ln \cos x + x\),则\(\,f(x) =\)
  \ifshowsol
    \uline{\makebox[16em]{\(\ln(1-x^2) + 2k\pi + \arcsin x,\ k \in \Z\)}}.
  \else
    \uline{\makebox[16em]{}}.
  \fi

\item 设
  \[
    f(x) =
    \begin{cases}
      x^2, & x \ge 0, \\
      2x - 1, & x < 0,
    \end{cases}
    \qquad
    g(x) =
    \begin{cases}
      -x^2, & x \le 1, \\
      \log_2(1+x), & x > 1,
    \end{cases}
  \]
  则\(\,f(g(0))\)和\(\,f(g(1))\)的值分别是
  \ifshowsol
    \uline{\makebox[3em]{\(0\)}}和\uline{\makebox[3em]{\(-3\)}}.
  \else
    \uline{\makebox[3em]{}}和\uline{\makebox[3em]{}}.
  \fi

\item 已知\(\,f(x-1) = \ln \dfrac{x^2+1}{x^2-3}\).若\(\,f(g(x)) = x^2\),则\(g(x)\)的定义域为
  \ifshowsol
    \uline{\makebox[10em]{\(\paren{-\infty, 0} \cup \paren{0, +\infty}\)}}.

    实际上,\(\mkern1mu g\)是可以求出来的,就是
    \[
      g(x) = \pm \sqrt{\frac{3e^{x^2}+1}{e^{x^2}-1}} - 1.
    \]
  \else
    \uline{\makebox[10em]{}}.
  \fi

\item 已知\(\,f(x) = x^5 - a\)且\(\,f(-1) = 0\).求\(\,f^{-1}(1) =\)
  \ifshowsol
    \uline{\makebox[3em]{\(0\)}}.
  \else
    \uline{\makebox[3em]{}}.
  \fi

\item 设函数\(y = f(x)\)的反函数为\(y = f^{-1}(x)\)且\(y = f(2x-1)\)的图像经过点\(\paren[\big]{\frac12, 1}\).那么\(y = f^{-1}(x)\)的图像必过点
  \ifshowsol
    \uline{\makebox[3em]{\(\paren{1, 0}\)}}.
  \else
    \uline{\makebox[3em]{}}.
  \fi

\item 设函数\(\,f(x) = \log_a(x+b)\ (a > 0, a \ne 1)\)的图像经过点\(\paren{2,1}\)且其反函数的图像经过点\(\paren{2,8}\).那么\(a + b =\)
  \ifshowsol
    \uline{\makebox[3em]{\(4\)}}.

    实际上,\(a = 3\)且\(b = 1\).
  \else
    \uline{\makebox[3em]{}}.
  \fi

\item 设函数\(\,f(x) = \dfrac{2x+3}{x-1}\)且函数\(y = h(x)\)的图像与\(y = f^{-1}(x+1)\)的图像关于直线\(y = x\)对称.求\(h(3)\)的值.

  \ifshowsol
    这题相当于是求\(3 = f^{-1}(x+1)\)时\(x\)的值.有\(\,f(3) = f(\,f^{-1}(x+1)) = x+1 = \dfrac92\),所以\(h(3) = \dfrac72\).

    实际上,\(h(x) = 1 + \dfrac{5}{x-1}\).
  \fi

\end{enumerate}
\fi

\section{函数的初等性质\label{sec:funcprops}}

\begin{definition*}
  设函数\(\,f\colon D \to Z_f\)\,.若\(\ran f \subset Z_f\)是有界集,则称\(\,f\mkern2mu\)是有界函数,\(\ran f\)的界就是\(\,f\mkern2mu\)在\(D\)上的界.
\end{definition*}

\begin{example*}
  证明\(\,f(x) = \dfrac1x\)在\(\paren{0,+\infty}\)上无界.

  \begin{proof}
    对于任意的\(M > 0\),取\(x_0 = \frac1{2M}\),则有\(x_0 \in \paren{0,+\infty}\)且\(\,f(x_0) = 2M > M\).
  \end{proof}
\end{example*}

\begin{definition*}
  设函数\(\,f\mkern2mu\)的定义域关于原点对称.对于定义域上的任意\(x\),函数\(\,f\mkern2mu\)都满足\(\,f(-x) = -f(x)\),这样的函数\(\,f\mkern2mu\)叫作奇函数.对于定义域上的任意\(x\),函数\(\,f\mkern2mu\)都满足\(\,f(-x) = f(x)\),这样的函数\(\,f\mkern2mu\)叫作偶函数.
\end{definition*}

\hypertarget{T:evenodd}{}
\begin{theorem*}
  若某函数的定义域关于原点对称,则此函数可以唯一地分解成奇函数与偶函数之和.

  \begin{proof}
    设函数\(\,f\mkern2mu\)的定义域关于原点对称.易证\(\dfrac{f(x)+f(-x)}{2}\)是偶函数、\(\dfrac{f(x)-f(-x)}{2}\)是奇函数且
    \[
      \frac{f(x)+f(-x)}{2} + \frac{f(x)-f(-x)}{2} = f(x).
    \]
    这就证明了分解的存在性.此外,设偶函数\(s\)和奇函数\(t\)且\(\,f = s + t\).那么
    \[
      f(x) + f(-x) = 2\fn s(x),
      \qquad
      f(x) - f(-x) = 2\fn t(x).
    \]
    这就证明了唯一性.
  \end{proof}
\end{theorem*}

\begin{example*}
  设函数\(\,f\mkern2mu\)是奇函数,函数\(g(x) = \lnp[\big]{\,f(x) + \sqrt{\,\smash{f^2(x)+1}\rule[-.3ex]{0ex}{1.8ex}}}\).试讨论函数\(\mkern1mu g\)的奇偶性.

  \begin{remark}
    因为
    \[
      \begin{split}
        g(-x)
        &= \lnp[\big]{\,f(-x) + \sqrt{\,\smash{f^2(-x)+1}\rule[-.3ex]{0ex}{1.8ex}}} \\
        &= \lnp[\big]{\sqrt{\smash{\paren{-f(x)}^2+1}\rule[-.3ex]{0ex}{1.8ex}} - f(x)} \\
        &= \lnp[\big]{\sqrt{\,\smash{f^2(x)+1}\rule[-.3ex]{0ex}{1.8ex}} - f(x)} \\
        &= \ln{\frac{
          \paren[\big]{\sqrt{\,\smash{f^2(x)+1}\rule[-.3ex]{0ex}{1.8ex}} + f(x)}
          \paren[\big]{\sqrt{\,\smash{f^2(x)+1}\rule[-.3ex]{0ex}{1.8ex}} - f(x)}
          }{
          \sqrt{\,\smash{f^2(x)+1}\rule[-.3ex]{0ex}{1.8ex}} + f(x)}
          } \\
        &= \ln \frac1{\sqrt{\,\smash{f^2(x)+1}\rule[-.3ex]{0ex}{1.8ex}} + f(x)} \\
        &= -\lnp[\big]{\,f(x) + \sqrt{\,\smash{f^2(x)+1}\rule[-.3ex]{0ex}{1.8ex}}} \\
        &= -g(x),
      \end{split}
    \]
    所以\(\mkern1mu g\)是奇函数.

    事实上,本例中的\(\mkern1mu g\)和\(\,f\mkern2mu\)的关系,可以用\(g(x) = \arcsinh f(x)\)来表示.那么自然有\(g(-x) = \arcsinh f(-x) = \arcsinh\paren[\big]{-f(x)} = - \arcsinh f(x) = -g(x)\).
  \end{remark}
\end{example*}

\begin{definition*}
  设\(\,f\)是定义在\(\R\)上的函数.若存在正数\(T\)使得\(\,f(x+T) = f(x)\)恒成立,则称\(\,f\mkern2mu\)为周期函数.

  \begin{remark}
    函数的周期通常是指最小正周期,若存在的话.并不是每个周期函数都有最小正周期,例如常函数和狄利克雷函数.
  \end{remark}
\end{definition*}

\begin{example*}
  设函数\(\,f\mkern2mu\)的周期是\(2\)且\(g(x) = f\,\paren[\Big]{\dfrac{x+1}{2}}\).函数\(\mkern1mu g\)是不是周期函数?若是,求其周期.

  \begin{remark}
    函数\(\mkern1mu g\)的周期是\(4\).
  \end{remark}
\end{example*}

\begin{definition}
  \label{defn:funcmono}
  对于任意的\(x_1 < x_2 \in D\)都有\(\,f(x_1) < f(x_2)\)成立,则称函数\(\,f\mkern2mu\)(严格)单调递增,简称单增.对于任意的\(x_1 < x_2 \in D\)都有\(\,f(x_1) > f(x_2)\)成立,则称函数\(\,f\mkern2mu\)(严格)单调递减,简称单减.
\end{definition}

\begin{theorem*}
  函数\(\,f\)单调递增的充分必要条件是:对于任意的\(x_1, x_2 \in D\),当\(x_1 \ne x_2\)时,都有
  \begin{equation*}
    \paren[\big]{x_1 - x_2}\paren[\big]{\,f(x_1) - f(x_2)} > 0.
  \end{equation*}
  函数\(\,f\)单调递减的充分必要条件是:对于任意的\(x_1, x_2 \in D\),当\(x_1 \ne x_2\)时,都有
  \begin{equation*}
    \paren[\big]{x_1 - x_2}\paren[\big]{\,f(x_1) - f(x_2)} < 0.
  \end{equation*}
\end{theorem*}

\begin{example*}
  证明\(\,f(x) = x^3\)是单调递增函数.

  \begin{proof}
    对于任意的\(x_1, x_2 \in D\),当\(x_1 \ne x_2\)时,都有
    \[
      \begin{split}
        (x_1 - x_2) \paren[\big]{\,f(x_1) - f(x_2)}
        &= (x_1 - x_2) (x_1^3 - x_2^3) \\
        &= (x_1 - x_2)^2 (x_1^2 + x_1 x_2 + x_2^2) \\
        &> 0.
      \end{split}
    \]
    所以\(\,f\mkern2mu\)是单增函数.
  \end{proof}
\end{example*}

\begin{definition*}
  设函数\(\,f\mkern2mu\)在区间\(\paren{a,b}\)上有定义.函数\(\,f\mkern2mu\)在区间\(\paren{a,b}\)上是下凸的,当且仅当:对于任意的\(x_1, x_2 \in \paren{a,b}\)和\(\alpha \in \paren{0,1}\),当\(x_1 \ne x_2\)时都有
  \[
    f\,\paren[\big]{\alpha x_1 + (1-\alpha) x_2} < \alpha\,f(x_1) + (1-\alpha)\,f(x_2).
  \]
  函数\(\,f\mkern2mu\)在区间\(\paren{a,b}\)上是上凸的,当且仅当:对于任意的\(x_1, x_2 \in \paren{a,b}\)和\(\alpha \in \paren{0,1}\),当\(x_1 \ne x_2\)时都有
  \[
    f\,\paren[\big]{\alpha x_1 + (1-\alpha) x_2} > \alpha\,f(x_1) + (1-\alpha)\,f(x_2).
  \]
\end{definition*}

\begin{example*}
  证明函数\(\,f(x) = x^2\)在\(\R\)上是下凸的.

  \begin{proof}
    对于任意的\(x_1, x_2 \in \R\)和\(\alpha \in \paren{0,1}\),当\(x_1 \ne x_2\)时都有
    \[
      \begin{split}
        f\,\paren[\big]{\alpha x_1 + (1-\alpha) x_2}
        &= \paren[\big]{\alpha x_1 + (1-\alpha) x_2}^2 \\
        &= \alpha^2 x_1^2 + (1-\alpha)^2 x_2^2 + 2 \alpha (1-\alpha) x_1 x_2 \\
        &< \alpha^2 x_1^2 + (1-\alpha)^2 x_2^2 + \alpha (1-\alpha) (x_1^2 + x_2^2) \\
        &= \alpha x_1^2 + (1-\alpha) x_2^2 \\
        &= \alpha\,f(x_1) + (1-\alpha)\,f(x_2). \qedhere
      \end{split}
    \]
  \end{proof}
\end{example*}

\begin{example*}
  证明函数\(\,f\mkern2mu\)在区间\(\paren{a,b}\)上下凸的充分必要条件是:对于任意互异的\(x, y, z \in \paren{a,b}\)和\kenten{正数}\(\alpha, \beta, \gamma\),当\(\alpha+\beta+\gamma=1\)时都有
  \[
    f\paren{\alpha x + \beta y + \gamma z} < \alpha\,f(x) + \beta\,f(y) + \gamma\,f(z).
  \]

  \begin{proof}
    先证充分性.对于任意的\(x_1, x_2\)和满足\(\alpha+\beta=1\)的正数\(\alpha, \beta\),分别取
    \[
      c_1 = \frac\alpha2,\ c_2 = \frac12,\ c_3 = \frac\beta2,
    \]
    易知有
    \begin{gather*}
      \begin{split}
        f(\alpha x_1 + \beta x_2)
        &= f\,\paren[\big]{c_1 x_1 + c_2 \paren{\alpha x_1 + \beta x_2}  + c_3 x_2} \\
        &< c_1\,f(x_1) + c_2\,f(\alpha x_1 + \beta x_2) + c_3\,f(x_2) \\
        &= \frac α2\,f(x_1) + \frac12\,f(α x_1 + β x_2) + \frac β2\,f(x_2),
      \end{split}
      \shortintertext{从而}
      f(α x_1 + β x_2) < α\,f(x_1) + β\,f(x_2).
    \end{gather*}
    % 对于任意的\(x_1, x_2\)和满足\(\alpha+\beta=1\)的正数\(\alpha, \beta\),三个点
    % \[
    %   A\colon \paren[\big]{x_1, f(x_1)},\quad B\colon \paren[\big]{\alpha x_1 + \beta x_2, f(\alpha x_1 + \beta x_2)},\quad C\colon \paren[\big]{x_2, f(x_2)}
    % \]
    % 必然不共线.因为当共线时,存在无数多组满足\(c_1+c_2+c_3=1\)的正数\(c_1, c_2, c_3\)使得
    % \[
    %   f\,\paren[\big]{c_1 x_1 + c_2\paren{\alpha x_1 + \beta x_2}  + c_3 x_2} = c_1\,f(x_1) + c_2\,f(\alpha x_1 + \beta x_2) + c_3\,f(x_2).
    % \]
    % 那么就可以在这三个点围成的三角形里面找到一个点\(D\)使得它的横坐标是\(\alpha x_1 + \beta x_2\)(这样的点有无数个).选择一组满足\(c_1+c_2+c_3=1\)的正数\(c_1, c_2, c_3\)使得\(c_1 x_1 + c_2\paren{\alpha x_1 + \beta x_2}  + c_3 x_2 = \alpha x_1 + \beta x_2 \),也就是
    % \[
    %   D\colon \paren[\big]{\alpha x_1 + \beta x_2, c_1\,f(x_1) + c_2\,f(\alpha x_1 + \beta x_2) + c_3\,f(x_2)}.
    % \]
    % 那么就有
    % \[
    %   \begin{split}
    %     f(\alpha x_1 + \beta x_2)
    %     &= f\,\paren[\big]{c_1 x_1 + c_2\paren{\alpha x_1 + \beta x_2}  + c_3 x_2} \\
    %     &< c_1\,f(x_1) + c_2\,f(\alpha x_1 + \beta x_2) + c_3\,f(x_2) \\
    %     &< \alpha\,f(x_1) + \beta\,f(x_2).
    %   \end{split}
    % \]
    % 此时可以判断点\(B\)位于线段\(\overline{AC}\)的下方,因为\(B\)的纵坐标小于\(D\)的纵坐标.因此,上面最后一个小于号成立.正数\(c_1, c_2, c_3\)可能的一组选择是
    % \[
    %   c_1 = \frac\alpha2,\ c_2 = \frac12,\ c_3 = \frac\beta2.
    % \]

    再证必要性.对于任意互异的\(x, y, z \in \paren{a,b}\)和满足\(\alpha+\beta+\gamma=1\)的正数\(\alpha, \beta, \gamma\),都有
    \[
      \begin{split}
        f(\alpha x + \beta y + \gamma z)
        &= f\,\brkt[\Big]{\paren{\alpha+\beta}\paren[\Big]{\frac{\alpha}{\alpha+\beta} x + \frac{\beta}{\alpha+\beta} y} + \gamma z} \\
        &< \paren{\alpha+\beta}\,f\,\paren[\Big]{\frac{\alpha}{\alpha+\beta} x + \frac{\beta}{\alpha+\beta} y} + \gamma\,f(z) \\
        &< \paren{\alpha+\beta}\paren[\Big]{\frac{\alpha}{\alpha+\beta}\,f(x) + \frac{\beta}{\alpha+\beta}\,f(y)} + \gamma\,f(z) \\
        &= \alpha\,f(x) + \beta\,f(y) + \gamma\,f(z). \qedhere
      \end{split}
    \]

    % 不失一般地,假设\(x < y < z\).记\(\alpha x + \beta y + \gamma z\)为\(u\).这时\(u\)不是落在区间\(\paren{x,y}\)就是落在\(\paren{y,z}\)上.无论落在哪个区间,都可以找到一组满足\(a+b=1\)的正数\(a, b\),将\(u\)表示成\(a x + b y\)或者\(a y + b z\)的形式.又因为函数是下凸的,所以有
    % \[
    %   f(a x + b y) < a\,f(x) + b\,f(y)
    %   \quad
    %   \text{或者}
    %   \quad
    %   f(a y + b z) < a\,f(y) + b\,f(z).
    % \]
  \end{proof}
  \begin{remark}
    满足\(\alpha, \beta, \gamma\)非负且\(\alpha + \beta + \gamma = 1\)的线性组合\(\alpha x + \beta y + \gamma z\)叫作凸组合\cite{ConvexCombWiki}.此外,将例题中的条件“正数\(\alpha, \beta, \gamma\)”改为“\(\alpha, \beta, \gamma \in \brktparen{0,1}\)”,充分性的证明会变得十分平凡.将\(\alpha, \beta, \gamma\)中的一个置为零,再套用下凸的定义,即可得证.
  \end{remark}
\end{example*}

\subpdfbookmark{思考}{B1.1.4.P}
\subsection*{思考}

\begin{enumerate}
\item 将任意一个定义在实数域上的函数\(\,f(x)\)分解成一个奇函数和一个偶函数之和.

  \ifshowsolp
    参见奇偶性定义后面的\hyperlink{T:evenodd}{定理}.
  \fi

\item 函数单调性是否是函数存在反函数的必要条件?

  \ifshowsolp
    不是.反例:函数\(y = 1/x\)在定义域上不是单调的,但是存在反函数且反函数就是它自己.
  \fi
\end{enumerate}

\ifshowex
\currentpdfbookmark{练习}{B1.1.4.E}
\subsection*{练习}

\begin{enumerate}
\item 下列函数中,在定义域上无界的是\uline{\makebox[4em]{}}.
  \begin{itemize}
    \renewcommand{\labelitemi}{\faCircleThin}
  \item \(\,f(x) = \dfrac1x \sin x\)
    \ifshowsol
    \item[\faCircle] \(\,f(x) = x^2 \sin \dfrac1x\)
    \else
    \item \(\,f(x) = x^2 \sin \dfrac1x\)
    \fi
  \item \(\,f(x) = \dfrac{\ln x}{1 + \ln^2 x}\)
  \item \(\,f(x) = \dfrac1{e^x + e^{-x}}\)
  \end{itemize}

\item 函数\(\,f(x) = x e^{\sin x} \tan x\)在其定义域上是\uline{\makebox[3em]{}}.
  \begin{itemize}
    \renewcommand{\labelitemi}{\faCircleThin}
  \item 偶函数
    \ifshowsol
    \item[\faCircle] 无界函数
    \else
    \item 无界函数
    \fi
  \item 周期函数
  \item 单调函数
  \end{itemize}

\item 函数\(\,f\mkern2mu\)的周期为\(T\).那么函数\(h(x) = f(x+a)\)和\(g(x) = f(ax)\ (a > 0)\)的周期分别为
  \ifshowsol
    \uline{\makebox[3em]{\(T\)}}和\uline{\makebox[3em]{\(T/a\)}}.
  \else
    \uline{\makebox[3em]{}}和\uline{\makebox[3em]{}}.
  \fi

\item 设函数
  \[
    f(x) =
    \begin{cases}
      \sin x - x^2, & -\pi \le x < 0, \\
      \sin x + x^2, & 0 \le x \le \pi.
    \end{cases}
  \]
  那么\(\,f\mkern2mu\)是一个\uline{\makebox[4em]{}}.
  \begin{itemize}
    \renewcommand{\labelitemi}{\faCircleThin}
    \ifshowsol
    \item[\faCircle] 奇函数
    \else
    \item 奇函数
    \fi
  \item 无界函数
  \item 单调减函数
  \item 周期函数
  \end{itemize}

\item 设函数\(\,f, g\)都在\(\R\)上单调递增.下列函数中,在\(\R\)上单调递增的是\uline{\makebox[3em]{}}.
  \begin{itemize}
    \renewcommand{\labelitemi}{\faCircleThin}
  \item \(\,f(x) \cdot g(x)\)
    \ifshowsol
    \item[\faCircle] \(\,f(\,g(x))\)
    \else
    \item \(\,f(\,g(x))\)
    \fi
  \item \(\,f(-g(x))\)
  \item \(\,f(\,g(-x))\)
  \end{itemize}

\item 在\(\R\)上,\(\,f\mkern2mu\)为奇函数,\(\mkern1mu g\)为偶函数,则下列函数为奇函数的是\uline{\makebox[3em]{}}.
  \begin{itemize}
    \renewcommand{\labelitemi}{\faCircleThin}
  \item \(\,f(\,g(x))\)
  \item \(g(f(x))\)
    \ifshowsol
    \item[\faCircle] \(\,f(\,f(x))\)
    \else
    \item \(\,f(\,f(x))\)
    \fi
  \item \(g(\,g(x))\)
  \end{itemize}

\item 用\(y = \floor x\)表示取整函数.设\(\,f(x) = \cos\paren{\frac\pi2 \floor x}\).那么\(\,f\mkern2mu\)是\uline{\makebox[4em]{}}.
  \begin{itemize}
    \renewcommand{\labelitemi}{\faCircleThin}
  \item 无界的奇函数
  \item 有界的偶函数
  \item 周期函数和偶函数
    \ifshowsol
    \item[\faCircle] 有界的周期函数
    \else
    \item 有界的周期函数
    \fi
  \end{itemize}

\item 下列函数中,哪一个是无界函数?
  \begin{itemize}
    \renewcommand{\labelitemi}{\faCircleThin}
  \item \(\,f(x) = \frac{x^2 + \sqrt{1+x^2}}{2 + x^2},\ x \in \R\)
  \item \(\,f(x) = \sgn x \cdot \sin \frac1x,\ x \ne 0\)
  \item \(\,f(x) = \frac{\floor x}{x},\ x > 0\)
    \ifshowsol
    \item[\faCircle] \(\,f(x) = \frac{x}{\ln x},\ x \in \paren{0,+\infty}\)
    \else
    \item \(\,f(x) = \frac{x}{\ln x},\ x \in \paren{0,+\infty}\)
    \fi
  \end{itemize}

\item 函数\(\,f\,\)和\(\mkern2mu g\mkern1mu\)是在\(\paren{a,b}\)上的单增函数.那么\(h_1(x) = \maxb{\,f(x), g(x)},\ h_2(x) = -\minb{\,f(x), g(x)}\)分别为\uline{\makebox[6em]{}}.
  \begin{itemize}
    \renewcommand{\labelitemi}{\faCircleThin}
  \item 单增函数,单增函数
    \ifshowsol
    \item[\faCircle] 单增函数,单减函数
    \else
    \item 单增函数,单减函数
    \fi
  \item 单减函数,单增函数
  \item 单减函数,单减函数
  \end{itemize}

\item 设函数\(\,f\mkern2mu\)的定义域是\(\paren{0,+\infty}\).若\(\,f(x)/x\)单调递减且\(a > 0,\ b > 0\),则\uline{\makebox[6em]{}}.
  \begin{itemize}
    \renewcommand{\labelitemi}{\faCircleThin}
  \item \(\,f(a+b) < f(a)\)
    \ifshowsol
    \item[\faCircle] \(\,f(a+b) \le f(a) + f(b)\)
    \else
    \item \(\,f(a+b) \le f(a) + f(b)\)
    \fi
  \item \(\,f(a+b) \le a + b\)
  \item \(\,f(a+b) > f(a) + f(b)\)
  \end{itemize}

  \ifshowsol
    \begin{proof}
      不失一般地,假设\(a \le b\).那么就有\(\,f(b)/\,b \le \,f(a)/\,a\),从而
      \begin{equation*}
        f(a+b)
        < \frac{a+b}{b} \,f(b) \\
        = \paren{a/b + 1} \,f(b) \\
        \le \,f(a) + f(b). \qedhere
      \end{equation*}
    \end{proof}
    函数\(\sqrt x + x\)是选项A和C的反例.选项B和D互相矛盾.
  \fi
\end{enumerate}
\fi

\section{初等函数}

\begin{definition*}
  常函数、幂函数、指数函数、对数函数、三角函数、反三角函数,这六类函数叫作基本初等函数.
\end{definition*}

% http://projecteuclid.org/euclid.pjm/1102991609
% https://muleshko.faculty.unlv.edu/handouts/Elementary%20Functions%20(1).pdf
\begin{definition*}
  由基本初等函数经过有限次四则运算或函数复合得到的函数叫作初等函数\footnote{初等函数的概念最早是由法国数学家Joseph Liouville在讨论不定积分的代数解时引入的.}.
\end{definition*}

\begin{remark}
  常见的非初等函数有:分段函数、隐函数、参数方程确定的函数、变限定积分函数、参变量积分函数、函数项级数的和函数.
\end{remark}

\begin{definition*}
  双曲正弦、双曲余弦、双曲正切函数的定义分别是
  \[
    \sinh x \coloneq \frac{e^x-e^{-x}}{2},
    \quad
    \cosh x \coloneq \frac{e^x+e^{-x}}{2},
    \quad
    \tanh x \coloneq \frac{\sinh x}{\cosh x}.
    \rule[-2ex]{0pt}{0pt}
  \]
  其余类推.
\end{definition*}

\begin{remark}
  关于双曲函数,常见的关系式有:
  \begin{gather*}
    \cosh^2 x - \sinh^2 x = 1,
    \quad
    \sech^2 x + \tanh^2 x = 1, \\
    \cosh(x+y) = \cosh x \cosh y + \sinh x \sinh y, \\
    \sinh(x+y) = \sinh x \cosh y + \cosh x \sinh y.
  \end{gather*}
\end{remark}

\begin{remark}
  双曲线的一个分支可以用含双曲函数的参数方程来表示:
  \[
    \setlength{\abovedisplayskip}{.8ex}
    \setlength{\belowdisplayskip}{.8ex}
    x = a \cosh t,
    \quad
    y = b \sinh t.
  \]
\end{remark}

\hypertarget{defn:inversehyper}{}
\begin{definition*}
  双曲函数的反函数有:
  \[
    \addtolength{\jot}{.8ex}
    \begin{split}
      \arccosh x &\coloneq \lnp[\big]{x + \sqrt{x^2-1}}
      \quad\text{对于}\quad
      x \ge 1, \\
      \arcsinh x &\coloneq \lnp[\big]{x + \sqrt{x^2+1}}
      \quad\text{对于}\quad
      x \in \R, \\
      \arctanh x
      &\coloneq \frac12 \ln\frac{1+x}{1-x}
      = \frac12\brkt*{\ln(1+x) - \ln(1-x)}
      \quad\text{对于}\quad
      x^2 < 1, \\
      \arccoth
      &\coloneq \frac12 \ln\frac{1+x}{x-1}
      = \frac12 \brkt[\Big]{\lnp[\Big]{1+\frac1x} - \lnp[\Big]{1-\frac1x}}
      \quad\text{对于}\quad
      x^2 > 1.
      \rule[-2.5ex]{0ex}{0ex}
    \end{split}
  \]
\end{definition*}

\subpdfbookmark{思考}{B1.1.5.P}
\subsection*{思考}

\begin{enumerate}
\item 画出三角函数、反三角函数、双曲函数、反双曲函数的图像.

  \ifshowsolp
    \begin{figure}[H]
      \centering
      \begin{minipage}{2.5in}
        \tikzsetnextfilename{B1.1.5.P.1.1}
        \begin{tikzpicture}[smooth,domain=-6.6:6.6,samples=50,scale=0.4,font=\small]
          \draw[->] (-8,0) -- (8,0) node[right] {\(x\)};
          \draw[->] (0,-8) -- (0,8) node[above] {\(y\)};
          \draw[color=Dark2-A] plot (\x,{sin(\x r)}) node[right,color=.,yshift=.8ex] {\(y = \sin x\)};
          \draw[color=Dark2-B] plot (\x,{cos(\x r)}) node[right,color=.,yshift=1ex] {\(y = \cos x\)};
          \draw[color=Dark2-C,domain=-pi-1.4:-pi+1.4] plot (\x,{tan(\x r)});
          \draw[dotted] (-.5*pi,-6) -- (-.5*pi,6);
          \draw[color=Dark2-C,domain=-1.4:1.4] plot (\x,{tan(\x r)});
          \draw[dotted] (.5*pi,-6) -- (.5*pi,6);
          \draw[color=Dark2-C,domain=pi-1.4:pi+1.4] plot (\x,{tan(\x r)}) node[right,color=.] {\(y = \tan x\)};
        \end{tikzpicture}
        \caption*{三角函数}
      \end{minipage}
      \hfill
      \begin{minipage}{2.5in}
        \tikzsetnextfilename{B1.1.5.P.1.2}
        \begin{tikzpicture}[smooth,domain=-1:1,samples=50,scale=.8,font=\small]
          \draw[->] (-4,0) -- (4,0) node[right] {\(x\)};
          \draw[->] (0,-4) -- (0,4) node[above] {\(y\)};
          \draw[color=Dark2-A] plot (\x,{rad(asin(\x))}) node[anchor=south west,color=.,xshift=-2.2ex] {\(y = \arcsin x\)};
          \draw[color=Dark2-B] plot (\x,{rad(acos(\x))}) node[above,color=.,xshift=-1.2ex] at (-1,pi) {\(y = \arcsin x\)};
          \draw[dotted] (-3.5,-.5*pi) -- (3.5,-.5*pi);
          \draw[color=Dark2-C,domain=-3.5:3.5] plot (\x,{rad(atan(\x))}) node[anchor=north west,color=.,xshift=-2ex] {\(y = \arctan x\)};
          \draw[dotted] (-3.5,.5*pi) -- (3.5,.5*pi);
        \end{tikzpicture}
        \caption*{反三角函数}
      \end{minipage}
      \hspace*{.5in}
    \end{figure}
  \begin{figure}[H]
    \centering
    \begin{minipage}{2.5in}
      \tikzsetnextfilename{B1.1.5.P.1.3}
      \begin{tikzpicture}[smooth,domain=-2.7:2.7,samples=50,scale=0.4,font=\small]
        \draw[->] (-8,0) -- (8,0) node[right] {\(x\)};
        \draw[->] (0,-8) -- (0,8) node[above] {\(y\)};
        \draw[color=Dark2-A] plot (\x,{sinh(\x)}) node[anchor=south east,color=.] at (-2.7,{sinh(-2.7)}) {\(y = \sinh x\)};
        \draw[color=Dark2-B] plot (\x,{cosh(\x)}) node[anchor=north east,color=.] at (-2.7,{cosh(-2.7)}) {\(y = \cosh x\)};
        \draw[dotted] (-3.8,-1) -- (3.8,-1);
        \draw[color=Dark2-C,domain=-3.8:3.8] plot (\x,{tanh(\x)}) node[right,color=.] {\(y = \tanh x\)};
        \draw[dotted] (-3.8,1) -- (3.8,1);
      \end{tikzpicture}
      \caption*{双曲函数}
    \end{minipage}
    \hfill
    \begin{minipage}{2.5in}
      \tikzsetnextfilename{B1.1.5.P.1.4}
      \begin{tikzpicture}[smooth,domain=-7.4:7.4,samples=50,scale=0.4,font=\small]
        \draw[->] (-8,0) -- (8,0) node[right] {\(x\)};
        \draw[->] (0,-8) -- (0,8) node[above] {\(y\)};
        \draw[color=Dark2-A] plot (\x,{ln(\x+sqrt(\x*\x+1))}) node[below,color=.,xshift=2ex] at (-7.4,-2.7) {\(y = \arcsinh x\)};
        \draw[color=Dark2-B,domain=1:7.4] plot (\x,{ln(\x+sqrt(\x*\x-1))}) node[anchor=south west,color=.,xshift=1ex] at (1,0) {\(y = \arccosh x\)};
        \draw[dotted] (-1,-3.8) -- (-1,3.8);
        \draw[color=Dark2-C,domain=-.999:.999] plot (\x,{ln((1+\x)/(1-\x))/2}) node[anchor=south west,color=.,xshift=-1.5ex] {\(y = \arctanh x\)};
        \draw[dotted] (1,-3.8) -- (1,3.8);
      \end{tikzpicture}
      \caption*{反双曲函数}
    \end{minipage}
    \hspace*{.5in}
  \end{figure}
  \fi

\item 能否写出反双曲函数与对数函数的关系式.

  \ifshowsolp
    参见反双曲函数的\hyperlink{defn:inversehyper}{定义}.
  \fi
\end{enumerate}

\ifshowex
\currentpdfbookmark{练习}{B1.1.5.E}
\subsection*{练习}

\begin{enumerate}
\item 下列各组函数中,两个函数相同的一组是\uline{\makebox[6em]{}}.
  \begin{itemize}
    \renewcommand{\labelitemi}{\faCircleThin}
  \item \(y = \arcsin(\cos x),\ y = \arccos(\sin x)\)
    \ifshowsol
    \item[\faCircle] \(y = \sin(\arccos x),\ y = \cos(\arcsin x)\)
    \else
    \item \(y = \sin(\arccos x),\ y = \cos(\arcsin x)\)
    \fi
  \item \(y = \arctan x,\ y = \arccot\frac1x\)
  \item \(y = \sin(\arcsin x),\ y = \tan(\arctan x)\)
  \end{itemize}

  \ifshowsol
    选项~A中的两个函数,值域不同、相位也不同.它们可以用绝对值函数和取整函数表示成
    \begin{gather*}
      \arcsin(\cos x) = \frac\pi2 - \abs*{\,x - 2\pi\floor[\bigg]{\frac{x}{2\pi} + \frac12}\,} \\
      \siand
      \arccos(\sin x) = \abs*{\,x - 2\pi \floor[\bigg]{\frac{x}{2\pi} + \frac14} - \frac\pi2\,}.
    \end{gather*}
    选项~B中的两个函数相同,都可以表示成
    \begin{equation*}
      \sin(\arccos x) = \cos(\arcsin x) = \sqrt{1-x^2}.
    \end{equation*}
    按照我们这里给出的反余切函数定义和主值区间\footnote{在Wolfram语言当中,反余切函数的定义和主值区间\cite{ArccotMathWorld}与此处不同.按照他们的定义,选项~C中的两个函数仍然不同,但只有定义域不同了.},选项~C中的两个函数不同,并且有
    \begin{equation*}
      \arctan x =
      \begin{cases}
        \arccot \frac1x,       & x > 0, \\
        0,                     & x = 0, \\
        \arccot \frac1x - \pi, & x < 0.
      \end{cases}
    \end{equation*}
    选项~D中,函数\(\sin(\arcsin x)\)的定义域是\(\brkt{-1,1}\),函数\(\tan(\arctan x)\)的定义域是\(\R\).
  \fi

\item 函数\(y = \cos x\ (\pi \le x \le 2\pi)\)的反函数是\uline{\makebox[6em]{}}.
  \begin{itemize}
    \renewcommand{\labelitemi}{\faCircleThin}
  \item \(y = \pi + \cos x\)
  \item \(y = \frac52\pi - \arcsin x\)
    \ifshowsol
    \item[\faCircle] \(y = \frac32\pi + \arcsin x\)
    \else
    \item \(y = \frac32\pi + \arcsin x\)
    \fi
  \item \(y = \pi - \arccos x\)
  \end{itemize}

  \ifshowsol
    它的反函数是
    \begin{equation*}
      2\pi - \arccos x
      = 2\pi - \paren[\bigg]{\frac\pi2 - \arcsin x}
      = \frac32\pi + \arcsin x.
    \end{equation*}
  \fi

\item 设\(\,f\mkern2mu\)是在\(\R\)上的减函数.若\(a + b \le 0\),则\uline{\makebox[6em]{}}.
  \begin{itemize}
    \renewcommand{\labelitemi}{\faCircleThin}
  \item \(\,f(a) + f(b) \le -\paren[\big]{\,f(a) + f(b)}\)
  \item \(\,f(a) + f(b) \le f(-a) + f(-b)\)
  \item \(\,f(a) + f(b) \ge -\paren[\big]{\,f(a) + f(b)}\)
    \ifshowsol
    \item[\faCircle] \(\,f(a) + f(b) \ge f(-a) + f(-b)\)
    \else
    \item \(\,f(a) + f(b) \ge f(-a) + f(-b)\)
    \fi
  \end{itemize}

  \ifshowsol
    选项~A和~C都可以通过举反例来证伪,比方说\(\,f(x) = \arccot x\)证伪了~A,\(\,f(x) = -\pi + \arccot x\)证伪了~C.

    对于~B和~D,因为\(a \le -b\)且\(b \le -a\),所以\(\,f(a) \ge f(-b)\)且\(\,f(b) \ge f(-a)\),从而
    \begin{equation*}
      f(a) + f(b) \ge f(-a) + f(-b).
    \end{equation*}
  \fi

\item 曲线\(y = \frac{2^x-2^{-x}}{2^x+2^{-x}}\)与曲线\(y = f(x)\)的图像关于直线\(y = x\)对称.求\(\,f(x) =\)
  \ifshowsol
    {\setlength{\ULdepth}{1.15ex}%
      \uline{\makebox[6em]{\(\frac12 \log_2\frac{1+x}{1-x}\)}}}.

    推导过程与\(\arctanh x\)类似.
  \else
    \uline{\makebox[6em]{}}.
  \fi

\item 设
  \[
    f(x) =
    \begin{cases}
      \log_2 (4-x), & x \le 0, \\
      \,f(x-1) - f(x-2), & x > 0.
    \end{cases}
  \]
  求\(\,f(3) =\)\uline{\makebox[3em]{\ifshowsol\(-2\)\fi}}.

\item 下列恒等式中,不正确的是\uline{\makebox[6em]{}}.
  \begin{itemize}
    \renewcommand{\labelitemi}{\faCircleThin}
  \item \(\sinh(x \pm y) = \sinh x \cosh y \pm \cosh x \sinh y\)
  \item \(\arcsinh x = \lnp[\big]{x + \sqrt{1+x^2}}\)
  \item \(\tanh x = \frac{e^x-e^{-x}}{e^x+e^{-x}}\)
    \ifshowsol
    \item[\faCircle]
    \else
    \item
    \fi
    \(\arctanh x = \frac12 \ln\frac{1-x}{1+x}\)
  \end{itemize}
\end{enumerate}
\fi

\section{极坐标方程与参数方程表示的几种曲线}

\begin{definition*}
  平面上的任意一点,除了用直角坐标来表示,也可以用极坐标来表示.极坐标由半径坐标\(r\)和角坐标\(\theta\)构成,其与直角坐标的关系可以用等式
  \[
    x = r \cos\theta \quad \text{和} \quad y = r \sin \theta
  \]
  来确定\footnote{极坐标的概念最早可以追溯到古希腊,文艺复兴以来通过类似极坐标变换的方法来求阿基米德螺线面积的方法,最早可以追溯到Cavalieri和Gregory of St.~Vincent.}\textsuperscript{,}\footnote{有些人也会用\(\rho\)来表示半径坐标,用\(\phi,\ \varphi\)或\(t\)来表示角坐标.角坐标又叫作极角.}.
  其中,半径坐标\(r\)表示点到原点的距离,角坐标\(\theta\)表示点到\(x\)轴正方向上的逆时针角度.
\end{definition*}

一般来说,若不加任何限制,平面上一点的极坐标不唯一.可以通过
\[
  r = \sqrt{x^2 + \smash{y^2}} \quad \text{和} \quad \theta = \atantwo(y,x)
\]
将直角坐标转换成极坐标.

\begin{definition*}
  圆的极坐标方程是
  \[
    r = a,
  \]
  其中\(a\)为圆的半径.
\end{definition*}

\begin{definition*}
  心形线\footnote{心形线的概念最早是由Ole Rømer于1674年在研究齿轮轮齿的最优形状时提出来的.}(cardioid)的极坐标方程是
  \[
    \setlength{\abovedisplayskip}{.8ex}
    \setlength{\belowdisplayskip}{.8ex}
    r = 2a (1-\cos\theta).
  \]
\end{definition*}

\begin{remark}
  这个心形线的开口向着\(x\)轴的正方向,如下图.
\end{remark}

\begin{figure}[H]
  \centering
  \tikzsetnextfilename{B1.1.6.1}
  \begin{tikzpicture}[domain=0:360,smooth,samples=50]
    \draw[->] (-3,0) -- (3,0) node[right] {\(x\)} node[left] at (-3,0) {\hphantom{\(x\)}};
    \draw[->] (0,-3) -- (0,3) node[above] {\(y\)};
    \draw[color=Dark2-C] (-1,0) circle (1);
    \draw[color=Dark2-B] plot (\x:{2*(1-cos(\x))});
    \path plot (\x:{2*(1+cos(\x))});
    \draw[dotted] (1,0) circle (1);
    \draw[dotted] ({2*cos(70)-1},{2*sin(70)}) circle (1);
    \draw[dotted] ({2*cos(135)-1},{2*sin(135)}) circle (1);
    \draw[dotted] ({2*cos(210)-1},{2*sin(210)}) circle (1);
    \fill[color=Dark2-D] (0,0) circle (1pt)
    (70:{2*(1-cos(70))}) circle (1pt)
    (135:{2*(1-cos(135))}) circle (1pt)
    (210:{2*(1-cos(210))}) circle (1pt);
  \end{tikzpicture}
  \caption*{心形线}
\end{figure}

\begin{definition*}
  伯努利双纽线\footnote{Jacob Bernoulli于1694年在\textit{Acta Eruditorum}上发表的一篇文章描述了这种曲线.}(lemniscate of Bernoulli)的极坐标方程是
  \[
    r^2 = a^2 \cos 2\theta, \quad a = \sqrt2\,c.
  \]
  其中,\(c\)是焦点到原点的距离,\(a\)是双纽线的半宽度.
\end{definition*}

\begin{figure}[H]
  \centering
  \tikzsetnextfilename{B1.1.6.2}
  \begin{tikzpicture}[smooth,samples=50]
    \draw[->] (-3,0) -- (3,0) node[right] {\(x\)} node[left] at (-3,0) {\hphantom{\(x\)}};
    \draw[->] (0,-3) -- (0,3) node[above] {\(y\)};
    \draw[color=Dark2-C] plot[domain=-45:45] (\x:{2*sqrt(2)*sqrt(cos(2*\x))})
    plot[domain=135:225] (\x:{2*sqrt(2)*sqrt(cos(2*\x))});
    \draw[dashed] (-2,0) -- (20:{2*sqrt(2)*sqrt(cos(2*20))}) -- (2,0);
    \fill[color=Dark2-D]
    (-2,0) circle (1pt)
    node[below,color=.] {\(F_1\)}
    (2,0) circle (1pt)
    node[below,color=.] {\(F_2\)}
    (20:{2*sqrt(2)*sqrt(cos(2*20))}) circle (1pt)
    node[above,color=.] {\(P\)};
    \node at (-1.75,2) {\(PF_1 \cdot PF_2 = c^2\)};
  \end{tikzpicture}
  \caption*{伯努利双纽线}
\end{figure}

\begin{definition*}
  阿基米德螺线\footnote{阿基米德(Archimedes of Syracuse)在公元前约225年的著作《论螺线》上研究了这种由Conon发现的曲线.}(Archimedean spiral)的参数方程是
  \[
    \setlength{\abovedisplayskip}{.8ex}
    \setlength{\belowdisplayskip}{.8ex}
    x = at \cos t, \quad y = at \sin t.
  \]

  \begin{remark}
    实际上,其极坐标方程更为简洁:\(r = a \theta\).
  \end{remark}
\end{definition*}

\begin{figure}[H]
  \centering
  \tikzsetnextfilename{B1.1.6.3}
  \begin{tikzpicture}[smooth,samples=50]
    \draw[->] (-4,0) node[left] {\hphantom{\(x\)}} -- (4,0) node[right] {\(x\)};
    \draw[->] (0,-4) -- (0,4) node[above] {\(y\)};
    \draw[dotted] plot[domain=0:-1080] (\x:{.2*rad(\x)});
    \draw[color=Dark2-C] plot[domain=0:1080] (\x:{.2*rad(\x)});
  \end{tikzpicture}
  \caption*{阿基米德螺线}
\end{figure}

\begin{definition*}
  摆线\footnote{旋轮线的概念是由Charles de Bovelles于1501年在做化圆为方的时候提出来的.伽利略最早严肃地研究了这种曲线.笛卡尔、费马、帕斯卡、惠更斯、莱布尼茨、伯努利兄弟等17世纪的数学大师们都研究过这种曲线,还由此生出种种龃龉.}(又称旋轮线、圆滚线)(cycloid)的参数方程是
  \[
    \setlength{\abovedisplayskip}{.8ex}
    x = a (t - \sin t), \quad y = a (1 - \cos t).
  \]
\end{definition*}

\begin{figure}[H]
  \centering
  \tikzsetnextfilename{B1.1.6.4}
  \begin{tikzpicture}[smooth,samples=50]
    \draw (-1,0) -- (4*pi+1,0);
    \foreach \x in {0,.8,1.6,2.4,3.2,4} {
      \pgfmathsetmacro\theta{\x*pi}
      \draw[dotted] (\theta,1) circle (1);
      \draw ({\theta - sin(\theta r)},{1 - cos(\theta r)}) -- (\theta,1);
      \fill[color=Dark2-D]
      ({\theta - sin(\theta r)},{1 - cos(\theta r)}) circle (1pt)
      (\theta,1) circle (1pt);
    }
    \draw[color=Dark2-C,domain=0:4*pi] plot ({\x - sin(\x r)},{1 - cos(\x r)});
  \end{tikzpicture}
  \caption*{旋轮线}
\end{figure}

\begin{definition*}
  星形线\footnote{星形线的概念最早也是由Ole Rømer于1674年在研究齿轮轮齿的最优形状时提出来的.后来的伯努利父子、莱布尼茨、达朗贝尔都研究过这种曲线.}(astroid)的参数方程是
  \[
    \left\{
      \addtolength{\jot}{1ex}
      \begin{alignedat}{2}
        x &= a \cos^3 t &&= \frac a4 \paren{3 \cos t + \cos 3t}, \\
        y &= a \sin^3 t &&= \frac a4 \paren{3 \sin t - \sin 3t}.
      \end{alignedat}
    \right.
  \]
\end{definition*}

\begin{remark}
  莱布尼茨1715年给出了此曲线的标准方程:\(x^{2/3} + y^{2/3} = a^{2/3}\).
\end{remark}

\begin{figure}[H]
  \centering
  \tikzsetnextfilename{B1.1.6.5}
  \begin{tikzpicture}[smooth,samples=50]
    \draw[color=Dark2-C] circle (4);
    \foreach \x in {0,1,2,3,4,5,-3,-2,-1} {
      \pgfmathsetmacro\theta{\x*acos(7/9)}
      \draw[dotted] ({3*cos(\theta)},{3*sin(\theta)}) circle (1);
      \draw ({3*cos(\theta)},{3*sin(\theta)})
      -- ({4*pow(cos(\theta),3)},{4*pow(sin(\theta),3)});
      \fill[color=Dark2-D]
      ({3*cos(\theta)},{3*sin(\theta)}) circle (1pt)
      ({4*pow(cos(\theta),3)},{4*pow(sin(\theta),3)}) circle (1pt);
    }
    \draw[color=Dark2-B,domain=0:360]
    plot ({4*pow(cos(\x),3)},{4*pow(sin(\x),3)});
  \end{tikzpicture}
  \caption*{星形线}
\end{figure}

% Local Variables:
% TeX-engine: luatex
% TeX-master: "微积分B"
% LaTeX-indent-begin-exceptions-list: ("ifshowex")
% End:
