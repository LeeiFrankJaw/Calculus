\chapter{原函数与不定积分}

\vskip-1.25em
\section{概念与性质}

\textbf{定义:}若函数$f(x)$是函数$ F(x) $在开区间$ \left(a, b\right) $上的导函数,
则称函数$F(x)$为函数$f(x)$在$(a,b)$上的一个原函数.

\textbf{例:} $ f(x) = \sin x $是在$\mathbb{R}$上的一个函数,则函数$ F(x) = -\cos x $是
$ f(x) $在$ \mathbb{R} $上的一个原函数,因为$ f(x) = F'(x) $.

\begin{table*}[h]
  \caption*{导数表}
  \centering
  \begin{tabular}{ l l }
    函数		&	导函数 \\
    $F(x)$		&	$f(x)$ \\
    $x^p$		&	$px^{p-1}$ \\
    $e^x$		&	$e^x$ \\
    $a^x$		&	$a^x \cdot \ln a$ \\
    $\ln |x|$	&	$\frac{1}{x}$
  \end{tabular}
  \begin{tabular}{ l l }
    函数			&	导函数 \\
    $F(x)$			&	$f(x)$ \\
    $\log_a |x|$	&	$\frac{1}{x \cdot \ln a}$ \\
    $\sin x$		&	$\cos x$ \\
    $\cos x$		&	$-\sin x$ \\
    $\tan x$		&	$\sec^2 x$
  \end{tabular}
  \begin{tabular}{ l l }
    函数		&	导函数 \\
    $F(x)$		&	$f(x)$ \\
    $\arcsin x$	&	$\frac{1}{\sqrt{1-x^2}}$ \\
    $\arccos x$&	$-\frac{1}{\sqrt{1-x^2}}$ \\
    $\arctan x$&	$\frac{1}{1+x^2}$ \\ {}
  \end{tabular}
\end{table*}

\begin{table*}[h]
  \caption*{原函数表}
  \centering
  \begin{tabular}{ l l }
    函数			&	原函数 \\
    $f(x)$			&	$F(x)$ \\
    $x^p(p\ne-1)$	&	$\frac{1}{p+1} x^{p+1}$ \\
    $e^x$			&	$e^x$ \\
    $a^x$			&	$\frac{a^x}{\ln a}$ \\
    $\frac{1}{x}$	&	$\ln |x|$
  \end{tabular}
  \begin{tabular}{ l l }
    函数						&	原函数 \\
    $f(x)$						&	$F(x)$ \\
    $\sin x$					&	$-\cos x$ \\
    $\cos x$					&	$\sin x$ \\
    $\sec^2 x$					&	$\tan x$ \\
    $\frac{1}{\sqrt{1-x^2}}$	&	$\arcsin x$
  \end{tabular}
  \begin{tabular}{ l l }
    函数						&	原函数 \\
    $f(x)$						&	$F(x)$ \\
    $-\frac{1}{\sqrt{1-x^2}}$	&	$\arccos x$ \\
    $\frac{1}{1+x^2}$			&	$\arctan x$ \\ \\ {}
  \end{tabular}
\end{table*}

\textbf{问题I:}什么样的$f(x)$在$(a,b)$上存在原函数?

\textbf{答:}\parbox[t]{5in}{
  (1) $ f \in C(a,b) $,则$ f $在$(a,b)$内一定存在原函数.(下一章,即第7章)\\
  (2) $f$在$(a,b)$不连续,是否还有可能存在原函数?}

\hypertarget{eg:discontI}{}
\textbf{反例}
\begin{gather*}
  f(x) =
  \begin{cases}
    2x \sin \frac{1}{x} - \cos \frac{1}{x}, & x \ne 0 \\
    0, & x = 0
  \end{cases} \\
  F(x) =
  \begin{cases}
    x^2 \sin \frac{1}{x}, & x \ne 0 \\
    0, & x = 0
  \end{cases}
\end{gather*}
$F(x)$在$\mathbb{R}$上是可导函数, 所以有$F'(x) = f(x)$, \ $x \in \mathbb{R}$.

\textbf{问题II:}什么样的函数$f(x)$在$(a,b)$上没有原函数?

\textbf{答:}要回答这个问题, 我们要先回顾一下微分学中学过的Darboux定理.

\hangpar{Darboux定理:}{
  若$F(x)$在$[a,b]$上可导(在$a$点右导数存在, 在b点左导数存在)且$F'(a) = \alpha$,
  $F'(b) = \beta$, $\alpha \ne \beta$, 则对于任何介于$\alpha$, $\beta$的实数$\eta$,
  存在$\xi \in (a,b) $使得$F'(\xi) = \eta$. (导数的介值定理) }

Darboux定理的逆否命题就说明:不满足介值性质的函数没有原函数.

\textbf{例:}若$f(x)$在$(a,b)$上有第一类间断点, 则$f(x)$在$(a,b)$不满足介值定理,
从而在$(a,b)$上没有原函数.
\[f(x) = \begin{cases}
  1, & x \ge 0 \\
  -1, & x < 0
\end{cases}\]
$x = 0$是第一类间断点, $f(x)$在$(a,b)\ (a<0, b>0)$上没有原函数.

\textbf{例:}
\[f(x) = \begin{cases}
  2x \sin \frac{1}{x} - \cos \frac{1}{x} + 2, & x \ne 0 \\
  -2, & x = 0
\end{cases}\]
$x=0$是第二类间断点, $f(x)$在$(a,b)\ (a<0, b>0)$上不存在原函数. (cf.~\hyperlink{eg:discontI}{反例})

\textbf{问题III:}若$f(x)$在$(a,b)$上有原函数, 有几个原函数?

\textbf{答:}有无数个.

\hangpar{不同原函数之间的关系:}{
  若$F(x)$是$f(x)$在$(a,b)$内的一个原函数, 则$F(x)+C$均为$f(x)$的原函数, 并且$f(x)$的
  所有原函数构成的集合为$\{F(x)+C\}$, 其中$C \in \mathbb{R}$为任意常数.}

\textbf{证明:}

\vspace{-6pt}
(i) 若$F(x)$是$f(x)$在$(a,b)$上的一个原函数, 则$F'(x)=f(x)$, \ $x\in (a,b)$. 从而$[F(x)+C]' = f(x)$,
\ $x\in (a,b)$, 所以$F(x)+C$都是$f(x)$的原函数.

(ii) 若$G(x)$是$f(x)$在$(a,b)$上的一个原函数, 则$G'(x)=f(x)=F'(x)$, $x\in (a,b)$, 从而$[G(x) - F(x)]' = 0$,
$x \in (a, b)$, 这就意味着 $G(x) - F(x) = C$, 所以$ G(x) = F(x) +C$, $x \in (a, b)$. \qed

我们称$\{F(x)+C\}$为原函数族, 只要找一个原函数为代表, 就能表示所有的原函数.

\textbf{定义:}我们把原函数族称为$f(x)$的不定积分. 记作
\[\{F(x)+C\} = \int f(x) \,dx.\]
\begin{alignat*}{3}
  F&(x) &&\xrightarrow{\text{求导}} &&f(x) \; \text{导函数} \\
  F(x) &+ C &&\xleftarrow[\text{不定积分}]{} &&f(x) \\
   &\veq \\
  \int f(&x) \, dx && &&\text{互为逆过程}
\end{alignat*}

通过上面的原函数表, 给每一个函数加上一个$C$, 就可以构成不定积分表.

\hangpar{不定积分的性质}{
  (1) 若$f,g \in R[a,b]$, 则$\int [f(x)+g(x)] \,dx = \int f(x) \,dx + \int g(x) \,dx$\,; \\
  (2) 若$f \in R[a,b]$, $\lambda \in \mathbb{R}$, 则$\int [\lambda f(x)] \,dx = \lambda \int f(x) \,dx$.}

\begin{proof}
  \textbf{对(1)的证明:}

  \vspace{-6pt}
  (i) 因为$ (F(x) + G(x))' = F'(x) + G'(x) = f(x) + g(x) $, 所以$ F(x) + G(x) $是$ f(x) + g(x) $的一个原函数.

  (ii)
  \begin{align*}
    \left( \int [f(x) + g(x)] \,dx \right)' = f(x) &+ g(x) \\
                                                   & \veq \\
    \left( \int f(x) \,dx + \int g(x) \,dx \right)' = f(x) &+ g(x) \\
    \intertext{所以}
    \int [f(x) + g(x)] \,dx = \int f(x) \,dx &+ \int g(x) \,dx. \qedhere
  \end{align*}
\end{proof}

\begin{proof}
  \textbf{对(2)的证明:}

  \vspace{-6pt}
  (i) 因为$ (\lambda F(x))' = \lambda F'(x) = \lambda f(x) $, 所以$ \lambda F(x) $是$ \lambda f(x) $的一个原函数.

  (ii)
  \begin{align*}
    \left( \int \lambda f(x) \,dx \right)' = \lambda &f(x) \\
                                                     & \veq \\
    \left( \lambda \int f(x) \,dx \right)' = \lambda &f(x) \\
    \intertext{所以}
    \int \lambda f(x) \,dx = \lambda &\int f(x) \,dx. \qedhere
  \end{align*}
\end{proof}

性质(1)叫做加法法则, 性质(2)叫做数乘法则, 同时使用这两个法则得到:
\[ \int [\lambda f(x) + \mu g(x)] \,dx = \int \lambda f(x) \,dx + \int \mu g(x) \,dx
  = \lambda \int f(x) \,dx + \mu \int g(x) \,dx. \]

所谓的减法法则, 可以看成上式的特例:
\[ \int [f(x) - g(x)] \,dx = \int f(x) \,dx - \int g(x) \,dx \quad (\lambda = 1, \mu = -1). \]

\textbf{例1:}$ \displaystyle \int (3x^2 - 2^x) \,dx. $
\begin{align*}
  \int (3x^2 - 2^x) \,dx
  &= 3 \int x^2 \,dx - \int 2^x \,dx \\
  &= 3 \left(\frac{x^3}{3} + C\right) - \left(\frac{2^x}{\ln 2} + C\right) \\
  &= x^3 - \frac{2^x}{\ln 2} + {\color{red} C}.
\end{align*}

\textbf{例2:}$ \displaystyle \int \frac{x^4}{1+x^2} \, dx. $

将这个代入上式
\begin{align*}
  x^4  = (x^4 -1) + 1
  &= (x^2-1)(x^2+1) + 1, \\
  \intertext{得}
  \int \frac{x^4}{1+x^2} \, dx
  &= \int \frac{(x^2-1)(x^2+1) + 1}{1+x^2} \, dx \\
  &= \int \left(x^2 -1 + \frac{1}{1+x^2}\right) \, dx \\
  & = \int x^2 \, dx - \int dx + \int \frac{1}{1+x^2} \, dx \\
  & = \frac{x^3}{3} - x + \arctan x + C.
\end{align*}

\textbf{例3:}$ \displaystyle \int \frac{\cos 2x}{\cos x - \sin x} \, dx. $

将倍角公式代入
\begin{align*}
  \cos 2x = \cos^2 x - \sin^2 x
  &= (\cos x + \sin x)(\cos x - \sin x) \\
  \int \frac{(\cos x + \sin x)\cancel{(\cos x - \sin x)}}{\cancel{\cos x - \sin x}} \, dx
  &= \int (\cos x + \sin x) \, dx \\
  &= \int \cos x \, dx + \int \sin x \, dx \\
  &= \sin x - \cos x + C.
\end{align*}

\textbf{例4:}$ \displaystyle \int \lvert x - 1 \rvert \, dx. $
\begin{align*}
  \lvert x - 1 \rvert
  &= \begin{cases}
    x - 1, & x \ge 1 \\
    1 - x, & x < 1
  \end{cases} \\
  \int \lvert x - 1 \rvert \, dx &= \begin{cases}
    \int (x - 1) \, dx, & x \ge 1 \\
    \int (1 - x) \, dx, & x < 1
  \end{cases} \\
  &= \begin{cases}
    \frac{x^2}{2} - x + C_1, & x \ge 1 \\
    x - \frac{x^2}{2} + C_2, & x < 1
  \end{cases} \\
  \intertext{为了保证$\int \lvert x - 1 \rvert \, dx$在$x=1$处可导, 则必须保证在此处连续. 所以}
  C_1 &= C_2 + 1 \\
  \intertext{设}
  F(x) &= \begin{cases}
    \frac{x^2}{2} - x + 1, & x \ge 1 \\
    x - \frac{x^2}{2}, & x < 1
  \end{cases} \\
  \intertext{则$ F(x) $为$ \lvert x - 1 \rvert $的一个原函数, 就有}
  \int \lvert x - 1 \rvert \, dx & = F(x) + C.
\end{align*}

\section{换元积分法}

\subsection{第一换元法\label{6.2.1}}

设$ \int f(u) \, dx = F(u) + C $且$\varphi(x) \in C^1$,
则$ \int f(\varphi(x)) \varphi'(x) \, dx = F(\varphi(x)) +C $.

\textbf{证明:}$ (\text{左边})' = f(\varphi(x)) \varphi'(x) = (\text{右边})' $. \qed

在实际使用中, 换元法体现了莱布尼茨微分记号的优越性. 将
\begin{align*}
  \varphi'(x) \, dx
  & = d\varphi(x) \\
  \intertext{代入原式, 得}
  \int f(\varphi(x)) \varphi'(x) \, dx
  & = \int f(\varphi(x)) \, d\varphi(x)
    \intertext{用$ u $代替$ \varphi(x) $}
  & = \int f(u) \, du \\
  & = F(u) + C \\
  & = F(\varphi(x)) + C.
\end{align*}

\exds{1}{ \int x \sin x^2 \dx. }

\begin{align*}
  \int x \sin x^2 \dx
  &= \frac{1}{2} \int \sin x^2 \diff(x^2)
  && \mreason{x \dx = \frac12 \diff(x^2)} \\
  &= \frac{1}{2} \int \sin u \du
  && \reason{用$u$代替$x^2$} \\
  &= -\frac{1}{2} \cos u + C \\
  &= -\frac{1}{2} \cos x^2 + C.
\end{align*}

\exds{2}{ \int \cot x \, dx. }

\begin{align*}
  \text{原式}
  &= \int \frac{\cos x}{\sin x} \, dx
  && \mreason{\cot x = \frac{\cos x}{\sin x} } \\
  &= \int \frac{1}{\sin x} \, d(\sin x)
  && \mreason{ d\sin x = \cos x \, dx } \\
  &= \int \frac{du}{u}
  && \mreason{ u = \sin x } \\
  &= \ln \abs{u} + C
  && \mreason{ d\ln \abs{u} = \frac{1}{u} \, du } \\
  &= \ln \abs[\big]{\,\sin x\,} + C.
  && \mreason{ u = \sin x }
\end{align*}

\hypertarget{eg:arctan}{}
\exds{3}{ \int \frac{dx}{a^2 + x^2} \quad (a \ne 0). }

\begin{align}
  \text{原式} &= \frac{1}{a^2} \int \frac{dx}{1+\paren*{\frac{x}{a}}^2}
  && \reason{因为$a \ne 0$, 提取因子$\frac{1}{a^2}$} \notag \\
              &= \frac{1}{a^2} \int \frac{d(au)}{1+u^2}
  && \mreason{ u = \frac{x}{a} } \notag \\
              &= \frac{1}{a} \int \frac{du}{1+u^2}
  && \mreason{ d(au) = a \, du } \notag \\
              &= \frac{1}{a}\arctan u + C
  && \mreason{ d\arctan u = \frac{1}{1+u^2} \, du } \notag \\
              &= \frac{1}{a}\arctan \frac{x}{a} + C.
  && \mreason{ u = \frac{x}{a} } \label{eq:arctanI}
\end{align}

\exds{4}{ \int \frac{dx}{\sqrt{a^2 - x^2}} \quad (a > 0). }

\begin{align*}
  \text{原式} &= \frac{1}{a} \int \frac{dx}{\sqrt{1-\paren*{\frac{x}{a}}^2}}
  && \reason{因为$a > 0$, 提取因子$\frac{1}{a}$} \\
              &= \frac{1}{a} \int \frac{d(au)}{\sqrt{1-u^2}}
  && \mreason{ u = \frac{x}{a} } \\
              &= \int \frac{du}{\sqrt{1-u^2}}
  && \mreason{ d(au) = a \, du } \\
              &= \arcsin u + C
  && \mreason{ d\arcsin u = \frac{1}{\sqrt{1-u^2}} \, du } \\
              &= \arcsin \frac{x}{a} + C.
  && \mreason{ u = \frac{x}{a} }
\end{align*}

\subsection{第二换元法\label{6.2.2}}

设$ f(x) $为连续函数, $ x = \varphi(t) $连续可导且有反函数, 则
\[ \int f(x) \, dx = \int f(\varphi(t)) \varphi'(t) \, dt. \]

若右边的原函数可求得, 记\disp{ G(t) = \int f(\varphi(t)) \varphi'(t) \, dt }, 则
\[ \int f(x) \, dx = G(\varphi^{-1}(x)) + C. \]

\hypertarget{eg:sinsub}{}
\exds{1}{ \int \sqrt{a^2 - x^2} \, dx. }

假设$ a > 0 $, 则$ -a \le x \le a$.
\begin{align*}
  \text{原式}
  &= \int a(\sqrt{1-\sin^2 t}) \, d(a\sin t)
  && \mreason{x = a \sin t, \  t \in [-\frac{\pi}{2}, \frac{\pi}{2}] } \\
  &= a^2 \int \cos^2 t \, dt
  && \mreason{ \cos t = \sqrt{1-\sin^2 t},\ d(a\sin t) = a\cos t \, dt } \\
  &= a^2 \int \frac{1+\cos 2t}{2} \, dt
  && \mreason{ \cos^2 t = \frac{1+\cos 2t}{2} } \\
  &= \frac{a^2}{2} t + \frac{a^2}{4} \sin 2t + C
  && \mreason{ d\frac{t}{2} = \frac{1}{2} \, dt,\ d\frac{\sin 2t}{4} = \frac{\cos 2t}{2} \, dt } \\
  &= \frac{a^2}{2} \arcsin \frac{x}{a} + \frac{a^2}{4} \sin(2\arcsin\frac{x}{a}) + C
  && \mreason{ t = \arcsin\frac{x}{a} } \\
  &= \frac{a^2}{2} \arcsin \frac{x}{a} + \frac{x}{2} \sqrt{a^2 - x^2} + C.
\end{align*}

\exds{2}{ \int \frac{dx}{\sqrt{x^2 - a^2}}. }

假设$a>0$, 则$x>a$或$x<-a$.
\begin{align*}
  \text{原式} &= \int \frac{a \sec t \tan t \dt}{a \sqrt{\sec^2 t - 1}}
  && \mreason{ x = a\sec t, \  t \in (0, \frac{\pi}{2}) } \\
              &= \int \frac{ \cancel{a \tan t} \sec t }{\cancel{a \tan t}} \, dt
  && \mreason{ d(a\sec t) = a \sec t \tan t \, dt } \\
              &= \int \frac{dt}{\cos t}
  && \mreason{ \sec t = \frac{1}{\cos t} } \\
              &= \int \frac{\cos t \, dt}{\cos^2 t}
  && \reason{分子分母同乘以$\cos t$} \\
              &= \int \frac{d\sin t}{1 - \sin^2 t}
  && \mreason{ d\sin t = \cos t \, dt, \  \cos^2 t = 1 - \sin^2 t} \\
              &= \int \frac{du}{1-u^2}
  && \mreason{ u = \sin t } \\
              &= \frac{1}{2}\paren*{\int \frac{du}{1-u} + \int \frac{du}{1+u} }
  && \mreason{ \frac{1}{1-u^2} = \frac{1}{2}\paren*{\frac{1}{1-u} + \frac{1}{1+u}} } \\
              &= \frac{1}{2}(\ln \abs{1+u} - \ln \abs{1-u} + C)
  && \mreason{ d\ln \abs{1+u} = \frac{1}{1+u},\ d\ln \abs{1-u} = -\frac{1}{1-u} } \\
              &= \frac{1}{2} \ln \abs[\bigg]{\frac{1+\sin t}{1 - \sin t}} + C
  && \reason{对数的性质, \(u = \sin t\)} \\
              &= \frac{1}{2} \ln \abs[\Bigg]{ \frac{1 + \sqrt{1-\paren{a/x}^2}}
                {1 - \sqrt{1-\paren{a/x}^2}} } + C
  && \mreason{ \sin^2 t = 1 - \paren{a/x}^2} \\
              &= \frac{1}{2} \ln \abs[\Bigg]{ \frac{\paren[\big]{1+\sqrt{1-\paren{a/x}^2}}^2}
                {\paren{a/x}^2} } + C
  && \reason{分母有理化} \\
              % &= \ln \paren[\Big]{x + \sqrt{x^2-a^2}} + C.
              &= \ln \paren[\Big]{x/a + \sqrt{\paren{x/a}^2-1}} + C
  && \reason{对数的性质} \\
              &= \arccosh \frac xa.
  && \reason{反双曲余弦的定义}
\end{align*}

这里用双曲余弦函数\(x = a \cosh u\)做换元, 可使步骤更为简洁, 如下
\begin{align*}
  \int \frac{dx}{\sqrt{x^2 - a^2}}
  = \int \frac{a \sinh u}{a \sqrt{\cosh^2 u - 1}} \du
  = \int \du
  = \arccosh \frac xa + C.
\end{align*}

\exds{3}{ \int \frac{dx}{x^2 \sqrt{x^2+1}}. }

\begin{align*}
  \text{原式} &= \int \frac{\diff \tan t}{\tan^2 t \sqrt{\tan^2 t + 1}}
  && \reason{$x = \tan t$,
     因为$x \ne 0$,
     所以$t \in (-\frac{\pi}{2},\frac{\pi}{2})\setminus\{0\}$} \\
              &= \int \frac{\cancel{\sec t} \sec t \dt}{\cancel{\sec t} \tan^2 t}
  && \mreason{ \sec^2 t - \tan^2 t = 1 } \\
              &= \int \frac{\cos t \dt}{\sin^2 t}
  && \mreason{ \sec t = \frac{1}{\cos t},\  \tan t = \frac{\sin t}{\cos t} } \\
              &= \int \frac{\diff \sin t}{\sin^2 t}
  && \mreason{ \diff \sin t = \cos t \dt } \\
              &= -\frac{1}{\sin t} + C
  && \mreason{ \int u^p \, du = \frac{u^{p+1}}{p+1} + C } \\
              &= -\frac{\sqrt{1+x^2}}{x} + C.
  && \frac{1}{\sin t} = \frac{\sqrt{1+x^2}}{x}
\end{align*}

\ifshowex
\currentpdfbookmark{练习}{B1.6.2.E}
\subsection*{练习}

\begin{enumerate}
\item 若\(f(x)\)的导函数是\(\sin x\), 求\(f(x)\)的原函数.

  \ifshowsol
    这题相当于是求函数\(\sin x\)的原函数的原函数, 所以有
    \[
      \int \paren*{\int \sin x \dx} \dx = \int \paren[\big]{-\cos x + C_1} \dx = -\sin x + C_1 x + C_2.
    \]
  \fi

\item 已知\(f(x)\)的一个原函数是\(\sin x\), \(g(x)\)的一个原函数是\(x^2\), 求复函数\(f(g(x))\)的原函数.

  \ifshowsol
    因为
    \[
      f(x) = \ddx \sin x = \cos x, \quad g(x) = \ddx x^2 = 2x,
    \]
    所以
    \begin{align*}
      \int f(g(x)) \dx
      &= \int \cos 2x \dx \\
      &= \frac12 \int \cos t \dt && \mreason{x = t/2} \\
      &= \frac12 \sin t + C = \frac12 \sin 2x + C.
    \end{align*}
  \fi

\item \disp{\int e^{-\abs{x}} \dx}.
  \ifshowsol
    \begin{align*}
      \int e^{-\abs{x}} \dx
      &=
        \begin{cases}
          \int e^{-x} \dx, & x \ge 0, \\
          \int e^x \dx, & x < 0,
        \end{cases} \\
      &=
        \begin{cases}
          - \int e^{t} \dt, & x \ge 0, \qquad \mreason{x = -t} \\
          e^x + C_2, & x < 0,
        \end{cases} \\
      &=
        \begin{cases}
          -e^t + C_1, & x \ge 0, \\
          e^x + C_2, & x < 0,
        \end{cases} \\
      &=
        \begin{cases}
          -e^{-x} + C_1, & x \ge 0, \qquad \mreason{t = -x} \\
          e^x + C_2, & x < 0,
        \end{cases} \\
      &= C +
        \begin{cases}
          -e^{-x} + 2, & x \ge 0, \\
          e^x, & x <0.
        \end{cases} \quad \reason{\(-1 + C_1 = 1 + C_2\), 取\(C_2 = 0\)}
    \end{align*}
  \fi

\item 已知曲线上任一点的二阶导数\(y\dpr = 6x\), 且在曲线上点\((0, -2)\)处的切线为\(2x - 3y = 6\), 求这条曲线的方程.

  \ifshowsol
    将切线的方程化为点斜式, 得到\(y = 2x/3 - 2\), 所以\(y'\big\vert_{x=0} = 2/3\).有
    \[
      y' = \int y\dpr \dx = \int 6x \dx = 3x^2 + c_1,
    \]
    所以\(c_1 = 2/3\). 又有
    \[
      y = \int y' \dx = \int \paren[\Big]{3x^2 + \frac23} \dx = x^3 + \frac23 x + c_2,
    \]
    所以\(c_2 = -2\).这条曲线的方程是
    \[
      y = x^3 + \frac23 x - 2.
    \]
  \fi

\item \disp{\int \frac{\ln x}{x^2} \dx}.

  \ifshowsol
    最好的办法是用\(x = e^t\)换元后用分部积分法, 可惜这里还没学到.有
    \begin{align*}
      \int \frac{\ln x}{x^2} \dx
      &= \int t e^{-t} \dt
      && \mreason{x = e^t} \\
      &= - \int t \diff e^{-t} \\
      &= - \paren[\Big]{t e^{-t} - \int e^{-t} \dt} \\
      &= - t e^{-t} - e^{-t} + C \\
      &= - \frac{\ln x}{x} - \frac1x + C.
    \end{align*}
  \fi

\item 设\(f(x) = e^{-x}\), 求\disp{\int \frac{f'(\ln x)}{x} \dx}.

  \ifshowsol
    使用第一换元法, 有
    \begin{equation*}
      \int \frac{f'(\ln x)}{x} \dx
      = \int f'(\ln x) \diff (\ln x)
      = f(\ln x) = \frac1x + C.
    \end{equation*}

    也可以直接求导\(f'(x) = - e^{-x}\), 然后直接代入, 有
    \begin{equation*}
      \int \frac{f'(\ln x)}{x} \dx
      = - \int \frac{1}{x^2} \dx
      = \frac1x + C.
    \end{equation*}
  \fi

\item 若\disp{\int f(x) \dx = x^2 + C}, 求\disp{\int x \, f(1-x^2) \dx}.

  \ifshowsol
    使用第一换元法, 有
    \begin{align*}
      \int x \, f(1-x^2) \dx
      = - \frac12 \int f(1-x^2) \diff (1-x^2)
      = - \frac12 \paren[\big]{1-x^2}^2 + C.
    \end{align*}

    或者通过求导得到
    \[
      f(x) = \ddx \int f(x) \dx = \ddx (x^2 + C) = 2x,
    \]
    所以
    \[
      \int x \, f(1-x^2) \dx
      = 2 \int x (1-x^2) \dx
      = x^2 - \frac{x^4}{2} + C.
    \]

    上面两种方法得到的答案形式不一样, 但实际上是等价的, 只要把第一个答案中外部的平方展开, 把产生的常数项吸收到后面的任意常数\(C\)中, 就是第二个答案中的形式.
  \fi

\item \disp{\int \frac{x}{(1-x)^3} \dx}.

  \ifshowsol
    尝试使用\(u = 1/(1-x)^2\)和\(t = 1/(1-x)\)来做换元, 观察到
    \[
      \du = \frac{2}{(1-x)^3} \dx
      \quad \text{和} \quad
      \dt = \frac{1}{(1-x)^2} \dx,
    \]
    所以有
    \begin{align*}
      \int \frac{x}{(1-x)^3} \dx
      &= \int \paren[\bigg]{\frac{x-1}{(1-x)^3} + \frac{1}{(1-x)^3}} \dx \\
      &= - \int \frac{1}{(1-x)^2} \dx + \int \frac{1}{(1-x)^3} \dx \\
      &= - \int \dt + \frac12 \int \du \\
      &= - t + \frac u2 + C \\
      &= - \frac1{1-x} + \frac1{2 (1-x)^2} + C.
    \end{align*}
  \fi

\item \disp{\int \frac{\sin x \cos x}{1 + \sin^4 x} \dx}.

  \ifshowsol
    使用\(u = \sin x\)和\(t = u^2\)来做换元, 就有
    \begin{align*}
      \int \frac{\sin x \cos x}{1 + \sin^4 x} \dx
      &= \int \frac{u}{1+u^4} \du
      && \mreason{\du = \cos x \dx} \\
      &= \frac12 \int \frac1{1+t^2} \dt
      && \mreason{\dt = 2u \du} \\
      &= \frac12 \arctan t + C \\
      &= \frac12 \arctan \sin^2 x + C
      && \mreason{t = u^2,\ u = \sin x}
    \end{align*}
  \fi

\item \disp{\int \paren[\big]{2^x + x^2} \dx}.

  \ifshowsol
    \[
      \int \paren[\big]{2^x + x^2} \dx
      = \int 2^x \dx + \int x^2 \dx
      = \frac{2^x}{\ln 2} + \frac{x^3}3 + C.
    \]
  \fi
\end{enumerate}
\fi

\section{分部积分法}

若$u(x)$, $v(x)$连续可导, 则$[u(x) v(x)]' = u'(x)v(x) + u(x)v'(x)$, 就有
\[
  \int u(x) v'(x) \, dx = u(x) v(x) - \int v(x) u'(x) \, dx
\]
或者
\[
  \underset{\text{难}}{\underline{\int u(x) \, dv(x)}} =
  u(x) v(x) - \underset{\text{易}}{\underline{\int v(x) \, du(x)}}.
\]

这种方法适用于函数本身比较难, 但是其导函数比较简单. 这样的函数$ u(x) $一般有
\[
  \begin{matrix}
    \ln x, & \arctan x, & \arcsin x, & \text{函数复杂, 导数简单,} \\
    e^x, & \sin x, & \cos x, & \text{函数导数, 难度相同.}
  \end{matrix}
\]

\exds{1}{\int \ln x \, dx .}
\begin{align*}
  \int \underset{u(x)}{\underline{\ln x}} \, \underset{dv(x)}{\underline{dx}}
  &= x \ln x - \int \underset{v(x)}{\underline{x}} \, \underset{du(x)}{\underline{d\ln x}}
  && \reason{分部积分法} \\
  &= x \ln x - \int dx
  && \mreason{d\ln x = \frac{1}{x} \, dx} \\
  &= x \ln x - x + C .
  && \mreason{dx = dx}
\end{align*}

\exds{2}{\int x \arctan x \, dx .}
\begin{align*}
  \int x \, \underset{u(x)}{\underline{\arctan x}} \, dx
  &= \int \arctan x \, d\frac{x^2}{2}
  && \mreason{d\frac{x^2}{2} = x \, dx} \\
  &= \frac{x^2}{2} \arctan x - \int \frac{x^2}{2} \, d(\arctan x)
  && \reason{分部积分法} \\
  &= \frac{x^2}{2} \arctan x - \int \frac{x^2}{2(1+x^2)} \, dx
  && \mreason{d(\arctan x) = \frac{1}{1+x^2}} \\
  &= \frac{x^2}{2} \arctan x - \frac{1}{2} \int \paren[\bigg]{1 - \frac{1}{1+x^2}} \, dx
  && \mreason{\frac{x^2}{1+x^2} = 1 - \frac{1}{1+x^2}} \\
  &= \frac{1}{2} \paren[\bigg]{x^2 \arctan x - \int dx + \int \frac{dx}{1+x^2}}
  && \reason{积分的加法法则} \\
  &= \frac{1}{2} \brkt[\big]{(x^2+1) \arctan x - x} + C .
  && \mreason{dx = dx,\, d(\arctan x) = \frac{1}{1+x^2}}
\end{align*}

\exds{3}{\int x^2 \, e^x \dx.}
\begin{align*}
  \text{原式}
  &= \int x^2 \diff e^x \\
  &= x^2 e^x - \int e^x \, d(x^2)
  && \reason{分部积分法} \\
  &= x^2 e^x - 2\int x \, de^x \\
  &= x^2 e^x - 2\paren[\Big]{x e^x - \int e^x \, dx}
  && \reason{分部积分法} \\
  &= (x^2 - 2x + 2) e^x + C.
\end{align*}

\exds{4}{\int x \sin(2x) \, dx.}
\begin{align*}
  \text{原式}
  &= \frac{1}{2} \int x \, d[-\cos(2x)] \\
  &= \frac{1}{2} \paren*{\int \cos(2x) \, dx - x \cos(2x)}
  && \reason{分部积分法} \\
  &= \frac{1}{2} \paren[\bigg]{\frac{1}{2} \sin(2x) - x \cos(2x)} + C.
\end{align*}

\exds{5}{\int e^x \sin x \, dx.}
\begin{align*}
  -\int e^x \, d\cos x
  &= \text{原式}
    = \int \sin x \, de^x \\
  - e^x \cos x + \int \cos x \, de^x
  &= \text{原式}
    = e^x \sin x - \int e^x \, d\sin x
  && \reason{分部积分法} \\
  \intertext{因为$\int \cos x \, de^x = \int e^x \cos x \, dx = \int e^x \, d\sin x$, 所以}
  \int e^x \cos x \, dx
  &= \frac{\sin x + \cos x}{2} e^x + C, \\
  \text{原式}
  &= e^x \sin x - \frac{\sin x + \cos x}{2} e^x + C \\
  &= \frac{\sin x - \cos x}{2} e^x + C.
\end{align*}

\section{有理函数的积分}

\subsection{四个特殊函数的不定积分}

\begin{enumerate}
\item \disp{\int \frac A{ax+b} \dx,\ (a \ne 0)}.
  \[
    \frac Aa \int \frac1{x + b/a} \dx
    = \frac Aa \ln \abs[\bigg]{x + \frac ba} + C.
  \]

\item \disp{\int \frac A{(ax+b)^n} \dx, \ (a \ne 0, n \in \N^+ \setminus \brce{1})}.
  \[
    \frac A{a^n} \int \frac1{(x+b/a)^n} \dx
    = \frac A{a^n} \int \frac1{(x+b/a)^n} \diff (x+b/a)
    = - \frac A{a^n (n-1) (x+b/a)^{n-1}} + C.
  \]

\item \disp{\int \frac{Bx+D}{px^2+qx+r} \dx, \ (p \ne 0, \Delta < 0, B \ne 0)}.

  设\(P(x) = px^2 + qx + r\), \(\Delta\)为二次多项式\(P(x)\)的判别式, 则有
  \begin{align*}
    \int \frac{Bx+D}{P(x)} \dx
    &= \frac{B}{2p} \int \frac{P'(x) - q + (2p/B)D}{P(x)} \dx
    && \reason{提取因子\(B/2p\)} \\
    &= \frac{B}{2p} \int \frac{P'(x)}{P(x)} \dx + \paren[\bigg]{D - \frac{B}{2p} q} \int \frac{\dx}{P(x)}
    && \reason{积分的加法法则} \\
    &= \frac{B}{2p} \int \frac{\diff P(x)}{P(x)} + \paren[\bigg]{D - \frac{B}{2p} q} \frac1p \int \frac{\dx}{P(x)/p}
    && \reason{第一换元法} \\
    &= \frac{B}{2p} \ln \abs[\big]{P(x)} + \paren[\bigg]{D - \frac{B}{2p} q} \frac1p \int \frac{\dx}{\paren[\big]{x+\frac{q}{2p}}^2 - \frac{\Delta}{4p^2}}
    && \mreason{\frac{P(x)}{p} = \paren[\Big]{x+\frac{q}{2p}}^2 - \frac{\Delta}{4p^2}} \\
    &= \frac{B}{2p} \ln \abs[\big]{P(x)} + \paren[\bigg]{D - \frac{B}{2p} q} \frac2{\sqrt{-\Delta}} \arctan \frac{P'(x)}{\sqrt{-\Delta}} + C.
    && \mreason{\int \frac{\dx}{x^2 + a^2} = \frac1a \arctan \frac{x}{a}}
  \end{align*}

  上面推导的过程中其实假定了\(p > 0\).其实当\(p < 0\)时, 上式也成立.

\item \disp{\int \frac{Bx+D}{\paren{px^2+qx+r}^n} \dx, \ (p \ne 0, \Delta < 0, B \ne 0, n \in \N^+ \setminus \brce{1})}.

  设
  \[
    I_{n,a^2}(x) = \int \frac{\dx}{\paren[\big]{x^2 + a^2}^n},
  \]
  那么\(I_{1,a^2}(x)\)就是~\ref{6.2.1}中的\hyperlink{eg:arctan}{例3}.此外还有
  \begin{align*}
    I_{n,a^2}(x)
    &= \frac{x}{\paren[\big]{x^2 + a^2}^n} - \int x \diff\brkt[\bigg]{\frac{1}{\paren[\big]{x^2 + a^2}^n}} \\
    &= \frac{x}{\paren[\big]{x^2 + a^2}^n} + 2n \int \frac{x^2}{\paren[\big]{x^2 + a^2}^{n+1}} \dx \\
    &= \frac{x}{\paren[\big]{x^2 + a^2}^n} + 2n \int \brkt[\bigg]{\frac{x^2 + a^2}{\paren[\big]{x^2 + a^2}^{n+1}} - \frac{a^2}{\paren[\big]{x^2 + a^2}^{n+1}}} \dx \\
    &= \frac{x}{\paren[\big]{x^2 + a^2}^n} + 2n \paren[\big]{I_{n,a^2}(x) - a^2 I_{n+1,a^2}(x)},
  \end{align*}
  得到递推关系
  \begin{equation}
    I_{n+1,a^2}(x) = \frac{1}{2na^2} \brkt[\Big]{(2n-1) I_{n,a^2}(x) + \frac{x}{\paren[\big]{x^2 + a^2}^n}}.
  \end{equation}
  用处理第三类特殊函数同样的方法, 得到
  \begin{align*}
    \int \frac{Bx+D}{P^n(x)} \dx
    &= \frac{B}{2p} \int \frac{P'(x) - q + (2p/B)D}{P^n(x)} \dx \\
    &= \frac{B}{2p} \int \frac{P'(x)}{P^n(x)} \dx + \paren[\bigg]{D - \frac{B}{2p} q} \int \frac{\dx}{P^n(x)} \\
    &= \frac{B}{2p} \int \frac{\diff P(x)}{P^n(x)} + \paren[\bigg]{D - \frac{B}{2p} q} \frac1{p^n} \int \frac{\dx}{\brkt*{P(x)/p}^n} \\
    &= \frac{B}{2p} \frac{P^{1-n}(x)}{1-n} + \paren[\bigg]{D - \frac{B}{2p} q} \frac1{p^n} \int \frac{\dx}{\brkt[\Big]{\paren[\big]{x+\frac{q}{2p}}^2 - \frac{\Delta}{4p^2}}^n} \\
    &= \frac{B}{2p} \frac{P^{1-n}(x)}{1-n} + \paren[\bigg]{D - \frac{B}{2p} q} \frac{I_{n,-\Delta/4p^2}\paren*{x+q/2p}}{p^n} + C.
  \end{align*}
\end{enumerate}

\subsection{有理分式函数的化简}

设\(P_n(x)\)和\(Q_m(x)\)分别为\(x\)的\(n\)次和\(m\)次多项式, 则当\(n < m\)时, 称
\[
  \frac{P_n(x)}{Q_m(x)}
\]
为真分式有理函数, 当\(n \ge m\)时, 称之为假分式有理函数.一个真分式有理函数可以分解成一个多项式与一个假分式有理函数的和的形式.

\exds{1}{\frac{x^4 + 2x^3 + x^2 + 3}{x^2 + 1} = x^2 + 2x + \frac{-2x+3}{x^2 + 1}}.

多项式的不定积分已经解决了, 剩下只要把真分式有理函数分解成前面提到的四类特殊函数的和的形式, 那么有理函数的不定积分就解决了.

\exds{2}{\frac{2x^2 + 2x + 13}{(x-2)\paren{x^2+1}^2}}.

使用待定系数法, 有
\begin{align*}
  \text{原式}
  &= \frac{A}{x-2} + \frac{Bx+C}{x^2+1} + \frac{Dx+E}{\paren{x^2+1}^2} \\
  &= \frac{A\paren{x^2+1}^2
    + \paren{Bx+C}\paren{x-2}\paren{x^2+1}
    + \paren{Dx+E}\paren{x-2}}{%
    (x-2)\paren{x^2+1}^2} \\
  &= \frac{
    \splitfrac{\paren{A + B} x^4
    + \paren{-2B + C} x^3
    + \paren{2A + B - 2C + D}x^2}{%
    + \paren{-2B + C - 2D + E} x
    + \paren{A - 2C - 2E}}}{%
    (x-2)\paren{x^2+1}^2}.
\end{align*}
写成方程组的形式
\[
  \hspace*{-2em}
  \systeme{%
    A + B = 0,
    {-2}B + C = 0,
    2A + B - 2C + D = 2,
    {-2}B + C - 2D + E = 2,
    A - 2C - 2E = 13}
  \quad
  \mathrel{\Longrightarrow}
  \qquad
  \systeme{%
    A = 1,
    B = -1,
    C = -2,
    D = -3,
    E = -4},
\]
所以有
\[
  \frac{2x^2 + 2x + 13}{(x-2)\paren{x^2+1}^2}
  = \frac{1}{x-2} - \frac{x+2}{x^2+1} - \frac{3x+4}{\paren{x^2+1}^2}.
\]

\exds{3}{\frac{x^2 + 3x + 1}{(x-2)^2 (x^2 + x + 2)^2}}.

使用待定系数法, 有
\[
  \frac{x^2 + 3x + 1}{(x-2)^2 (x^2 + x + 2)^2}
  = \frac{A}{x-2} + \frac{B}{(x-2)^2} + \frac{Cx+D}{x^2 + x + 2} + \frac{Ex+F}{(x^2 + x + 2)^2}.
\]
右边通分后, 整理成方程组的形式
\[
  \systeme{%
    A + C = 0,
    B - 3C + D = 0,
    A + 2B + 2C - 3D + E = 0,
    {-6}A + 5B - 4C + 2D - 4E + F = 1,
    {-4}A + 4B + 8C - 4D + 4E - 4F = 3,
    {-8}A + 4B + 8D + 4F = 1}
  \quad
  \mathrel{\Longrightarrow}
  \quad
  \systeme{%
    A = -27/256,
    B = 11/64,
    C = 27/256,
    D = 37/256,
    E = -1/64,
    F = -27/64},
\]
所以有
\[
  \frac{x^2 + 3x + 1}{(x-2)^2 (x^2 + x + 2)^2}
  = - \frac{27}{256(x-2)} + \frac{11}{64(x-2)^2} + \frac{27x+37}{256(x^2 + x + 2)} - \frac{x+27}{64(x^2 + x + 2)^2}.
\]

\subsection{有理分式函数的不定积分\label{6.4.3}}

\exds{1}{\int \frac{x^4 + 2x^3 + x^2 + 3}{x^2 + 1} \dx}.
\begin{align*}
  \int \paren[\bigg]{x^2 + 2x + \frac{-2x+3}{x^2 + 1}} \dx
  &= \frac{\,x^3\!}{3} + x^2 - \int \frac{2x}{x^2 + 1} \dx + 3 \int \frac{\dx}{x^2 + 1} \\
  &= \frac{\,x^3\!}{3} + x^2 - \ln\paren{x^2+1} + 3 \arctan x + C.
\end{align*}

\exds{2}{\int \frac{2x^2 + 2x + 13}{(x-2)\paren{x^2+1}^2} \dx}.

\begin{align*}
  \int \frac{2x^2 + 2x + 13}{(x-2)\paren{x^2+1}^2} \dx
  &= \int \brkt[\bigg]{\frac{1}{x-2} - \frac{x+2}{x^2+1} - \frac{3x+4}{\paren{x^2+1}^2}} \dx \\
  &= \ln\abs[\big]{x-2}
    - \frac12 \int \frac{2x}{x^2+1} \dx
    - 2 \int \frac{\dx}{x^2+1} \\
  &\qquad - \frac32 \int \frac{2x}{(x^2+1)^2} \dx - 4 \int \frac{\dx}{(x^2+1)^2} \\
  &= \ln\abs[\big]{x-2} - \frac12 \ln(x^2+1) - 2 \arctan x \\
  &\qquad + \frac{3}{2(x^2+1)} - 4 \, I_{2,1} \\
  &= \ln\abs[\big]{x-2} - \frac12 \ln(x^2+1) - 2 \arctan x \\
  &\qquad + \frac{3}{2(x^2+1)} - 2 \, \paren*{I_{1,1} + \frac{x}{x^2+1}} \\
  &= \ln\abs[\big]{x-2} - \frac12 \ln(x^2+1) - 2 \arctan x \\
  &\qquad + \frac{3}{2(x^2+1)} - 2 \arctan x  - \frac{2x}{x^2+1} + C \\
  &= \ln\frac{\abs{x-2}}{\scriptstyle\sqrt{x^2+1}} - 4 \arctan x + \frac{3-4x}{2(x^2+1)} + C.
\end{align*}

\subsection{三角有理函数的不定积分}

对于\(R(\sin x, \cos x)\)这样带有三角函数的分式有理函数, 可以使用变量替换\(t = \tan\frac{x}{2}\)将其转换成关于\(t\)的有理函数.因为
\begin{gather*}
  \sin x = \frac{2\tan\frac{x}{2}}{1+\tan^2\frac{x}{2}} = \frac{2t}{1+t^2},
  \quad
  \cos x = \frac{1 - \tan^2\frac{x}{2}}{1+\tan^2\frac{x}{2}} = \frac{1-t^2}{1+t^2}, \\[1ex]
  x = 2 \arctan t,
  \quad
  \dx = \frac{2}{1+t^2} \dt,
\end{gather*}
所以
\[
  R(\sin x, \cos x) \dx = R\paren[\bigg]{\frac{2t}{1+t^2}, \frac{1-t^2}{1+t^2}} \frac{2}{1+t^2} \dt.
\]
设
\[
  I(t) = \int R\paren[\bigg]{\frac{2t}{1+t^2}, \frac{1-t^2}{1+t^2}} \frac{2}{1+t^2} \dt,
\]
则
\[
  \int R(\sin x, \cos x) \dx = I\paren[\Big]{\tan\frac{x}{2}}.
\]

\ifshowex
\currentpdfbookmark{练习}{B1.6.4.E}
\subsection*{练习}

\begin{enumerate}
  \ifshowsol
    \setlength{\itemsep}{10pt plus 4pt minus 4pt}
    \setlength{\abovedisplayskip}{5pt plus 2pt minus 5pt}
    \setlength{\belowdisplayskip}{10pt plus 3pt minus 2pt}
  \fi
\item 设\(\displaystyle J_k = \int \frac{\dx}{\brkt{(x+a)^2 + b^2}^k}\), 求\(J_k\)的递推表达式.

  \ifshowsol
    因为
    \[
      J_k
      = \int \frac{\dx}{\brkt{(x+a)^2 + b^2}^k}
      = \int \frac{\diff(x+a)}{\brkt{(x+a)^2 + b^2}^k}
      = \int \frac{\du}{\paren{u^2 + b^2}^k}
      = I_{k,b^2}(u)
      = I_{k,b^2}(x+a),
    \]
    所以
    \begin{equation*}
      \begin{split}
        J_{k+1}
        = I_{k+1,b^2}(x+a)
        &= \frac{1}{2kb^2} \brce[\bigg]{(2k-1) I_{k,b^2}(x+a) + \frac{x+a}{\brkt{(x+a)^2 + b^2}^k}} \\
        &= \frac{1}{2kb^2} \brce[\bigg]{(2k-1) J_k + \frac{x+a}{\brkt{(x+a)^2 + b^2}^k}}.
      \end{split}
    \end{equation*}
  \fi

\item \(\displaystyle \int \frac{\dx}{(x+a)^2 + b^2} \quad (b \ne 0)\).

  \ifshowsol
    \[
      \int \frac{\dx}{(x+a)^2 + b^2}
      = \int \frac{\diff(x+a)}{(x+a)^2 + b^2}
      = \int \frac{\du}{u^2 + b^2}
      = \frac1b \arctan \frac ub + C
      = \frac1b \arctan \frac{x+a}b + C.
    \]
  \fi

\item \(\displaystyle \int \frac{\dx}{1-x^2}\).

  \ifshowsol
    \[
      \int \frac{\dx}{1-x^2}
      = \arctanh x + C
      = \frac12 \ln\frac{1+x}{1-x} + C.
    \]
  \fi

\item \(\displaystyle \int \frac{t+4}{t^2 + 5t -6} \dt\).

  \ifshowsol
    \[
      \begin{split}
        \int \frac{t+4}{t^2 + 5t -6} \dt
        &= \int \frac{t+4}{(t-1)(t+6)} \dt
          = \frac57 \int \frac{\dt}{t-1} + \frac27 \int \frac{\dt}{t+6} \\
        &= \frac57 \ln\abs[\big]{t-1} + \frac27 \ln\abs[\big]{t+6} + C
          = \frac17 \ln\abs[\big]{(t-1)^5 (t+6)^2} + C.
      \end{split}
    \]
  \fi

\item \(\displaystyle \int \frac{2x^2+2x+13}{(x-2)(x^2+1)^2} \dx\).

  \ifshowsol
    \pskip
    这题就是~\ref{6.4.3}的例2.
    \pskip
  \fi

\item \(\displaystyle \int \frac{\dx}{x^4+1}\).

  \ifshowsol
    \[
      \begin{split}
        \int \frac{\dx}{x^4+1}
        &= \int \frac{\dx}{(x^2+1)^2 - 2x^2}
          = \int \frac{\dx}{(x^2+{\scriptstyle\sqrt2}\,x+1)(x^2-{\scriptstyle\sqrt2}\,x+1)} \\
        &= \frac14 \int \paren[\bigg]{\frac{{\scriptstyle\sqrt2}\,x+2}{x^2+{\scriptstyle\sqrt2}\,x+1} - \frac{{\scriptstyle\sqrt2}\,x-2}{x^2+{\scriptstyle\sqrt2}\,x+1}} \dx \\
        &= \frac{\scriptstyle\sqrt2\,}{8} \int \frac{2x+{\scriptstyle\sqrt2}}{x^2+{\scriptstyle\sqrt2}\,x+1} \dx + \frac14 \int \frac{\dx}{x^2+{\scriptstyle\sqrt2}\,x+1} \\
        &\qquad - \frac{\scriptstyle\sqrt2\,}{8} \int \frac{2x-{\scriptstyle\sqrt2}}{x^2-{\scriptstyle\sqrt2}\,x+1} \dx + \frac14 \int \frac{\dx}{x^2-{\scriptstyle\sqrt2}\,x+1} \\
        &= \frac{\scriptstyle\sqrt2\,}{8} \ln\paren[\bigg]{\frac{x^2+{\scriptstyle\sqrt2}\,x+1}{x^2-{\scriptstyle\sqrt2}\,x+1}} + \frac14 \int \frac{\dx}{(x+{\scriptstyle\sqrt2}/2)^2 + 1/2} \\
        &\qquad + \frac14 \int \frac{\dx}{(x-{\scriptstyle\sqrt2}/2)^2 + 1/2} \\
        &= \frac{\scriptstyle\sqrt2\,}{8} \ln\paren[\bigg]{\frac{x^2+{\scriptstyle\sqrt2}\,x+1}{x^2-{\scriptstyle\sqrt2}\,x+1}} + \frac{\scriptstyle\sqrt2\,}4 \int \frac{\diff({\scriptstyle\sqrt2}\,x+1)}{({\scriptstyle\sqrt2}\,x+1)^2 + 1} \\
        &\qquad + \frac{\scriptstyle\sqrt2\,}4 \int \frac{\diff({\scriptstyle\sqrt2}\,x-1)}{({\scriptstyle\sqrt2}\,x-1)^2 + 1} \\
        &= \frac{\scriptstyle\sqrt2\,}{8} \ln\paren[\bigg]{\frac{x^2+{\scriptstyle\sqrt2}\,x+1}{x^2-{\scriptstyle\sqrt2}\,x+1}} + \frac{\scriptstyle\sqrt2\,}4 \brkt*{\arctan({\scriptstyle\sqrt2}\,x+1) + \arctan({\scriptstyle\sqrt2}\,x-1)} + C \\
        &= \frac{\scriptstyle\sqrt2\,}{8} \ln\paren[\bigg]{\frac{x^2+{\scriptstyle\sqrt2}\,x+1}{x^2-{\scriptstyle\sqrt2}\,x+1}} + \frac{\scriptstyle\sqrt2\,}4 \arctan\frac{{\scriptstyle\sqrt2}\,x}{1-x^2} + C.
      \end{split}
    \]
  \fi
\end{enumerate}
\fi

\section{简单无理式的积分\label{6.5}}

类型I

\begin{itemize}
\item \(\displaystyle \int R\paren[\big]{x, \sqrt{(x+p)^2 - q^2}} \dx\), 用\(x + p = q \sec t\)做替换.
\item \(\displaystyle \int R\paren[\big]{x, \sqrt{(x+p)^2 + q^2}} \dx\), 用\(x + p = q \tan t\)做替换.
\item \(\displaystyle \int R\paren[\big]{x, \sqrt{q^2 - (x+p)^2}} \dx\), 用\(x + p = q \sin t\)做替换.
\end{itemize}

前面~\ref{6.2.2}的\hyperlink{eg:sinsub}{例1}就属于这一类型的替换.

\exds{1}{\int \frac{\dx}{1+\sqrt{x^2+2x+2}}}.

类型II
\[
  \int R\paren[\big]{x, \sqrt[n]{ax+b}} \dx, \quad (a \ne 0).
\]
用
\[
  t = \sqrt[n]{ax+b}, \quad
  x = \frac{t^n-b}{a}, \quad
  \dx = \frac{nt^{n-1}}{a} \dt
\]
做替换, 就有
\[
  \int R\paren[\big]{x, \sqrt[n]{ax+b}} \dx
  = \int R\paren[\Big]{\frac{t^n-b}{a}, t} \frac{nt^{n-1}}{a} \dt.
\]

\exds{2}{\int \frac{x}{\sqrt{x-1}} \dx}.

类型III

\[
  \int R\paren[\Big]{x, \sqrt[n]{\frac{ax+b}{cx+d}}} \dx, \quad (a \ne 0, c \ne 0).
\]
用
\[
  t = \sqrt[n]{\frac{ax+b}{cx+d}}, \quad
  x = \frac{dt^n-b}{a-ct^n}, \quad
  \dx = \frac{(ad-bc)nt^{n-1}}{(a-ct^n)^2} \dt
\]
做替换, 就有
\[
  \int R\paren[\Big]{x, \sqrt[n]{\frac{ax+b}{cx+d}}} \dx
  = \int R\paren[\Big]{\frac{dt^n-b}{a-ct^n}, t} \frac{(ad-bc)nt^{n-1}}{(a-ct^n)^2} \dt.
\]

\hypertarget{eg:rootI}{}
\exds{3}{\int \frac{\dx}{\sqrt[3]{(x-1)(x+1)^2}}}.

用\(t = \sqrt[3]{\frac{x+1}{x-1}}\)做换元, 则\(x = \frac{t^3+1}{t^3-1},\ \dx = -\frac{2(3t^2)}{(t^3-1)^2} \dt\), 那么
\[
  \begin{split}
    \int \frac{\dx}{\sqrt[3]{(x-1)(x+1)^2}}
    &= \int \sqrt[3]{\frac{x+1}{x-1}} \frac{\dx}{x+1}
      = \int t \cdot \frac{t^3-1}{2t^3} \cdot \frac{-2(3t^2)}{(t^3-1)^2} \dt
      = -3 \int \frac{\dt}{t^3-1} \\
    &= -3 \int \frac{\dt}{(t-1)(t^2+t+1)}
      = - \int \paren[\Big]{\frac{1}{t-1} - \frac{t+2}{t^2+t+1}} \dt \\
    &= \frac12 \int \frac{2t+1}{t^2+t+1} \dt + \frac32 \int \frac{\dt}{\paren[\big]{t+\frac12}^2+\frac34} - \ln\abs[\big]{t-1} \\
    &= \frac12 \ln(t^2+t+1) - \ln\abs[\big]{t-1} + 2 \int \frac{\dt}{\paren[\Big]{\frac{2t+1}{\scriptscriptstyle\sqrt3}}^2 + 1} \\
    &= \frac12 \ln(t^2+t+1) - \ln\abs[\big]{t-1} + \sqrt3 \arctan \frac{2t+1}{\scriptstyle\sqrt3} + C.
  \end{split}
\]

\ifshowex
\subpdfbookmark{练习}{B1.6.5.E}
\subsection*{练习}

\begin{enumerate}
  \ifshowsol
    \setlength{\parskip}{8pt plus 3pt minus 2pt}
    \setlength{\itemsep}{7pt plus 2pt minus 2pt}
    \setlength{\abovedisplayskip}{7pt plus 5pt minus 2pt}
    \setlength{\belowdisplayskip}{8pt plus 4pt minus 3pt}
  \fi
\item \(\displaystyle \int \frac{\sqrt{x^2 - 2x + 1}}{x-1} \dx\).

  \ifshowsol
    \vspace*{-1em}
    \[
      \int \frac{\sqrt{x^2 - 2x + 1}}{x-1} \dx
      = \int \frac{\sqrt{(x-1)^2}}{x-1} \dx
      = \int \frac{\abs{x-1}}{x-1} \dx
      = \int \sgn(x-1) \dx
      =
      \begin{cases}
        x + C_1, & x > 1, \\
        -x + C_2, & x < 1.
      \end{cases}
    \]
  \fi

\item \(\displaystyle \int \sqrt{e^x+1} \dx\).

  \ifshowsol
    用\(t = \sqrt{e^x + 1}\)做换元, 有
    \[
      \begin{split}
        \int \sqrt{e^x+1} \dx
        &= 2 \int \frac{t^2}{t^2-1} \dx
          = 2 \int \paren[\Big]{1 - \frac{1}{1-t^2}} \dx
          = 2t - \ln\abs[\bigg]{\frac{1+t}{1-t}} + C \\
        &= 2 \sqrt{e^x + 1} - \ln\abs[\Bigg]{\frac{1+\sqrt{e^x + 1}}{1-\sqrt{e^x + 1}}} + C.
      \end{split}
    \]
  \fi

\item \(\displaystyle \int \frac{\dx}{\sqrt[3]{(x-1)(x+1)^2}}\).

  \ifshowsol
    这题就是~\ref{6.5}的\hyperlink{eg:rootI}{例3}
  \fi

\item \(\displaystyle \int \frac{\dx}{1 + \sin x}\).

  \ifshowsol
    用\(t = \tan\frac{x}{2}\)做换元, 有
    \[
      \begin{split}
        \int \frac{\dx}{1 + \sin x}
        &= \int \frac{1}{\scriptstyle 1 + \tfrac{2t}{1+t^2}} \frac{2}{\scriptstyle 1+t^2} \dt
          = 2 \int \frac{\dt}{t^2+2t+1}
          = 2 \int \frac{\dt}{(t+1)^2} \\
        &= - \frac{2}{t+1} + C
          = - \frac{2}{1+\tan\frac{x}{2}} + C.
      \end{split}
    \]
  \fi

\item \(\displaystyle \int \frac{\dx}{1 - \cos x}\).

  \ifshowsol
    同样用\(t = \tan\frac{x}{2}\)做换元, 有
    \[
      \begin{split}
        \int \frac{dx}{1-\cos x}
        = \int \frac{1}{\scriptstyle 1-\tfrac{1-t^2}{1+t^2}} \frac{2}{1+t^2} \dt
        = \int \frac{dt}{t^2}
        = - \frac1t + C
        = - \cot\frac{x}{2} + C.
      \end{split}
    \]
  \fi

\item \(\displaystyle \int \frac{1 + \sin x}{1 + \cos x} \dx\).

  \ifshowsol
    同样用\(t = \tan\frac{x}{2}\)做换元, 有
    \[
      \begin{split}
        \int \frac{1 + \sin x}{1 + \cos x} \dx
        &= \int \tfrac{1 + \tfrac{2t}{1+t^2}}{1+\tfrac{1-t^2}{1+t^2}} \frac{2}{1+t^2} \dt
          = \int \paren[\Big]{1 + \frac{2t}{1+t^2}} \dt \\
        &= t + \ln\paren{1+t^2} + C
          = \tan\frac{x}{2} + \ln\sec^2\frac{x}{2} + C \\
        &= \tan\frac{x}{2} + \ln\frac{2}{1 + \cos x} + C
          = \tan\frac{x}{2} - \ln\paren{1 + \cos x} + C.
      \end{split}
    \]
  \fi

\item \(\displaystyle \int \frac{\tan x}{a^2 \cos^2 x + b^2 \sin^2 x} \dx,\ (a \ne 0,\ b \ne 0)\).

  \ifshowsol
    用\(u = \tan x\)做换元, 有
    \[
      \begin{split}
        \int \frac{\tan x}{a^2 \cos^2 x + b^2 \sin^2 x} \dx
        &= \int \frac{\tan x}{a^2 + b^2 \tan^2 x} \frac{1}{\cos^2 x} \dx
          = \int \frac{u}{a^2 + b^2 u^2} \du \\
        &= \frac1{2b^2} \ln\paren{a^2 + b^2 u^2} + C
          = \frac1{2b^2} \ln\paren{a^2 + b^2 \tan^2 x} + C.
      \end{split}
    \]
  \fi

\item \(\displaystyle \int \frac{\sin 2x}{\cos^2 x + 2 \sin x} \dx\).

  \ifshowsol
    用\(u = \sin x\)做换元, 有
    \[
      \int \frac{\sin 2x}{\cos^2 x + 2 \sin x} \dx
      = \int \frac{2 \sin x \cos x}{1 - \sin^2 x + 2 \sin x} \dx
      = \int \frac{2u}{1 - u^2 + 2u} \du.
    \]
    上面最右边的式子, 我们可以有两种方式处理.两种方式处理后得到的结果形式略有不同, 但是变形后其实是一样的.第一种就是继续部分分式分解, 有
    \[
      \begin{split}
        \int \frac{2u}{1 - u^2 + 2u} \du
        &= - \frac12 \int \paren[\bigg]{\frac{2 + \sqrt2}{u - 1 - \sqrt2} + \frac{2 - \sqrt2}{u - 1 + \sqrt2}} \du \\
        &= - \frac{2+\sqrt2}{2} \ln\abs[\Big]{u - 1 - \sqrt2} - \frac{2-\sqrt2}{2} \ln\abs[\Big]{u - 1 + \sqrt2} + C \\
        &= - \frac{2+\sqrt2}{2} \ln\abs[\Big]{\sin x - 1 - \sqrt2} - \frac{2-\sqrt2}{2} \ln\abs[\Big]{\sin x - 1 + \sqrt2} + C.
      \end{split}
    \]
    第二种就是直接凑微分, 有
    \[
      \begin{split}
        \int \frac{2u}{1 - u^2 + 2u} \du
        &= - \int \paren[\bigg]{\frac{2u-2}{u^2 - 2u - 1} + \frac{2}{u^2 - 2u - 1}} \du \\
        &= - \ln\abs[\Big]{u^2 - 2u - 1} - \frac{\sqrt2}{2} \int \paren[\bigg]{\frac{1}{u - 1 - \sqrt2} - \frac{1}{u - 1 + \sqrt2}} \du \\
        &= - \ln\abs[\Big]{u^2 - 2u - 1} + \frac{\sqrt2}{2} \ln\abs[\Bigg]{\frac{u - 1 + \sqrt2}{u - 1 - \sqrt2}} + C \\
        &= - \ln\abs[\Big]{\cos^2 x + 2 \sin x} + \frac{\sqrt2}{2} \ln\abs[\Bigg]{\frac{\sin x - 1 + \sqrt2}{\sin x - 1 - \sqrt2}} + C.
      \end{split}
    \]
  \fi

\item \(\displaystyle \int \frac{\dx}{\sqrt x \paren{\sqrt x + \sqrt[3]x}}\).

  \ifshowsol
    用\(t = \sqrt[6]x\)做换元, 有
    \[
      \int \frac{\dx}{\sqrt x \paren{\sqrt x + \sqrt[3]x}}
      = \int \frac{6t^5}{t^3(t^3+t^2)} \dt
      = 6 \int \frac{\dt}{t+1}
      = 6 \ln(1+t) + C
      = 6 \ln(1+\sqrt[6]x) + C.
    \]
  \fi

\item \(\displaystyle \int x \sqrt{x+2} \dx\).

  \ifshowsol
    用\(t = \sqrt{x+2}\)做换元, 有
    \[
      \begin{split}
        \int x \sqrt{x+2} \dx
        &= \int \paren{t^2-2}t \cdot 2t \dt
          = \int \paren{2t^4 - 4t^2} \dt \\
        &= \frac25 t^5 - \frac43 t^3 + C
          = \frac25 (x+2)^{5/2} - \frac43 (x+2)^{3/2} + C.
      \end{split}
    \]
  \fi
\end{enumerate}
\fi

% Local Variables:
% TeX-engine: luatex
% TeX-master: "微积分B"
% LaTeX-indent-begin-exceptions-list: ("ifshowex")
% End:
