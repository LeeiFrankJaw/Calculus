\chapter{极限论}

\section{数列极限的概念与性质}

\begin{definition*}
  设\(D \subset \N\)且函数\( a\colon D \to \R\).我们把\(a\)这样的函数叫作\kenten{整标函数},简称\kenten{数列}.通常讨论的数列,其定义域就是\(\N\).一般将\(a(1),\, a(2),\, \dots,\, a(n),\, \dots\)记作\(a_1,\, a_2,\, \dots,\, a_n,\, \dots\),简记作\(\Seq{a_n}\).
\end{definition*}

\begin{definition*}
  设\(\Seq{a_n}\)是数列,且整标函数\(n\colon \N \to \N\)单调递增.我们把复合函数\(a \circ n\)叫作原数列\(\Seq{a_n}\)的一个子数列,简称子列,记作\(\Seq{a_{n_k}}\).
  \begin{remark}
    易知对于任意的自然数\(k\)都有\(n_k \ge k\).
  \end{remark}
\end{definition*}

\begin{definition*}
  设\(\Seq{a_n}\)是数列,\(A\)是常数.我们把\(A\)叫作数列\(\Seq{a_n}\)的极限,记作\(\lim\limits_{n\to\infty} a_n = A\),当且仅当:对于\kenten{任意的}\(ε > 0\)都\kenten{存在}正整数\(N\)使得\kenten{当}\(n > N\)\kenten{时}都有\(\abs*{\,a_n - A\,} < ε\).

  \begin{remark}
    可以用形式化的符号来表示上述定义,即
    \[
      \lim_{n\to\infty} a_n = A \iff
      \paren[\big]{\forall ε > 0}
      \paren[\big]{\exists N > 0}
      \paren[\big]{\forall n > N}
      \paren[\big]{\abs[\big]{\,a_n - A\,} < ε}.
    \]
    它的否定形式就是
    \[
      \lim_{n\to\infty} a_n \ne A \iff
      \paren[\big]{\exists ε > 0}
      \paren[\big]{\forall N > 0}
      \paren[\big]{\exists n > N}
      \paren[\big]{\abs[\big]{\,a_n - A\,} \ge ε}.
    \]
  \end{remark}
\end{definition*}

\begin{example*}
  \(\lim\limits_{n\to\infty} \paren*{1+\frac1n} = 1\).

  \begin{proof}
    取\(N = \ceil{1/ε}\)即可\footnote{这里的符号\(\ceil\cdot\)是上取整的意思,即\(\ceil x \coloneq - \floor{-x}\).}.
  \end{proof}
\end{example*}

\begin{example}
  \label{eg:limqn}
  \(\lim\limits_{n\to\infty} q^n = 0\ (\abs{\,q\,} < 1)\).

  \begin{proof}
    当\(q = 0\)时,取\(N = 1\)即可.当\(q \ne 0\)时,令\(x = 1/\abs[\big]{\,q\,} - 1\).那么,对于任意的\(ε > 0\),取\(N = \ceil[\big]{(1/ε-1)/x}\),就能使得当\(n > N\)时都有
    \begin{equation*}
      \abs{\,q\,}^n
      = \frac{1}{(1+x)^n}
      < \frac{1}{1+nx}
      < ε. \qedhere
    \end{equation*}
  \end{proof}

  \begin{remark}
    此处使用了伯努利不等式.课堂上给出的证明,取\(N = \ceil[\big]{\,\ln ε /\! \ln\abs[\big]{\,q\,}\,}\),但作为证明,其实这是有问题的.
  \end{remark}
\end{example}

\begin{example*}
  \(\lim\limits_{n\to\infty} a^{1/n} = 1\ (a > 1)\).

  % \begin{proof}
  %   取\(N = \ceil*{\ln a /\! \lnp{1+ε}}\)即可.
  % \end{proof}
  % Same as the above.  As a proof, it is problematic.

  \begin{proof}
    把\(a^{1/n}\!\)分解成一个常数与一个无穷小量和的形式.因为\(a > 1\),所以\(a^{1/n} > 1\).令\(ε_n = a^{1/n} - 1\),就有
    \begin{align*}
      a^{1/n}
      &= 1 + ε_n
      && \reason{移项} \\
      a
      &= (1 + ε_n)^n
      && \reason{取\(n\)次幂} \\
      &= 1 + nε_n + \sum_{k=2}^n \binom{n}{k} ε_n^k
      && \reason{二项式定理} \\
      &> nε_n.
    \end{align*}
    因此,
    \begin{equation*}
      \paren[\big]{\forall ε > 0}
      \paren[\big]{\exists N = \ceil{a/ε} > 0}
      \paren[\big]{\forall n > N}
      \paren[\big]{\abs[\big]{\,a^{1/n} - 1\,} = ε_n < a/n < ε}.
      \qedhere
    \end{equation*}
  \end{proof}

  \begin{remark}
    也可以用伯努利不等式来替换上面用到的二项式定理.实际上,此例的结论对于\(a > 0\)的情况都成立.对于\(0 < a < 1\)的情况,取\(N = \ceil{(1/ε-1)/(1/a^{1/n}-1)}\),再使用一次伯努利不等式即可.
  \end{remark}
\end{example*}

\begin{example*}
  若\(\lim\limits_{n\to\infty} a_n = A\),则\(\lim\limits_{n\to\infty} \abs*{\,a_n\,} = \abs*{\,A\,}\).

  \begin{proof}
    因为\(\lim\limits_{n\to\infty} a_n = A\),所以
    \begin{equation*}
      \paren[\big]{\forall ε > 0}
      \paren[\big]{\exists N > 0}
      \paren[\big]{\forall n > N}
      \paren[\big]{\abs[\big]{\,a_n - A\,} < ε}.
    \end{equation*}
    由反三角不等式
    \begin{equation*}
      \abs*{\,\abs{\,a_n\,} - \abs A\,} \le \abs*{\,a_n - A\,}
    \end{equation*}
    可知,取同样的\(N\)即可.
  \end{proof}
\end{example*}

\begin{example}
  \label{eg:limavg}
  若\(\lim\limits_{n\to\infty} a_n = A\),则\(\lim\limits_{n\to\infty} \frac1n \sum\limits_{\mathclap{k=1}}^n a_k = A\).

  \begin{proof}
    对于任意的\(ε > 0\),因为\(\lim\limits_{n\to\infty} a_n = A\),所以存在正整数\(N_0\)使得当\(n > N_0\)时都有
    \begin{align*}
      \abs[\bigg]{\,\frac1n \sum_{k=1}^n a_k - A\,}
      &= \abs[\bigg]{\,\frac1n \sum_{k=1}^n (a_k - A)\,} \\
      &\le \frac1n \sum_{k=1}^n \abs[\big]{\,a_k - A\,} \\
      &= \frac1n \paren[\Bigg]{
        \sum_{k=1}^{N_0} \abs[\big]{\,a_k - A\,}
        \ + \ \smashoperator{\sum_{k={\scriptscriptstyle N_0}+1}^n} \abs[\big]{\,a_k - A\,}} \\
      &< \frac1n \brkt[\big]{c + \paren[\big]{n-N_0}ε} \\
      &< \frac{c}{n} + ε.
    \end{align*}
    其中,\(c\)是某个取决于\(N_0\)的常数.此时,取\(N = \maxb*{\,N_0, \ceil{c/ε}\,}\),就有:当\(n > N\)时都有
    \begin{equation*}
      \abs[\bigg]{\,\frac1n \sum_{k=1}^n a_k - A\,}
      < 2ε. \qedhere
    \end{equation*}
  \end{proof}
\end{example}

\begin{theorem}
  \label{thm:cvguniq}
  若极限\(\lim\limits_{n\to\infty} a_n\)存在,则其值唯一.

  \begin{proof}
    设\(A_1\)和\(A_2\)都是数列\(\Seq{a_n}\)的极限.假设\(A_1 \ne A_2\),则对于任意的\(0 < ε \le \abs*{\,A_1-A_2\,}/2\),就可以找到正整数\(N_1\)和\(N_2\)使得:当\(n > N_1\)时都有\(\abs*{\,a_n - A_1} < ε\),当\(n > N_2\)时都有\(\abs*{\,a_n - A_2} < ε\).此时,取\(N = \maxb*{\,N_1,\, N_2}\).那么,当\(n > N\)时都有
    \begin{equation*}
      \abs*{\,A_1 - A_2\,}
      = \abs*{(a_n - A_1) - (a_n - A_2)}
      \le \abs*{\,a_n - A_1} + \abs*{\,a_n - A_2}
      < 2ε
      \le \abs*{\,A_1 - A_2\,}.
    \end{equation*}
    矛盾.因此,\(A_1 = A_2\).
  \end{proof}
\end{theorem}

\begin{theorem}
  \label{thm:cvgbnd}
  若数列\(\Seq{a_n}\)收敛,则其有界.

  \begin{proof}
    设数列\(\Seq{a_n}\)收敛于\(A\).任取一个\(ε > 0\),则存在正整数\(N\)使得当\(n > N\)时都有\(\abs*{\,a_n - A\,} < ε\),即
    \begin{equation*}
      A - ε < a_n < A + ε.
    \end{equation*}
    易知\(\maxb{\,\abs{\,A+ε\,},\, \abs{\,A-ε\,},\, \abs{\,a_1\,},\, \dots,\, \abs{\,a_N\,}\,}\)就是数列\(\Seq{a_n}\)的一个界.
  \end{proof}
\end{theorem}

\begin{theorem}[保号性]
  \label{thm:cvgsgnprsv}
  若\(\lim\limits_{n\to\infty} a_n = A\)且\(A > 0\),则存在正整数\(N\)使得当\(n > N\)时都有\(a_n > 0\); 若\(a_n \ge 0\)且\(\lim\limits_{n\to\infty} a_n\)存在,则\(\lim\limits_{n\to\infty} a_n \ge 0\).

  \begin{proof}
    对前一个命题的证明:令\(ε = A\),就能找到一个正整数\(N\)使得当\(n > N\)时都有\(\abs*{\,a_n - A\,} < A\),即\(0 < a_n < 2\,A\).对后一个命题的证明:假设\(\lim\limits_{n\to\infty} a_n < 0\),那么根据前一个命题,就存在正整数\(N\)使得当\(n > N\)时都有\(a_n < 0\),这和\(a_n \ge 0\)是矛盾的,所以\(\lim\limits_{n\to\infty} a_n \ge 0\).
  \end{proof}
\end{theorem}

\begin{theorem}[数列极限的四则运算]
  \label{thm:seq4ops}
  若\(\lim\limits_{n\to\infty} a_n = A\)和\(\lim\limits_{n\to\infty} b_n = B\),则
  \begin{enumerate}
    \renewcommand{\labelenumi}{\enumparen{\arabic{enumi}}}
  \item \(\lim\limits_{n\to\infty} \paren{a_n \pm b_n} = A \pm B\),
  \item \(\lim\limits_{n\to\infty} a_n b_n = AB\),
  \item \(\lim\limits_{n\to\infty} {a_n}/{b_n} = A/B\ (B \ne 0)\).
  \end{enumerate}

  \begin{proof}
    对加减法的证明:利用绝对值的三角不等式.对于任意的\(ε\)都能找到\(N_1\)和\(N_2\)使得当\(n > N_1\)时都有\(\abs*{\,a_n-A\,} < ε\)和当\(n > N_2\)时都有\(\abs*{\,b_n-B\,} < ε\).取\(N = \maxb*{\,N_1,\, N_2}\),就有
    \begin{equation*}
      \abs*{\paren{a_n \pm b_n} - \paren{A \pm B}}
      = \abs*{\,a_n - A \pm b_n \mp B\,}
      = \abs*{\paren{a_n - A} \pm \paren{b_n - B}}
      \le \abs*{\,a_n - A\,} + \abs*{\,b_n - B\,}
      < 2ε.
    \end{equation*}
    对乘法的证明:由于定理~\ref{thm:cvgbnd},可以找到数列\(\Seq{b_n}\)的一个界\(M > 0\).对于任意的\(ε\)都能找到\(N_1\)和\(N_2\)使得:当\(n > N_1\)时都有\(\abs*{\,a_n-A\,} < ε\),当\(n > N_2\)时都有\(\abs*{\,b_n-B\,} < ε\).取\(N = \maxb*{\,N_1,\, N_2}\),就有
    \begin{align*}
      \abs*{\,a_n b_n - AB\,}
      &= \abs*{\,a_n b_n - Ab_n + Ab_n - AB\,} \\
      &= \abs*{\paren{a_n - A}b_n + A\paren{b_n - B}} \\
      &\le \abs*{\,a_n-A\,}\,\abs*{\,b_n\,} + \abs*{\,A\,}\,\abs*{\,b_n-B\,} \\
      &\le M\abs*{\,a_n-A\,} + \abs*{\,A\,}\,\abs*{\,b_n-B\,} \\
      &< \paren*{M + \abs*{\,A\,}}ε.
    \end{align*}
    对除法的证明:只需证明\(\lim\limits_{n\to\infty} 1/{b_n} = 1/B\),再利用一次极限的乘法运算即可.对于任意的\(ε\)都能找到一个\(N_1\)使得当\(n > N_1\)时都有\(\abs*{\,b_n-B\,} < ε\).又由于\(B \ne 0\),可以找到一个正整数\(N_2\)使得\(m = \inf\limits_{\,\mathclap{n > N_2}}\Set*{\,\abs*{\,b_n\,}\,} > 0\).取\(N = \maxb*{\,N_1,\, N_2\,}\),就有
    \begin{equation*}
      \abs[\bigg]{\,\frac1{b_n} - \frac1B\,}
      = \abs[\bigg]{\,\frac{B - b_n}{b_n B}\,}
      \le \frac{\abs*{\,b_n - B\,}}{m \abs*{\,B\,}}
      < \frac{ε}{m \abs*{\,B\,}}.
      \qedhere
    \end{equation*}
  \end{proof}
\end{theorem}

\begin{definition*}
  对于任意的正数\(M\)都有正整数\(N\)使得当\(n > N\)时都有\(\abs*{\,a_n\,} > M\).这样的数列\(\Seq{a_n}\)叫作无穷大量,记作\(\lim\limits_{n\to\infty} a_n = \infty\).
  \begin{remark}
    可以用形式化的符号来表示无穷大量的定义,即
    \begin{alignat*}{2}
      \lim_{n\to\infty} a_n &= \infty &{}\iff{}
      &\paren[\big]{\forall M > 0}
        \paren[\big]{\exists N > 0}
        \paren[\big]{\forall n > N}
        \paren[\big]{\abs*{\,a_n\,} > M}, \\
      \lim_{n\to\infty} a_n &= +\infty &{}\iff{}
      &\paren[\big]{\forall M > 0}
        \paren[\big]{\exists N > 0}
        \paren[\big]{\forall n > N}
        \paren[\big]{a_n > M}, \\
      \lim_{n\to\infty} a_n &= -\infty &{}\iff{}
      &\paren[\big]{\forall M > 0}
        \paren[\big]{\exists N > 0}
        \paren[\big]{\forall n > N}
        \paren[\big]{a_n < -M}.
    \end{alignat*}
    其中,\(+\infty\)叫作正无穷大量,\(-\infty\)叫作负无穷大量.
  \end{remark}
\end{definition*}

\begin{definition*}
  极限为\(0\)的数列叫作无穷小量.
\end{definition*}

\begin{example*}
  非零无穷小量的倒数是无穷大量.

  \begin{proof}
    设\(\lim\limits_{n\to\infty} a_n = 0\)且\(a_n \ne 0\).对于任意的\(M > 0\),取\(ε = 1/M > 0\),能找到一个正整数\(N\)使得当\(n > N\)时都有
    \begin{equation*}
      \abs*{\,a_n\,} < ε,
      \txt{从而}
      \abs[\bigg]{\,\frac{1}{\,a_n\!}\,} = \frac{1}{\abs*{\,a_n\,}} > \frac1ε = M.
      \qedhere
    \end{equation*}
  \end{proof}
\end{example*}

\begin{theorem}
  \label{thm:sequnbndsubinf}
  无界数列包含无穷大量子列.

  \begin{proof}
    设数列\(\Seq{a_n}\)无界.证明分为两步:首先构造子列\(\Seq{a_{n_k}}\),然后指出此子列为无穷大量.子列的构造过程是递推的.因为数列无界,所以存在\(n_1\)使得\(\abs[\big]{\,a_{n_1}} > 1\).假设子列前\(k\)项都已构造完毕.还是因为数列无界,所以存在\(n_{k+1} > n_k\)使得\(\abs[\big]{\,a_{n_{k+1}}} > k+1\).若非如此,\(\maxb{\,\abs{\,a_1\,},\, \abs{\,a_2\,},\, \dots,\, \abs{\,a_{n_k}\,},\, k+1\,}\)就是原数列的界,矛盾.子列构造完毕.此子列为无穷大量,因为:对于任意的\(M > 0\)都能找到\(K = \ceil{M}\)使得当\(k > K\)时都有\(\abs[\big]{\,a_{n_k}} > k > K \ge M\).
  \end{proof}
\end{theorem}

\subpdfbookmark{思考}{B1.2.1.P}
\subsection*{思考}

\begin{enumerate}
\item 在数列极限的定义中,\(ε\)是不是一个无限小的正数?正整数\(N\)的选取是不是与\(ε\)有关?

  \ifshowsolp
    对\(ε\)来讲,就是一个任意的正实数,一旦给定,就确定下来了.题目中的无限小的正数,在标准实分析里是不存在的.正整数\(N\)的选取当然是和\(ε\)有关,有时候为了分析方便,还会给\(N\)标上下标\(N_ε\).
  \fi

\item 在数列极限的定义中,如果将\(n > N\)改成\(n \ge N\),是否有影响?如果将\(\forall ε > 0\)改成\(\forall ε \in \paren{0,1}\),是否有影响?

  \ifshowsolp
    都不影响.
  \fi
\end{enumerate}

\ifshowex
\currentpdfbookmark{练习}{B1.2.1.E}
\subsection*{练习}

\begin{enumerate}
\item 下列说法中,与\(\lim\limits_{n\to\infty} a_n = A\)不等价的是\uline{\makebox[10em]{}}.
  \begin{itemize}
    \renewcommand{\labelitemi}{\faCircleThin}
  \item \(
    \paren[\big]{\forall ε > 0}
    \paren[\big]{\exists N \in \N^+}
    \paren[\big]{\forall n > N}
    \paren[\big]{\abs[\big]{\,a_n - A\,} < \sqrt ε}
    \)
  \item \(
    \paren[\big]{\forall k \in \N}
    \paren[\big]{\exists N_k \in \Z^+}
    \paren[\big]{\forall n > N_k}
    \paren[\big]{\abs[\big]{\,a_n - A\,} < 1/2^k}
    \)
  \item \(
    \paren[\big]{\forall ε > 0}
    \paren[\big]{\exists N \in \N^+}
    \paren[\big]{\forall n > N}
    \paren[\big]{\abs[\big]{\,a_n - A\,} < 2ε}
    \)
    \ifshowsol
    \item[\faCircle]
    \else
    \item
    \fi
    \(
    \paren[\big]{\forall ε > 0}
    \paren[\big]{\exists N \in \N^+}
    \paren[\big]{\forall n > N}
    \paren[\big]{\abs[\big]{\,a_n - A\,} < ε/\!\sqrt n}
    \)
  \end{itemize}

  \ifshowsol
    一个反例是\(\lim\limits_{n\to\infty}\paren*{1 + 1/\!\sqrt n} = 1\),但是按照选项~D,它并不收敛.实际上,选项~D是数列收敛的充分不必要条件.
  \fi

\item 下列说法中,与“数列\(\Seq{a_n}\)不收敛于\(A\)”等价的是\uline{\makebox[10em]{}}.
  \begin{itemize}
    \renewcommand{\labelitemi}{\faCircleThin}
    \ifshowsol
    \item[\faCircle]
    \else
    \item
    \fi
    存在\(ε_0 > 0\)使得数列\(\Seq{a_n}\)中有无穷多项满足\(\abs*{\,a_n - A\,} \ge ε_0\)
  \item \(
    \paren[\big]{\exists ε_0 > 0}
    \paren[\big]{\exists N \in \Z^+}
    \paren[\big]{\forall n > N}
    \paren[\big]{\abs[\big]{\,a_n - A\,} \ge ε_0}
    \)
  \item \(
    \paren[\big]{\forall ε > 0}
    \paren[\big]{\exists N}
    \paren[\big]{\forall n > N}
    \paren[\big]{\abs[\big]{\,a_n - A\,} \ge ε_0}
    \)
  \item 数列\(\Seq{a_n}\)中,除有限项外,都满足\(\abs*{\,a_n - A\,} \ge ε_0\),其中\(ε_0\)是某个正数
  \end{itemize}

  \ifshowsol
    选项~B、C和~D都是题干的充分不必要条件.例如数列\(\Seq[\big]{\,A + \frac{1+(-1)^n}{2}\,}\)不收敛于\(A\),但是不满足选项~B、C、D.实际上,选项~B和~D是等价的,然后选项~C是选项~B的充分不必要条件.
  \fi

\item 下列说法中,正确的是\uline{\makebox[10em]{}}.
  \begin{itemize}
    \renewcommand{\labelitemi}{\faCircleThin}
  \item 数列\(\Seq{a_n}\)是否收敛与其前\(1000\)项有关
  \item 数列\(\Seq{a_n}\)是否收敛与其所有项均有关
  \item 数列\(\Seq{a_n}\)是否收敛仅与\(n\)充分大以后的某些项有关
    \ifshowsol
    \item[\faCircle]
    \else
    \item
    \fi
    数列\(\Seq{a_n}\)是否收敛仅与\(n\)充分大以后的所有项有关
  \end{itemize}

\item 下列数列中,极限为\(0\)的是\uline{\makebox[6em]{}}.
  \begin{itemize}
    \renewcommand{\labelitemi}{\faCircleThin}
  \item \(e^n/2^n\)
    \ifshowsol
    \item[\faCircle]
    \else
    \item
    \fi
    \(\paren{-1}^n/n\)
  \item \(n - 1/n\)
  \item \(n \sinp{1/n}\)
  \end{itemize}

\item 下列数列中,极限为\(1\)的是\uline{\makebox[6em]{}}.
  \begin{itemize}
    \renewcommand{\labelitemi}{\faCircleThin}
  \item \(n/a^n\ \paren{a > 1}\)
    \ifshowsol
    \item[\faCircle]
    \else
    \item
    \fi
    \(a^{1/n}\ \paren{a > 1}\)
  \item \(\paren{\sin n^2}/{n}\)
  \item \(\paren{n \sqrt{n+\smash{1}\mathstrut}}/\brkt{\sqrt n \paren{2n-1}}\)
  \end{itemize}

\item 下列说法中,正确的是\uline{\makebox[10em]{}}.
  \begin{itemize}
    \renewcommand{\labelitemi}{\faCircleThin}
  \item 收敛数列的子列极限不一定相同
    \ifshowsol
    \item[\faCircle]
    \else
    \item
    \fi
    收敛数列的极限与其前有限项无关
  \item 数列不收敛则必无收敛子列
  \item 若收敛数列的极限大于零,则数列恒大于零
  \end{itemize}

\item 下列数列中,不是无穷大量的是\uline{\makebox[8em]{}}.
  \begin{itemize}
    \renewcommand{\labelitemi}{\faCircleThin}
  \item \(n/\!\ln n\)
  \item \(-n^2 + n\)
    \ifshowsol
    \item[\faCircle]
    \else
    \item
    \fi
    \(\brkt{n\paren{n^{7/3}+1}}/n^{15/4}\)
  \item \(\paren{-1}^n n^3 + n^2 - 10n\)
  \end{itemize}

\item 下列数列中,无界但不是无穷大量的是\uline{\makebox[6em]{}}.
  \begin{itemize}
    \renewcommand{\labelitemi}{\faCircleThin}
  \item \(n/\!\ln n\)
  \item \(\paren{-1}^n n^2 + n\)
    \ifshowsol
    \item[\faCircle]
    \else
    \item
    \fi
    % https://mathworld.wolfram.com/KroneckersApproximationTheorem.html
    % https://mathworld.wolfram.com/DirichletsApproximationTheorem.html
    % https://en.wikipedia.org/wiki/Dirichlet%27s_approximation_theorem#Proof_by_the_pigeonhole_principle
    % https://artofproblemsolving.com/wiki/index.php/Rational_approximation
    % https://qr.ae/pCnmfz
    % https://mathoverflow.net/q/54668/81708
    % TODO: Raise the question on Math.SE.  If confirmed, report
    \(n \sinp{n^n\!/2}\)
  \item \(e^n\!/n!\)
  \end{itemize}

\item 与命题“当\(n\to\infty\)时,\(a_n\to\infty\)”等价的是\uline{\makebox[10em]{}}.
  \begin{itemize}
    \renewcommand{\labelitemi}{\faCircleThin}
  \item \(
    \paren[\big]{\forall M > 0}
    \paren[\big]{\forall N > 0}
    \paren[\big]{\exists n > N}
    \paren[\big]{\abs*{\,a_n\,} > M}
    \)
  \item \(
    \paren[\big]{\exists M > 0}
    \paren[\big]{\exists N > 0}
    \paren[\big]{\forall n > N}
    \paren[\big]{\abs*{\,a_n\,} > M}
    \)
    \ifshowsol
    \item[\faCircle]
    \else
    \item
    \fi
    \(
    \paren[\big]{\forall M > 0}
    \paren[\big]{\exists N > 0}
    \paren[\big]{\forall n > N}
    \paren[\big]{\abs*{\,a_n\,} > M}
    \)
  \item \(
    \paren[\big]{\exists N > 0}
    \paren[\big]{\forall M > 0}
    \paren[\big]{\forall n > N}
    \paren[\big]{\abs*{\,a_n\,} > M}
    \)
  \end{itemize}

  \ifshowsol
    选项~A其实相当于是说数列\(\Seq{a_n}\)无界,选项~D是一个不可满足的命题.
  \fi

\item 下列说法中,正确的是\uline{\makebox[10em]{}}.
  \begin{itemize}
    \renewcommand{\labelitemi}{\faCircleThin}
  \item 一个数列如果不收敛,则它一定无界
  \item 若\(\lim\limits_{n\to\infty} a_n = A\)且\(A \ge 0\),则存在正整数\(N\)使得当\(n > N\)时都有\(a_n \ge 0\)
  \item 设有两个数列\(\Seq{a_n}\)和\(\Seq{b_n}\),那么\(\lim\limits_{n\to\infty} \paren{a_n - b_n} = \lim\limits_{n\to\infty} a_n - \lim\limits_{n\to\infty} b_n\)
    \ifshowsol
    \item[\faCircle]
    \else
    \item
    \fi
    存在发散数列\(\Seq{a_n}\)使得\(\Seq{\abs{\,a_n\,}}\)收敛
  \end{itemize}

  \ifshowsol
    对于选项~C,加上一个限制条件就成立了,就是这两个数列都收敛.设\(a_n = n + \frac1n, b_n = n\),那么\(\lim\limits_{n\to\infty} \paren{a_n - b_n} = \lim\limits_{n\to\infty} \frac1n = 0 \ne \infty - \infty = \lim\limits_{n\to\infty} a_n - \lim\limits_{n\to\infty} b_n\).
  \fi

\item 设\(a_k \ge 0,\ k = 1, 2, \dots, m\).求\(\lim\limits_{n\to\infty} \paren{a_1^n + a_2^n + \dots + a_m^n}^{1/n} =\)\uline{\makebox[6em]{}}.
  \begin{itemize}
    \renewcommand{\labelitemi}{\faCircleThin}
    \ifshowsol
    \item[\faCircle]
    \else
    \item
    \fi
    \(\max_{1 \le k \le m} \Set{\,a_k\,}\)
  \item \(\min_{1 \le k \le m} \Set{\,a_k\,}\)
  \item \(\paren{a_1 + a_2 + \dots + a_m}/{m}\) % report
  \item \(1\)
  \end{itemize}

  \ifshowsol
    \begin{proof}
      令\(A = \max_{1 \le k \le m} \Set{\,a_k\,}\).若\(A = 0\),则说明\(a_k = 0\)对于所有的\(1 \le k \le m\),所求极限自然也就等于零.若\(A > 0\),就有
      \begin{align*}
        \abs*{\,\paren{a_1^n + a_2^n + \dots + a_m^n}^{1/n} - A\,}
        &= A \abs[\bigg]{\,\brkt[\Big]{\paren[\Big]{\frac{a_1}{A}}^n + \paren[\Big]{\frac{a_2}{A}}^n + \dots + \paren[\Big]{\frac{a_m}{A}}^n}^{\frac1n} - 1\,} \\
        &\le A \paren{m^{1/n} - 1}.
      \end{align*}
      根据极限的四则运算,显然有\(\lim\limits_{n\to\infty} A \paren{m^{1/n} - 1} = 0\).所以对于任意的\(ε > 0\)都存在正整数\(N\)使得当\(n > N\)时都有
      \begin{equation*}
        A \paren{m^{1/n} - 1} = \abs*{\,A \paren{m^{1/n} - 1}} < ε.
        \qedhere
      \end{equation*}
      % 对于任意的\(ε > 0\)取\(N = \ceil*{\ln m/\!\lnp{1+ε/A}}\)就可使得当\(n > N\)时都有
      % \begin{align*}
      %   \frac{\ln m}{\lnp{1+ε/A}}
      %   &\le N < n
      %   && \iff \\
      %   \ln m
      %   &< n \lnp{1+ε/A}
      %   && \iff \\
      %   m
      %   &< \paren{1+ε/A}^n
      %   && \iff \\
      %   A \paren{m^{1/n} - 1}
      %   &< ε.
      %   &&\qedhere
      % \end{align*}
    \end{proof}
    % \begin{remark}
    %   上述证明中,最后一步也可以取\(N = \ceil{A(m-1)/ε}\),然后使用一次伯努利不等式即可.
    % \end{remark}
  \fi
\end{enumerate}
\fi

\section{数列极限存在的充分条件}

\begin{theorem}[数列极限的夹逼定理]
  \label{thm:squeeze}
  若数列\(\Seq{a_n},\, \Seq{b_n},\, \Seq{c_n}\)满足:
  \begin{enumerate}[topsep=0ex,itemsep=0ex]
    \renewcommand{\labelenumi}{\enumparen{\arabic{enumi}}}
  \item 存在正整数\(N\)使得当\(n > N\)时都有\(a_n \le b_n \le c_n\),
  \item \(\lim\limits_{n\to\infty} a_n = \lim\limits_{n\to\infty} c_n = A\),
  \end{enumerate}
  则\(\lim\limits_{n\to\infty} b_n = A\).

  \begin{proof}
    对于任意的\(ε > 0\)都存在正整数\(N_1\)和\(N_2\)使得:当\(n > N_1\)时都有\(\abs*{\,a_n - A\,} < ε\),当\(n > N_2\)时都有\(\abs*{\,c_n - A\,} < ε\).取\(N = \maxb*{\,N_1,\, N_2}\).当\(n > N\)时就有
    \begin{equation*}
      A - ε < a_n \le b_n \le c_n < A + ε,
      \txt{即}
      \abs*{\,b_n - A\,} < ε.
      \qedhere
    \end{equation*}
  \end{proof}
\end{theorem}

\begin{example}
  \label{eg:factexp}
  求\(\lim\limits_{n\to\infty} {a^n}\!/{n!}\). % \rule[-2ex]{0ex}{0ex}

  \begin{remark}
    当\(a = 0\)时,极限显然是零.当\(a \ne 0\)时,构造数列
    \begin{equation*}
      c_n =
      \begin{cases}
        C_a, & n \le n_a, \\
        C_a \cdot \paren[\bigg]{\frac{a}{n_a + 1}}^{n-n_a}, & n > n_a,
      \end{cases}
    \end{equation*}
    其中\(n_a = \ceil{\abs{\,a\,}},\ C_a = a^{n_a}\!/n_a!\).自然有\(0 \le \abs*{\,a^n\!/n!\,} \le c_n\)且\(\lim\limits_{n\to\infty} c_n = 0\).根据夹逼定理,自然有\(\lim\limits_{n\to\infty} \abs{\,a^n/n!\,} = 0\),从而\(\lim\limits_{n\to\infty} a^n/n! = 0\).
  \end{remark}

  % \begin{remark}
  %   当\(a = 0\)时,极限显然是零.当\(a \ne 0\)时,我们构造数列\(a_n\)和\(c_n\)使得\(a_n \le a^n\!/n! \le c_n\)且\(\lim\limits_{n\to\infty} a_n = \lim\limits_{n\to\infty} c_n = 0\).令\(n_a = \ceil{\abs{\,a\,}},\ C_a = \frac{a^{n_a}}{n_a!}\),构造数列
  %   \begin{equation*}
  %     d_n =
  %     \begin{cases}
  %       C_a, & n \le n_a, \\
  %       C_a \cdot \paren[\bigg]{\frac{a}{n_a + 1}}^{n-n_a}, & n > n_a,
  %     \end{cases}
  %   \end{equation*}
  %   易知\(\lim\limits_{n\to\infty} d_n = 0\).当\(a > 0\)时,只要构造数列\(a_n = 0\)和\(c_n = d_n\)即可.当\(a < 0\)时,只要构造数列
  %   \begin{equation*}
  %     a_n =
  %     \begin{cases}
  %       0, & \text{\(n\)是偶数时}, \\
  %       d_n, & \text{\(n\)是奇数时},
  %     \end{cases}
  %     \quad\text{和}\quad
  %     c_n =
  %     \begin{cases}
  %       d_n, & \text{\(n\)是偶数时}, \\
  %       0, & \text{\(n\)是奇数时},
  %     \end{cases}
  %   \end{equation*}
  %   就有\(a_n \le a^n\!/n! \le c_n\)且\(\lim\limits_{n\to\infty} a_n = \lim\limits_{n\to\infty} c_n = 0\).综上所述,\(\lim\limits_{n\to\infty} \frac{a^n}{n!} = 0\).\rule[-2ex]{0ex}{0ex}

  %   上面是为了直接套用夹逼定理的形式,实际上一种常见的方法是:欲证\(\lim\limits_{n\to\infty} a_n = A\),只需证明\(0 \le \abs*{\,a_n - A\,} \le b_n\)且\(\lim\limits_{n\to\infty} b_n = 0\)即可.
  % \end{remark}
\end{example}

\begin{example*}
  求\(\lim\limits_{n\to\infty} \sum\limits_{k=1}^n \! {k}/\paren{n^2+k}\).

  \begin{remark}
    构造数列
    \begin{equation*}
      a_n
      = \sum_{k=1}^n \frac{k}{n^2 + n}
      = \frac{1}{n^2 + n} \cdot \frac{n(n+1)}{2}
      = \frac12
      \quad\text{和}\quad
      c_n
      = \sum_{k=1}^n \frac{k}{n^2}
      = \frac{1}{n^2} \cdot \frac{n(n+1)}{2}
      = \frac{1 + 1/n}{2},
    \end{equation*}
    易知\(a_n \le \sum\limits_{k=1}^n \frac{k}{n^2+k} \le c_n\)且\(\lim\limits_{n\to\infty} a_n = \lim\limits_{n\to\infty} c_n = \frac12\).\rule[-3ex]{0ex}{0ex}
  \end{remark}
\end{example*}

\begin{theorem*}[单调有界收敛定理]
  若数列\(\Seq{a_n}\)单调递增且有上界,则其收敛且\(\lim\limits_{n\to\infty} a_n = \supb{\,a_n}\); 若数列\(\Seq{b_n}\)单调递减且有下界,则其收敛且\(\lim\limits_{n\to\infty} b_n = \infb{\,b_n}\).

  \begin{proof}
    只证单增的情况,单减的情况类似.因为数列有界,所以根据\kenten{确界存在公理},有上确界\(M = \supb{\,a_n}\).那么对于任意的\(ε > 0\)都存在正整数\(N\)使得\(a_N > M - ε\).当\(n > N\)时,自然有
    \begin{equation*}
      M - ε < a_N < a_n \le M < M + ε,
      \txt{即}
      \abs*{\,a_n - M\,} < ε.
      \qedhere
    \end{equation*}
  \end{proof}
\end{theorem*}

\begin{theorem}
  \label{thm:seqe}
  极限\(\lim\limits_{n\to\infty} \paren[\big]{1 + \frac1n}^n\)存在.

  \begin{proof}
    记\(a_n = \paren{1+1/n}^n\).根据单调有界收敛定理,只需证明对于所有正整数\(n\)都有\(a_n < a_{n+1}\)且\(a_n < 3\)即可.根据二项式定理,有
    \begin{align*}
      a_n
      &= \paren[\bigg]{1+\frac1n}^n
        = 1 + \binom{n}{1} \frac1n  + \binom{n}{2} \frac1{n^2} + \dots + \binom{n}{n} \frac1{n^n} \\
      &= 1 + 1 + \frac{n(n-1)}{2!\,n^2} + \dots + \frac{n(n-1)\dotsm1}{n!\,n^n} \\
      &= 2 + \frac1{2!} \cdot 1 \cdot \paren[\bigg]{1 - \frac1n} + \dots
        + \frac1{n!} \cdot 1 \cdot \paren[\bigg]{1 - \frac1n} \dotsm \paren[\bigg]{1 - \frac{n-1}n} \\
      &< 2 + \frac1{2!} \cdot 1 \cdot \paren[\bigg]{1 - \frac1{n+1}} + \dots
        + \frac1{(n+1)!} \cdot 1 \cdot \paren[\bigg]{1 - \frac1{n+1}} \dotsm \paren[\bigg]{1 - \frac{n}{n+1}} \\
      &= a_{n+1}.
    \end{align*}
    要证明\(a_n < 3\),只需证明\(1/2! + 1/3! + \dots + 1/n! < 1\)即可.有
    \begin{align*}
      \frac1{2!} + \frac1{3!} + \dots + \frac1{n!}
      &\le \frac{1}{1 \cdot 2} + \frac{1}{2 \cdot 3} + \dots + \frac{1}{\paren{n-1}n} \\
      &= \paren[\bigg]{1 - \frac12} + \paren[\bigg]{\frac12 - \frac13} + \dots + \paren[\bigg]{\frac1{n-1} - \frac1n} \\
      &= 1 - \frac1n < 1. \qedhere
    \end{align*}
  \end{proof}

  \begin{remark}
    实际上,这个极限就是自然常数\(e\).
  \end{remark}
\end{theorem}

\begin{example*}
  令\(a_n = 1 + \frac12 + \dots + \frac1n - \ln n\).证明数列\(\Seq{a_n}\)的极限存在.\rule{0ex}{3.5ex}

  \begin{proof}
    当\(x > 0\)时,有
    \begin{equation*}
      \frac{x}{1+x} < \lnp{1+x} < x.
    \end{equation*}
    因此有
    \begin{align*}
      a_{n+1} - a_n
      &= \frac{1}{1+n} - \lnp{1+n} + \ln n \\
      &= \frac{1/n}{1+1/n} - \lnp[\bigg]{1+\frac1n} < 0
    \end{align*}
    和
    \begin{align*}
      a_n
      &= 1 + \frac12 + \dots + \frac1n - \ln n \\
      &> \lnp{1+1} + \lnp[\bigg]{1+\frac12} + \dots + \lnp[\bigg]{1+\frac1n} - \ln n \\
      &= \lnp[\bigg]{2 \cdot \frac32 \dotsm \frac{n+1}{n}} - \ln n \\
      &= \lnp{n + 1} - \ln n > 0. \qedhere
    \end{align*}
  \end{proof}

  \begin{remark}
    此数列的极限叫作Euler-Mascheroni常数,一般记作\(γ\).目前,尚不知道此数是否为无理数或者超越数.
  \end{remark}
\end{example*}

\begin{example*}
  证明极限\(\lim\limits_{n\to\infty} \sum\limits_{k=1}^n \! \frac1k\)不存在.

  \begin{proof}
    实际上,\(\lim\limits_{n\to\infty} \sum\limits_{k=1}^n \! \frac1k = +∞\).对于任意的\(M > 0\)都存在正整数\(N = \ceil[\big]{e^M-1}\)使得当\(n > N\)时都有
    \begin{equation*}
      \sum_{k=1}^n \frac1{\,k\,}
      > \sum_{k=1}^n \lnp[\bigg]{1+\frac1{\,k\,}}
      = \lnp{n+1} > M.
      \qedhere
    \end{equation*}
  \end{proof}
\end{example*}

\begin{example*}
  令\(a_n = \sum\limits_{k=1}^n 1/k^p\).试讨论当\(p > 0\)时数列\(\Seq{a_n}\)的敛散性.

  \begin{proof}
    当\(0 < p \le 1\)时,有\(k^p \le k\).所以
    \begin{equation*}
      a_n = \sum_{k=1}^n \frac1{\,k^p} \ge \sum_{k=1}^n \frac1{\,k\,}
      > \sum_{k=1}^n \lnp[\bigg]{1+\frac1{\,k\,}} = \lnp{n+1}.
    \end{equation*}
    对于任意的\(M > 0\),取\(N = \ceil[\big]{e^M - 1}\),那么当\(n > N\)时,自然有\(a_n > \lnp{n+1} > M\).所以有
    \begin{equation*}
      \lim_{n\to\infty} a_n = +\infty.
    \end{equation*}
    当\(p > 1\)时,由\(a_{n+1} - a_n = 1/(n+1)^p > 0\)可知,此数列单调递增,只需证明其有上界即可.令\(\ell = \floor[\big]{\log_2 n} + 1\),有
    \begin{equation*}
      a_n
      = \sum_{k=1}^n \frac1{\,k^p}
      \le \sum_{k=1}^{2^\ell-1} \frac1{\,k^p}
      = \sum_{m=0}^{\ell-1} \! \sum_{\,j=2^m\!}^{\,2^{m+1}-1\!\!} \frac{1}{\,j^p}
      \le \sum_{m=0}^{\ell-1} \frac{2^m}{\paren{2^m}^p}
      = \sum_{m=0}^{\ell-1} \paren[\big]{2^{1-p}}^m
      % = \frac{1-\paren{2^{1-p}}^\ell}{1-2^{1-p}}
      < \frac{1}{1-2^{1-p}}.
    \end{equation*}
    由单调有界收敛定理可知,数列\(\Seq{a_n}\)收敛.
  \end{proof}

  \begin{remark}
    实际上,当\(0 < p \le 1\)时,也可以同样用处理\(p > 1\)时的思路.令\(\ell = \floor[\big]{\log_2 n}\),则
    \begin{equation*}
      a_n
      = \sum_{k=1}^n \frac1{\,k^p}
      \ge \sum_{k=1}^n \frac1{\,k\,}
      > \sum_{k=1}^{2^\ell-1} \frac1{\,k}
      = \sum_{m=1}^{\ell} \! \sum_{\,\,j=2^{m-1}\!\!}^{2^m-1} \frac{1}{\,j\,}
      \ge \sum_{m=1\,}^{\ell} \frac{2^{m-1}}{2^m}
      = \frac{\ell}{\,2\,}
      = \frac{\floor{\log_2 n}}{2}.
    \end{equation*}
  \end{remark}
\end{example*}

\subpdfbookmark{思考}{B1.2.2.P}
\subsection*{思考}

\begin{enumerate}
\item 若单调递增的数列\(\Seq{a_n}\)没有上界,这个数列是不是无穷大量?能否给出严格证明?

  \ifshowsolp
    是无穷大量,证明如下.

    \begin{proof}
      因为数列\(\Seq{a_n}\)没有上界,所以对任意的\(M > 0\)都存在一个正整数\(N\)使得\(a_N > M\).此时,对于所有的\(n > N\),因为数列\(\Seq{a_n}\)单调递增,所以
      \begin{equation*}
        a_n \ge a_N > M. \qedhere
      \end{equation*}
    \end{proof}
  \fi

\item 若数列\(\Seq{a_n}\)是正无穷大量,则\(\Seq{a_n}\)一定单调递增吗?

  \ifshowsolp
    未必.\(a_n = n + (-1)^n\),易证\(\lim\limits_{n\to\infty} a_n = +\infty\).此时,对于任意的正偶数\(k\)都有
    \begin{equation*}
      a_{k+1} = (k+1) + (-1)^{k+1} = k < k + 1 = k + (-1)^k = a_k.
    \end{equation*}
  \fi
\end{enumerate}

\ifshowex
\currentpdfbookmark{练习}{B1.2.2.E}
\subsection*{练习}

\begin{enumerate}
\item 下列数列中,收敛但极限不为\(1\)的是\uline{\makebox[6em]{}}.
  \begin{itemize}
    \renewcommand{\labelitemi}{\faCircleThin}
  \item \(\paren{2+1/n}^{1/n}\)
  \item \(n^{1/n}\)
    \ifshowsol
    \item[\faCircle]
    \else
    \item
    \fi
    \(\frac{1}{n^2+1} + \frac{2}{n^2+2} + \dots + \frac{n}{n^2+n}\)
  \item \(\paren{n!}^2\!/n^n\)
  \end{itemize}

  \ifshowsol
    对选项~B的证明:
    \begin{proof}
      对于任意的\(ε > 0\),取\(N = \ceil[\big]{2/ε^2 + 1}\),那么当\(n > N\)时都有
      \begin{gather*}
        n
        < \frac{n(n-1)}{2} ε^2
        < 1 + nε + \frac{n(n-1)}{2} ε^2 + \dots + ε^n
        = (1+ε)^n, \\
        \shortintertext{即}
        n^{1/n} < 1 + ε.
        \qedhere
      \end{gather*}
    \end{proof}
    对选项~D的证明:
    \begin{proof}
      当\(n \ge 2\)时,有
      \begingroup
      \addtolength{\jot}{1ex}
      \begin{align*}
        \frac{(n!)^2}{n^n}
        &= \frac{n}{n} \cdot \frac{2(n-1)}{n} \cdot \frac{3(n-2)}{n}
          \dotsm \frac{2(n-1)}{n} \cdot \frac{n}{n} \\
        &=
          \begin{dcases}
            \brkt[\bigg]{2\paren[\bigg]{1-\frac1n} \cdot 3\paren[\bigg]{1-\frac2n}
            \dotsm \floor*{\frac{n}{2}} \paren[\bigg]{1-\frac{\floor{n/2}-1}{n}}}^2,
            & \text{\(n\)是偶数时,} \\
            \brkt[\bigg]{2\paren[\bigg]{1-\frac1n} \cdot 3\paren[\bigg]{1-\frac2n}
            \dotsm \floor*{\frac{n}{2}} \paren[\bigg]{1-\frac{\floor{n/2}-1}{n}}}^2
            \cdot \paren[\bigg]{\frac{\ceil{n/2}}{n}}^2,
            & \text{\(n\)是奇数时,}
          \end{dcases} \\
        &> \paren[\bigg]{\frac{\floor{n/2}!}{2^{\floor{n/2}+1}}}^2.
      \end{align*}
      \endgroup
      易知\(\lim\limits_{n\to\infty} {(n!)^2}\!/{n^n} = +\infty\).
    \end{proof}

    选项~A和~C可以直接套用夹逼定理.
  \fi

  % report: Cauchy sequence appeared before its definition
\item 以下说法中,错误的是\uline{\makebox[10em]{}}.
  \begin{itemize}
    \renewcommand{\labelitemi}{\faCircleThin}
    \addtolength{\itemsep}{1ex}
  \item 设\(x_1 > 0,\ y_1 > 0,\ x_{n+1} = \sqrt{x_n \, \smash[b]{y_n} \mathstrut},\ y_{n+1} = (x_n+y_n)/2 \; (n \in \N^+)\).那么数列\(\Seq{x_n}\)与\(\Seq{y_n}\)收敛于同一个实数
    \ifshowsol
    \item[\faCircle]
    \else
    \item
    \fi
    若对所有的\(p \in \N^+\)都有\(\lim\limits_{n\to\infty} \abs[\big]{\,a_{n+p}-a_n} = 0\),则数列\(\Seq{a_n}\)是柯西列.
  \item 极限\(\lim\limits_{n\to\infty} \smash{\sum\limits_{k=1}^n} \frac{\sin k}{2^k}\)存在
  \item 若\(\lim\limits_{n\to\infty} \abs*{\,\frac{a_{n+1}}{a_n}\,} = q < 1\),则\(\lim\limits_{n\to\infty} a_n = 0\)
  \end{itemize}

  \ifshowsol
    对选项~A的证明:
    \begin{proof}
      对于\(x_1 = y_1\)的情况,显然有\(\lim\limits_{n\to\infty} x_n = \lim\limits_{n\to\infty} y_n = x_1\).对于\(x_1 \ne y_1\)的情况,当\(n \ge 2\)时,有
      \begin{equation*}
        x_n < x_{n+1} < y_{n+1} < x_n.
      \end{equation*}
      从第2项开始,数列\(\Seq{x_n}\)单调递增有上界,数列\(\Seq{y_n}\)单调递减有下界.由单调有界收敛定理可知,它们都收敛.记数列\(\Seq{y_n}\)的极限为\(A\).对于任意的\(ε>0\)都存在正整数\(N\)使得当\(n > N\)时都有
      \begin{gather*}
        x_n < A < \frac{x_n + y_n}{2}< y_n < A+ε,
        \shortintertext{从而}
        A-ε < A - (y_n-A) = 2\,A - y_n < x_n < A.
        \qedhere
      \end{gather*}
    \end{proof}
    选项~B的反例:令\(a_n = \ln n\),则对于所有的\(p \in \N^+\)都有
    \begin{equation*}
      \lim_{n\to\infty} \abs*{\,\ln\paren*{n+p}- \ln n\,}
      = \lim_{n\to\infty} \abs*{\,\ln\paren[\Big]{1+\frac pn}\,}
      = 0.
    \end{equation*}
    但是显然数列\(\Seq{a_n}\)不是柯西列.

    对选项~D的证明:
    \begin{proof}
      存在正整数\(N_1\)使得当\(n \ge N_1\)时都有
      \begin{gather*}
        \abs[\bigg]{\,\frac{a_{n+1}}{a_n}\,} < \frac{1+q}{2} = q + \frac{1-q}{2}, \\
        \shortintertext{从而当\(n > N\)时有}
        0 < \abs[\big]{\,a_n} < \abs[\big]{\,a_{N_1}} \, \paren[\bigg]{\frac{1+q}{2}}^{n-N_1}.
      \end{gather*}
      易知\(\lim\limits_{n\to\infty} \abs[\big]{\,a_{N_1}} \, \paren[\Big]{\frac{1+q}{2}}^{n-N_1} = 0\),所以由夹逼定理可知\(\lim_{n\to\infty} \abs*{\,a_n\,} = 0\),从而
      \begin{equation*}
        \lim_{n\to\infty} a_n = 0.
        \qedhere
      \end{equation*}
    \end{proof}
  \fi

\item 下列说法中,错误的是\uline{\makebox[10em]{}}.
  \begin{itemize}
    \renewcommand{\labelitemi}{\faCircleThin}
    \addtolength{\itemsep}{.67ex}
  \item 若\(k\)是某一个正整数,则\(\lim\limits_{n\to\infty} x_n = a\)的充分必要条件是\(\lim\limits_{n\to\infty} x_{n+k} = a\)
    \ifshowsol
    \item[\faCircle]
    \else
    \item
    \fi
    若\(\lim\limits_{n\to\infty} a_n = A,\ b_n = \frac{a_1 + a_2 + \dots + a_n}{n}\),则数列\(\Seq{b_n}\)不一定收敛.
  \item 若\(\lim\limits_{n\to\infty} a_n = a > 0,\ b_n = \sqrt[n]{a_1 a_2 \dotsm a_n \mathstrut}\)有意义且\(a_n \ne 0\),则\(\lim\limits_{n\to\infty} b_n = a\)
  \item 单调有界数列是柯西列
  \end{itemize}

  \ifshowsol
    选项~B参见例~\ref{eg:limavg}.

    对选项~C的证明:
    \begin{proof}
      对于任意的\(0 < ε < a\),因为数列\(\Seq{a_n}\)收敛于\(a\),所以存在正偶数\(N_1\)使得当\(n > N_1\)时都有
      \begin{equation*}
        \sqrt[n]{\,c\,\mathstrut} \cdot \sqrt[n]{(a-ε)^{n-N_1}}
        < \sqrt[n]{\,a_1 a_2 \dotsm a_n \mathstrut}
        = \sqrt[n]{\,c\,\mathstrut} \cdot \sqrt[n]{\,\smash{a_{N_1+1} \dotsm a_n} \mathstrut}
        < \sqrt[n]{\,c\,\mathstrut} \cdot \sqrt[n]{(a+ε)^{n-N_1}},
      \end{equation*}
      其中\(c = a_1 a_2 \dotsm a_{N_1}\).又因为
      \begin{gather*}
        \lim_{n\to\infty} \sqrt[n]{\,c\,\mathstrut} \cdot \sqrt[n]{(a-ε)^{n-N_1}} = \lim_{n\to\infty} (a-ε)^{1-N_1/n} = a-ε,\\
        \lim_{n\to\infty} \sqrt[n]{\,c\,\mathstrut} \cdot \sqrt[n]{(a+ε)^{n-N_1}} = \lim_{n\to\infty} (a+ε)^{1-N_1/n} = a+ε,\\
        \intertext{所以存在正整数\(N_2\)使得当\(n > N_2\)时都有}
        a - 2ε
        < \sqrt[n]{\,c\,\mathstrut} \cdot \sqrt[n]{(a-ε)^{n-N_1}}
        < \sqrt[n]{\,c\,\mathstrut} \cdot \sqrt[n]{(a+ε)^{n-N_1}}
        < a + 2ε.
      \end{gather*}
      此时,取\(N = \maxb*{\,N_1,\, N_2}\),那么就有
      \begin{equation*}
        \abs*{\,\sqrt[n]{\,a_1 a_2 \dotsm a_n \mathstrut} - a\,} < 2ε.
        \qedhere
      \end{equation*}
      \begin{remark}
        其他的证明方法还有:\enumparen{1}利用AM-GM不等式和夹逼定理;\enumparen{2}取对数后利用例~\ref{eg:limavg}或者Stolz-Cesàro定理.
      \end{remark}
    \end{proof}
  \fi

\item 下列说法中,正确的是\uline{\makebox[10em]{}}.
  \begin{itemize}
    \renewcommand{\labelitemi}{\faCircleThin}
    \ifshowsol
    \item[\faCircle]
    \else
    \item
    \fi
    单调递增数列要么收敛,要么是无穷大量
  \item 数列若不单调有界,则必不收敛
  \item 存在不收敛的柯西列
  \item 收敛数列不一定有界
  \end{itemize}

\item 已知\(\lim\limits_{n\to\infty} a_n = 1,\ \lim\limits_{n\to\infty} b_n = 2\),求\(\lim\limits_{n\to\infty} \frac{a_1 b_n + a_2 b_{n-1} + \dots + a_n b_1}{n}\).

  \ifshowsol
    可以模仿例~\ref{eg:limavg}的思路来证明这个极限的值是\(2\).
  \fi
\end{enumerate}
\fi

\section{Bolzano--Weierstrass定理与Cauchy收敛准则}

\begin{theorem}
  \label{thm:seqcvgsubseq}
  数列\(\Seq{a_n}\)的所有\kenten{真}子列都收敛是其自身收敛的充分必要条件.

  \begin{remark}
    易证必要性.至于充分性,证明所有真子列都收敛于同一个数即可.
  \end{remark}
\end{theorem}

\begin{example*}
  % https://mathworld.wolfram.com/ContinuedFraction.html
  设\(a_1 = 2,\ a_{n+1} = 2 + {1}/{a_n}\).试讨论数列\(\Seq{a_n}\)的敛散性.

  \begin{remark}
    数列\(\Seq{a_n}\)收敛于\(1+\sqrt2\).证明思路:子列\(\Seq{a_{2n-1}}\)单调递增有上界,子列\(\Seq{a_{2n}}\)单调递减有下界.然后证明这两个子列的极限相等.这就说明原数列收敛.
  \end{remark}
  \begin{proof}
    因为\(2 < a_n < 3,\ 2 + 1/\paren[\big]{1+\sqrt2} = 1 + \sqrt2\)且
    \begin{align*}
      a_{n+2} - a_n
      &= 2 + \frac{1}{2+1/a_n} - a_n \\
      &= \frac{5a_n + 2}{2a_n + 1} - a_n \\
      &= - \frac{2}{2a_n + 1}\paren[\big]{a_n^2 - 2a_n - 1} \\
      &= - \frac{2}{2a_n + 1}\paren[\big]{a_n - 1 - \sqrt2}\paren[\big]{a_n - 1 + \sqrt2},
    \end{align*}
    所以当\(a_n > 1 + \sqrt2\)时有\(a_{n+2} < a_n,\ a_{n+1} < 1 + \sqrt2\)且当\(a_n < 1 + \sqrt2\)时有\(a_{n+2} > a_n,\ a_{n+1} > 1 + \sqrt2\).又因为\(a_1 = 2\),所以子列\(\Seq{a_{2n-1}}\)单调递增有上界、子列\(\Seq{a_{2n}}\)单调递减有下界.由单调有界收敛定理可知,它们都收敛.

    记\(\lim\limits_{n\to\infty} a_{2n-1} = A,\ \lim\limits_{n\to\infty} a_{2n} = B\).由极限的四则运算可知,有
    \begin{gather*}
      A = 2 + \frac1B \qand B = 2 + \frac1A,
      \shortintertext{从而}
      (AB-1)(B-A) = 0.
    \end{gather*}
    这就说明了\(A = B\).
  \end{proof}
\end{example*}

\begin{definition*}
  闭区间序列\(\Seq{\brkt{\,a_n, b_n}\,}\)是区间套,当且仅当:\(\lim\limits_{n\to\infty} (b_n - a_n) = 0\)且对于所有的正整数\(n\)都有\(\brkt{\,a_{n+1}, b_{n+1}} ⊂ \brkt{\,a_n, b_n}\).
\end{definition*}

\begin{theorem}[区间套定理]
  % report: 单调递增、单调递减的定义和这里子集的定义不一致
  \label{thm:nested}
  若\(\Seq{\brkt{\,a_n, b_n}\,}\)是区间套,则存在唯一实数\(ξ\)使得\(ξ \in \brkt{\,a_n, b_n}\)对于所有的正整数\(n\)都成立.

  \begin{remark}
    可以使用单调有界收敛定理来证明此定理.
  \end{remark}
  \begin{proof}
    因为\(\Seq{\brkt{\,a_n, b_n}\,}\)是区间套,所以对所有的正整数\(n\)都有
    \begin{equation*}
      a_n \le a_{n+1} \le b_{n+1} \le b_n.
    \end{equation*}
    由单调有界收敛定理可知,数列\(\Seq{a_n}\)和\(\Seq{b_n}\)都收敛.记\(\lim\limits_{n\to\infty} a_n = A\)且\(\lim\limits_{n\to\infty} b_n = B\).又因为
    \begin{equation*}
      \lim_{n\to\infty} (b_n - a_n) = B - A = 0,
    \end{equation*}
    所以\(A = B\).% 实际上,有\(ξ = A = \supb{\,a_n} = B = \infb{\,b_n} \).
    记\(ξ = A\).对于任意的非\(ξ\)的实数\(η\),都存在正整数\(N\)使得当\(n > N\)时都有
    \begin{equation*}
      a_n \le ξ \le b_n < η
      \txt{或者}
      η < a_n \le ξ \le b_n.
    \end{equation*}
    这就说明了\(ξ\)是唯一的这样的实数.
  \end{proof}
\end{theorem}

\begin{example*}
  从区间套定理可以推出确界存在公理.

  \begin{remark}
    可以通过二分法来构造区间套,进而推出确界存在公理.这就说明确界存在公理、单调有界定理、区间套定理在逻辑上是等价的.
  \end{remark}
  \begin{proof}
    设\(A\)为非空实数集且有上界.因为\(A\)非空,所以存在元素\(a_0 \in A\);又因为其上界,所以存在实数\(b_0\)为其有上界.构造数列\(\Seq{a_n}\)和\(\Seq{b_n}\)满足
    \begin{equation*}
      a_{n+1} =
      \begin{cases*}
        m_n, & \(m_n\)不是\(A\)的上界, \\
        a_n, & \(m_n\)是\(A\)的上界
      \end{cases*}
      \qand
      b_{n+1} =
      \begin{cases*}
        b_n, & \(m_n\)不是\(A\)的上界, \\
        m_n, & \(m_n\)是\(A\)的上界,
      \end{cases*}
    \end{equation*}
    其中\(m_n = (a_n + b_n)/2\).显然,序列\(\Seq{\brkt{\,a_n, b_n}\,}\)是区间套,因为
    \begin{gather*}
      a_n \le a_{n+1} \le b_{n+1} \le b_n
      \txt{且}
      \lim_{n\to\infty} (b_n - a_n) = \lim_{n\to\infty} \frac{b_0 - a_0}{2^n} = 0.
    \end{gather*}
    存在唯一实数\(ξ\)使得对于所有\(n\)都有\(ξ \in \brkt{\,a_n, b_n}\).假设实数\(ξ\)不是集合\(A\)的上界.那么就存在\(b \in A\)满足\(ξ < b\).取\(ε_1 = b - ξ\),则存在正整数\(N_1\)使得当\(n > N_1\)时都有
    \begin{equation*}
      b_n < a_n + ε_1 \le ξ + ε_1 = b.
    \end{equation*}
    这是不可能的,因为我们的构造方法保证了每个\(b_n\)都是集合\(A\)的上界.这就说明了\(ξ\)是上界.对于任意的\(ε > 0\)都存在正整数\(N_2\)使得当\(n \ge N_2\)时都有
    \begin{equation*}
      ξ - ε \le b_n - ε < a_n.
    \end{equation*}
    因为我们的构造方法保证了对于每个\(a_n\)都存在\(x \in A\)满足\(a_n \le x\),所以存在\(a \in A\)满足\(a_{N_2} \le a\),从而
    \begin{equation*}
      ξ - ε < a_{N_2} \le a.
    \end{equation*}
    这就说明了\(ξ\)是上确界.

    证明下确界的方法类似.
  \end{proof}
\end{example*}

\begin{theorem}[Bolzano--Weierstrass定理]
  \label{thm:bw}
  若数列\(\Seq{a_n}\)有界,则存在收敛子列\(\Seq{a_{n_k}}\).

  \begin{remark}
    还是可以通过构造区间套来证明此定理的.
  \end{remark}
  \begin{proof}
    因为数列\(\Seq{a_n}\)有界,所以其既有上界又有下界.令\(b_1\)和\(c_1\)分别为其某个下界和上界.构造数列\(\Seq{b_n}\)和\(\Seq{c_n}\)满足
    \begin{equation*}
      b_{n+1} =
      \begin{cases*}
        m_n, & 集合\(\Set{a_n} \cap \brkt{m_n, c_n}\)无穷, \\
        b_n, & 集合\(\Set{a_n} \cap \brkt{m_n, c_n}\)有穷
      \end{cases*}
      \qand
      c_{n+1} =
      \begin{cases*}
        c_n, & 集合\(\Set{a_n} \cap \brkt{m_n, c_n}\)无穷, \\
        m_n, & 集合\(\Set{a_n} \cap \brkt{m_n, c_n}\)有穷,
      \end{cases*}
    \end{equation*}
    其中\(m_n = (b_n + c_n)/2\).易知序列\(\Seq{\brkt{\,b_n, c_n}\,}\)是区间套.令\(n_1 = 1\).假设\(a_{n_k} \in \brkt{\,b_k, c_k\,}\).我们的构造方式保证了存在\(n_{k+1}\)满足
    \begin{equation*}
      n_{k} < n_{k+1}
      \txt{且}
      a_{n_{k+1}} \in \brkt{\,b_{k+1}, c_{k+1}}.
    \end{equation*}
    这就说明了存在数列\(\Seq{n_k\,}\)使得对于所有\(k\)都有\(a_{n_k} \in \brkt{\,b_k, c_k\,}\).又由定理~\ref{thm:squeeze}和~\ref{thm:nested}可知,存在\(ξ\)使得
    \begin{equation*}
      \lim_{k\to∞} a_{n_k} = \lim_{n\to∞} b_n = \lim_{n\to∞} c_n = ξ.
      \qedhere
    \end{equation*}
  \end{proof}
\end{theorem}

\begin{example}
  \label{eg:seqbndcmnidx}
  设数列\(\Seq{a_n}\)和\(\Seq{b_n}\)有界.存在共同的下标集\(\Set{n_k}\)使得子列\(\Seq{a_{n_k}}\)和\(\Seq{b_{n_k}}\)都收敛.

  \begin{proof}
    由上述定理可知,存在下标集\(\Set{\bar n_k}\)使得子列\(\Seq{a_{\bar n_k}}\)收敛.再对有界子列\(\Seq{b_{\bar n_k}}\)使用一次此定理,从而存在下标集\(\Set{\tilde n_k}\)使得子列\(\Seq{a_{\bar n_{\tilde n_k}}}\)和\(\Seq{b_{\bar n_{\tilde n_k}}}\)都收敛.记\(n_k = \bar n_{\tilde n_k}\)即可.
  \end{proof}
\end{example}

% https://mathworld.wolfram.com/CauchySequence.html
% https://en.wikipedia.org/wiki/Cauchy_sequence
\begin{definition*}
  数列\(\Seq{a_n}\)是柯西列(Cauchy sequence),当且仅当:对于任意的\(ε > 0\)都存在正整数\(N\)使得当\(m > N,\, n > N\)时都有\(\abs*{\,a_m - a_n} < ε\).
\end{definition*}

\begin{definition*}[柯西列的等价定义]
  \begin{math}
    \paren[\big]{\forall ε > 0}
    \paren[\big]{\exists N > 0}
    \paren[\big]{\forall n > N}
    \paren[\big]{\forall p > 0}
    \paren[\big]{\abs[\big]{\,a_{n+p} - a_n} < ε}.
  \end{math}
\end{definition*}

\begin{example*}
  证明数列\(\Seq{q^n}\)在\(\abs*{\,q\,} < 1\)条件下是柯西列.

  \begin{proof}
    对于任意的\(ε > 0\),存在正整数\(N\)使得:当\(n > N\)时,对于所有的正整数\(p\)都有
    \begin{equation*}
      \abs*{\,q^{n+p} - q^n}
      = \abs*{\,q^n\,} \, \abs*{\,q^p - 1}
      < 2\abs*{\,q^n\,}         % report
      < 2ε.
    \end{equation*}
    这样的\(N\)存在,是因为前面我们已经证明了\(\lim\limits_{n\to\infty} q^n = 0\),见例~\ref{eg:limqn}.
  \end{proof}
\end{example*}

\begin{example*}
  证明数列\(\Seq[\big]{\,\sum\limits_{\mathclap{k=1}}^n \frac1k\,}\)不是柯西列.

  \begin{proof}
    前面我们已经证明了此数列单调递增且为无穷大量.将此数列记为\(\Seq{a_n}\).先取某\(ε_0 > 0\).对于任意正整数\(N\),取某\(n_0 > N\).因为数列\(\Seq{a_n}\)单调递增且为无穷大量,所以存在\(m_0 > N\)使得\(a_{m_0} > a_{n_0} + ε_0 > a_{n_0}\).那么就有\(\abs[\big]{\,a_{m_0} - a_{n_0}} = a_{m_0} - a_{n_0} > ε_0\).
  \end{proof}
\end{example*}

% https://mathworld.wolfram.com/CauchyCriterion.html
% https://hsm.stackexchange.com/a/2861/13364
% https://en.wikipedia.org/wiki/Cauchy%27s_convergence_test
% https://personal.math.ubc.ca/~cass/courses/m220-00/cauchy.pdf
\begin{theorem}[柯西收敛准则]
  \label{thm:seqcvgcauchy}
  数列收敛的充分必要条件是其为柯西列.

  \begin{proof}
    易证必要性,使用三角不等式即可.下面证明充分性.易从柯西列的定义推出柯西列都有界.根据定理~\ref{thm:bw},有界列必有收敛子列,将此子列记为\(\Seq{a_{n_k}}\),其极限记为\(A\).那么对于任意的\(ε > 0\)都存在正整数\(K\)使得当\(k > K\)时都有\(\abs[\big]{\,a_{n_k} - A\,} < ε\).同时,也存在正整数\(N\)使得当\(m > N,\, n > N\)时都有\(\abs*{\,a_m - a_n} < ε\).但当\(n > N\)时又有
    \begin{equation*}
      \abs[\big]{\,a_n - A\,}
      \le \abs[\big]{\,a_n - a_{n_{k_0}}} + \abs[\big]{\,a_{n_{k_0}} - A\,}
      < 2ε,
    \end{equation*}
    其中\(n_{k_0}\)是满足\(k_0 > K\)且\(n_{k_0} > N\)的\kenten{某个数}.
  \end{proof}

  \begin{remark}
    当然,也可以直接构造区间套来证明.
  \end{remark}
\end{theorem}

\begin{example}
  \label{eg:limsumabsdiff}
  设\(b_n = \sum\limits_{k=1}^n \! \abs*{\,a_{k+1} - a_k\,} \le C\).试证明数列\(\Seq{a_n}\)收敛.

  \begin{proof}
    数列\(\Seq{b_n}\)单调递增且有界,从而收敛且为柯西列.那么对于任意的\(ε > 0\)都存在正整数\(N\)使得当\(m \ge N,\, n \ge N\)时都有
    \begin{align*}
      ε
      &> \abs[\big]{\,b_m - b_n} \\
      &= \abs[\Bigg]{\ \sum_{k=1}^m \abs*{\,a_{k+1} - a_k\,} - \sum_{k=1}^n \abs*{\,a_{k+1} - a_k\,}\ } \\
      &= \abs[\Bigg]{\ \smashoperator[r]{\sum_{k=s}^{t-1}} \abs*{\,a_{k+1} - a_k\,}\ }
        = \smashoperator{\sum_{k=s}^{t-1}} \abs*{\,a_{k+1} - a_k\,} \\
      &\ge \abs[\Bigg]{\ \smashoperator[r]{\sum_{k=s}^{t-1}} \paren*{\,a_{k+1} - a_k\,}\,}
        = \abs[\big]{\,a_t - a_s} \\
      &= \abs[\big]{\,a_{m+1} - a_{n+1}}\,,
    \end{align*}
    其中\(s = 1 + \minb{m,n},\, t = 1 + \maxb{m,n}\).当\(m > N,\, n > N\)时就有\(\abs[\big]{\,a_m - a_n} < ε\).这就说明了数列\(\Seq{a_n}\)是柯西列,从而收敛.
  \end{proof}
\end{example}

\subpdfbookmark{思考}{B1.2.3.P}
\subsection*{思考}

\begin{enumerate}
\item 在闭区间套定理中,如果将闭区间改成开区间,结论是否成立?

  \ifshowsolp
    不成立,最终可能是一个空集.设\(a_n = 0,\ b_n = 1/n\),显然有\(\paren{a_{n+1}, b_{n+1}} \subset \paren{a_n, b_n}\)且\(\lim_{n\to\infty} \paren{b_n - a_n} = 0\).但是,
    \begin{equation*}
      \bigcap_{n=1}^\infty \paren{a_n, b_n} = \bigcap_{n=1}^\infty \paren[\Big]{0, \frac1n} = \emptyset.
    \end{equation*}
  \fi

\item 如何利用数列收敛与子列收敛的关系来证明一个数列非收敛?

  \ifshowsolp
    只要找到该数列的一个不收敛子列即可.
  \fi
\end{enumerate}

\ifshowex
\currentpdfbookmark{练习}{B1.2.3.E}
\subsection*{练习}

\begin{enumerate}
\item 若数列\(\Seq{a_n}\)满足条件\(\abs*{\,a_{n+1} - a_n} \le 1/2^n\),则\uline{\makebox[6em]{}}.
  \begin{itemize}
    \renewcommand{\labelitemi}{\faCircleThin}
  \item \(\lim\limits_{n\to\infty} a_n = 0\)
  \item 数列\(\Seq{a_n}\)不一定收敛
    \ifshowsol
    \item[\faCircle]
    \else
    \item
    \fi
    \(\lim\limits_{n\to\infty} a_n = A\)且\(\abs*{\,A - a_1\,} \le 1\)
  \item \(\lim\limits_{n\to\infty} a_n = A\)且\(\abs*{\,A - a_1} > 1\)
  \end{itemize}

  \ifshowsol
    选项~A:可以由反例\(a_n = 1\)来说明.选项~B:可以从三角不等式推出此数列为柯西列,从而收敛;也可以直接使用例~\ref{eg:limsumabsdiff}的结论.选项~C可以先推出\(\abs*{\,a_n - a_1} \le 1 - 1/2^{n-1}\),再取极限即可.
  \fi

\item 设\(\Seq*{\paren{\,a_n, b_n}}\)是一个开区间序列且满足:\enumparen{1}\(a_1 < a_2 < \dots < a_n < \dots < b_n < \dots < b_2 < b_1\);\enumparen{2}\(\lim\limits_{n\to\infty} \paren{\,b_n - a_n} = 0\).那么就有\uline{\makebox[10em]{}}.
  \begin{itemize}
    \renewcommand{\labelitemi}{\faCircleThin}
    \ifshowsol
    \item[\faCircle]
    \else
    \item
    \fi
    存在唯一的实数\(ξ\)属于所有开区间\(\paren{\,a_n, b_n}\)且\(ξ = \lim\limits_{n\to\infty} a_n\)
  \item 存在唯一的实数\(ξ\)属于所有开区间\(\paren{\,a_n, b_n}\)且\(ξ \ne \lim\limits_{n\to\infty} a_n\)
  \item 至少存在两个不同的实数\(ξ\)和\(η\)属于所有开区间\(\paren{\,a_n, b_n}\)
  \item 不存在实数\(ξ\)属于所有开区间\(\paren{\,a_n, b_n}\)
  \end{itemize}

  \ifshowsol
    可以参考区间套定理的证明方式.
  \fi

\item 关于数列,下列四个结论中,正确的有\uline{\makebox[6em]{%
      \ifshowsol
        \enumparen{1}%
        \enumparen{2}
      \fi}}.
  \begin{enumerate}
    \renewcommand{\labelenumii}{\enumparen{\arabic{enumii}}}
  \item 单调的无界数列一定为无穷大量
  \item 无界数列存在无穷大量子列
  \item 数列收敛等价于有无穷多个子列收敛
  \item 发散数列存在无穷大量子列
  \end{enumerate}

\item 下面四个数列中,是柯西列的有\uline{\makebox[6em]{%
      \ifshowsol
        \enumparen{2}%
        \enumparen{3}
      \fi}}.
  \begin{enumerate}
    \renewcommand{\labelenumii}{\enumparen{\arabic{enumii}}}
  \item \(\Seq[\big]{1 + \frac12 + \dots + \frac1n}\)
  \item \(\Seq[\big]{1 + \frac1{2^2} + \dots + \frac1{n^2}}\)
  \item \(\Seq[\big]{\frac{n!}{n^n}}\)
  \item \(\Seq[\big]{\sin\frac{n^n}{4}}\) % report
  \end{enumerate}

\item 关于区间\(\paren{0,1}\)中所有的有理点排成的点列\(\Seq[\big]{\frac12, \frac13, \frac23, \frac14, \frac34, \frac15, \frac25, \frac35, \frac45, \dots}\),下列四个结论中,正确的有\uline{\makebox[6em]{%
      \ifshowsol
        \enumparen{1}
      \fi}}.
  \begin{enumerate}
    \renewcommand{\labelenumii}{\enumparen{\arabic{enumii}}}
  \item 对于任意的\(x \in [0,1]\)都存在该点列的一个子列收敛于\(x\)
  \item 不存在\(x \in [0,1]\)使得该点列的一个子列收敛于\(x\)
  \item 仅存在有限个\(x \in [0,1]\)使得该点列的一个子列收敛于\(x\)
  \item 至少存在有限个\(x \in [0,1]\)使得该点列的任何一个子列都不收敛于\(x\)
  \end{enumerate}

\item 下列说法,错误的是\uline{\makebox[6em]{}}.
  \begin{itemize}
    \renewcommand{\labelitemi}{\faCircleThin}
  \item 数列\(\Seq{a_n}\)单调,则\(\lim\limits_{n\to\infty} a_n = A\)的充要条件是存在子列\(\Seq{a_{n_k}}\)满足\(\lim\limits_{n\to\infty} a_{n_k} = A\)
    \ifshowsol
    \item[\faCircle]
    \else
    \item
    \fi
    若数列\(\Seq{a_n}\)不收敛,则必存在两个子列\(\Seq{a_{n_k}^{(1)}}\)和\(\Seq{a_{n_k}^{(2)}}\)分别收敛于两个不同的值
  \item 若数列\(\Seq{a_n}\)无界但非无穷大量,则必存在一个无穷大量子列和一个收敛子列
  \item 设\(S\)为非空有上界的实数集.若\(\sup S = A \notin S\),则存在单调递增数列\(\Seq{a_n} \subset S\)使得\(\lim\limits_{n\to\infty} a_n = A\).
  \end{itemize}
\end{enumerate}
\fi

\section{函数极限的概念与性质}

\begin{definition*}
  \label{defn:limfunc}
  设函数\(\,f\mkern2mu\)在点\(x_0\)附近\footnote{在点\(x_0\)附近是指在此点的某个去心邻域上,也就是说在集合\(\paren{x_0 - δ, x_0 + δ} \setminus \Set{x_0}\)上,其中\(δ\)是某个正数.}有定义,\(A\)是某个常数.函数\(\,f\mkern2mu\)在\(x\)趋向于\(x_0\)时的极限是\(A\),记作
  \begin{equation*}
    \lim_{\,x \to x_0\!} \,f(x) = A,
  \end{equation*}
  当且仅当:对于任意的\(ε > 0\),存在\(δ > 0\)使得当\(0 < \abs*{\,x-x_0} < δ\)时都有\(\abs*{\mkern6mu f(x) - A\,} < ε\).
\end{definition*}

\begin{example*}
  证明\(\lim\limits_{x\to1} x^2 = 1\).

  \begin{proof}
    对于任意的\(ε > 0\),取\(δ = \minb{\,1, ε/3\,}\),当\(0 < \abs{\ x - 1\,} < δ\)时就有
    \begin{equation*}
      \abs[\big]{\mkern6mu f(x) - 1\,}
      = \abs[\big]{\ x^2 - 1\,}
      = \abs[\big]{\ x+1\,} \, \abs[\big]{\ x-1\,}
      < \abs[\big]{\,x+1\,} \, δ
      < 3δ
      \le ε.
      \qedhere
    \end{equation*}
  \end{proof}
\end{example*}

\begin{example*}
  证明\(\lim\limits_{x\to0} \cos x = 1\).

  \begin{proof}
    对于任意的\(ε > 0\),若要\(\abs*{\,\cos x - 1\,} < ε\)成立,只需\(1 - ε < \cos x\),即\(1 - \cos x < ε\).因为
    \begin{equation*}
      1 - \cos x = 2 \sin^2 \frac{x}{2} \le \frac{x^2}{2},
    \end{equation*}
    所以取\(δ = \sqrt{2\,ε}\)就能使得当\(0 < \abs{\,x\,} < δ\)时都有\(1 - \cos x < ε\).
  \end{proof}
\end{example*}

\begin{definition*}[单侧极限]
  设函数\(\,f\mkern2mu\)在点\(x_0\)左侧有定义,\(A\)是某个常数.函数\(\,f\mkern2mu\)在\(x\)从左侧趋向于\(x_0\)时的极限是\(A\),当且仅当:对于任意的\(ε > 0\)都存在\(δ > 0\)使得当\(0 < x_0 - x < δ\)时都有\(\abs*{\mkern6mu f(x) - A\,} < ε\).这样的极限叫作左侧极限,简称左极限,记作
  \begin{equation*}
    \lim_{x \to x_0^-\!} \,f(x) = A,
    \quad
    f_-(x_0) = A
    \txt{或}
    f(x_0-0) = A.
  \end{equation*}

  设函数\(\,f\mkern2mu\)在点\(x_0\)右侧有定义,\(A\)是某个常数.函数\(\,f\mkern2mu\)在\(x\)从右侧趋向于\(x_0\)时的极限是\(A\),当且仅当:对于任意的\(ε > 0\)都存在\(δ > 0\)使得当\(0 < x - x_0 < δ\)时都有\(\abs*{\mkern6mu f(x) - A\,} < ε\).这样的极限叫作右侧极限,简称右极限,记作
  \begin{equation*}
    \lim_{x \to x_0^+\!} \,f(x) = A,
    \quad
    f_+(x_0) = A
    \txt{或}
    f(x_0+0) = A.
  \end{equation*}
\end{definition*}

\begin{theorem}
  \label{thm:limfuncsided}
  函数极限存在的充分必要条件是它的左极限和右极限都存在且相等.

  \begin{proof}
    易证必要性.关于充分性,只需取\(δ = \minb{\,δ_1, δ_2}\)即可,其中\(δ_1\)和\(δ_2\)分别是左右极限所取的德尔塔.
  \end{proof}
\end{theorem}

\begin{example*}
  函数
  \begin{equation*}
    f(x) =
    \begin{cases}
      x+1, & x \ge 0, \\
      x-1, & x < 0
    \end{cases}
  \end{equation*}
  在\(x = 0\)处的极限不存在.
\end{example*}

\begin{example*}
  函数\(\,f(x) = \arctan(1/x)\)在\(x = 0\)处的极限不存在. % \rule[-2ex]{0ex}{5.5ex}
\end{example*}

\begin{example*}
  函数\(\,f(x) = {x}/\paren{\abs{\,x\,} + 1}\)在无穷处的极限不存在. % \rule[-2ex]{0ex}{0ex}
\end{example*}

\begin{theorem*}[唯一性]
  函数的极限若存在则唯一.

  \begin{remark}
    参考定理~\ref{thm:cvguniq}的证明即可.
  \end{remark}
\end{theorem*}

\begin{theorem*}[有界性]
  若函数\(\,f\mkern2mu\)在\(x \to x_0\!\)时的极限存在,则其在\(x_0\)附近有界.

  \begin{remark}
    参考定理~\ref{thm:cvgbnd}的证明即可.
  \end{remark}
\end{theorem*}

\begin{theorem}[保号性]
  \label{thm:limfuncsgnprsv}
  若函数\(\,f\mkern2mu\)在\(x \to x_0\!\)时收敛于某个非零数\(A\),则其在\(x_0\)附近拥有与\(A\)相同的正负号.

  \begin{remark}
    参考定理~\ref{thm:cvgsgnprsv}的证明即可.
  \end{remark}
\end{theorem}

% https://gallica.bnf.fr/ark:/12148/bpt6k3116z/f688.item
% https://zbmath.org/46.0295.04
\begin{theorem}
  \label{thm:limequiv}
  函数\(\,f\mkern2mu\)在\(x \to x_0\!\)时收敛于\(A\)的充分必要条件\footnote{这个条件又叫作极限的Heine定义.Sierpiński在1916年发现证明这两个定义等价,其实暗含了弱形式的选择公理.}是:对于任意的非\(x_0\)数列\(\Seq{x_n}\),当它收敛于\(x_0\)时,就一定有数列\(\Seq{\,f(x_n)}\)收敛于\(A\).

  \begin{proof}
    易证必要性.关于充分性,可以通过反证法证明.假设函数\(\,f\mkern2mu\)在\(x \to x_0\!\)时不收敛于\(A\),则能构造一个数列\(\Seq{x_n}\)使得它收敛于\(x_0\)但是数列\(\Seq{\,f(x_n)}\)不收敛于\(A\).
  \end{proof}

  \begin{remark}
    实际上,对于单侧极限,也有类似的结论.只需把上面的“非\(x_0\)数列\(\Seq{x_n}\)”改成“小于\(x_0\)的数列\(\Seq{x_n}\)”(左极限)和“大于\(x_0\)的数列\(\Seq{x_n}\)”(右极限)即可.
  \end{remark}
\end{theorem}

\begin{example*}
  函数\(\,f\mkern2mu\)在\(\brktparen{\,a, +\infty}\)上有定义,在\(\brkt{\,a, A\,}\)上有界,其中\(A\)是任意一个大于等于\(a\)的数,并且在趋向于正无穷时极限存在.证明函数\(\,f\mkern2mu\)在\(\brktparen{\,a, +\infty}\)上有界.

  \begin{proof}
    记\(B\)为函数在趋向正无穷时的极限.那么对于任意的\(ε > 0\)都存在\(δ > 0\)使得当\(x > δ\)时都有\(\abs*{\mkern6mu f(x) - B\,} < ε\),所以\(\abs[\big]{\mkern6mu f(x)} < \abs[\big]{\,B\,} + ε\).因此,函数\(\,f\mkern2mu\)在\(\paren{δ, +\infty}\)上有界.又因为函数\(\,f\mkern2mu\)在\(\brkt{\,a, δ\,}\)上也有界,所以它在\(\brktparen{\,a, +\infty}\)上有界.
  \end{proof}
\end{example*}

\begin{example*}
  试判断极限\(\,\lim\limits_{x\to0} \cos(1/x)\)存在与否.

  \begin{remark}
    不存在.下面给出两种证明.可以发现,对于这种否定性的命题,使用极限的Heine定义要方便得多.
  \end{remark}

  \begin{proof}
    对于任意的\(A \in \R\),都存在\(ε_0 > 0\)使得当\(δ > 0\)时都能找到一个\(x_0\)满足\(0 < \abs{\,x_0\,} < δ\)且\(\abs[\big]{\,\cos(1/x_0) - A\,} \ge ε_0\).当\(A \notin \brkt{-1, 1}\)时,只需取\(ε_0 = \abs{\,A\,} - 1\)即可.当\(A \in \paren{-1, 1}\)时,取\(ε_0 = \paren{1-\abs{\,A\,}}/{2}\)和\(x_0 = {1}/\brkt{2\pi\paren{\ceil{1/2\piδ}+1}}\)即可.当\(A \in \Set{-1, 1}\)时,取\(ε_0 = 1/2\)和\(x_0 = {1}/\paren{\pi/2 + 2\pi\ceil{1/2\piδ}}\)即可.
  \end{proof}

  \begin{proof}
    令\(x_n = 1/\paren{2n\pi},\ \bar x_n = 1/\paren{\pi/2 + 2n\pi}\).显然有非零数列\(\Seq{x_n}\)和数列\(\Seq{\bar x_n}\)都收敛于\(0\),但数列\(\Seq{\,f(x_n)}\)和\(\Seq{\,f(\bar x_n)}\)收敛于不同的值.
  \end{proof}
\end{example*}

% Bartle & Sherbert, 4th ed. (128)
% Baby Rudin, 3rd ed. (100)
% Cooke's Solutions to Baby Rudin (57)
% Dunham (149)
% Abbott, 2nd ed. (114)
% http://resolver.sub.uni-goettingen.de/purl?PPN575586591
% https://mdz-nbn-resolving.de/details:bsb11309553
\begin{example*}[Thomae函数]
  设\(x_0\)是一个实数.试证明函数\footnote{此函数由Johannes Thomae于1875年在讨论可积性时提出.}
  \begin{equation*}
    f(x) =
    \begin{cases}
      1/q, & \text{\(x = p/q\),其中\(p \in \Z\)与\(q \in \N^+ \)互质,} \\
      1, & x = 0, \\
      0, & \otherwise
    \end{cases}
  \end{equation*}
  在\(x \to x_0\!\)时收敛于\(0\).

  \begin{remark}
    下面给出两种证明.
  \end{remark}

  \begin{proof}
    我们可以将此函数看成直尺上的刻度.可以先考察的此函数的一些性质,易知\(1\)和\(0\)分别是它的上下界,它是一个周期为\(1\)的函数.那么只需证明它在区间\(\paren{0,1}\)上的极限、在点\(0\)处的右极限、在点\(1\)处的左极限都是\(0\)即可.对于任意的\(ε > 0\),大于\(ε\)的刻度精度集合
    \begin{equation*}
      Q_ε^* = \Set[\big]{\, 1/q \,\mid\, q \in \N^+ \tand 1/q \ge ε \,}
    \end{equation*}
    显然是有限的.又因为由刻度精度\(1/p\)派生出来的刻度集合
    \begin{equation*}
      P_{1/q} = \Set[\big]{\, p/q \,\mid\, p \in \N^+ \tand p < q \,}
    \end{equation*}
    也是有限的,那么刻度集合
    \begin{equation*}
      P_{Q_ε^*}
      = {}\smashoperator{\bigcup_{1/q \in Q_ε^*}} \, P_{1/q}
      = \Set[\big]{\, p/q \,\mid\, p,q \in \N^+,\ 1/q \ge ε,\  p < q\,}
    \end{equation*}
    也是有限的.当\(x_0 \in \paren{0,1}\)时,令\(X = \paren[\big]{P_{Q_ε^*} \!\setminus \Set{x_0}} \cup \Set{0,1}\),取
    \begin{equation*}
      δ = \min\Set[\big]{\, \abs{\ x - x_0\,} \,\mid\, x \in X\,}
    \end{equation*}
    即可使当\(0 < \abs{\ x - x_0\,} < δ\)时都有\(\,f(x) < ε\).同理,可证函数\(\,f\mkern2mu\)在点\(0\)处的右极限和在点\(1\)处的左极限也是\(0\),从而它在\(\R\)上的所有极限都是\(0\).
  \end{proof}
  % TODO: Tidy up the proof
  \begin{proof}
    对于任意的收敛于\(x_0\)的非\(x_0\)数列\(\Seq{x_n}\),它一定能分割成有理数和无理数两个部分,这两个部分不可能都是有限的,也就是说这两个部分当中,至少有一个部分是无限的.若无理数的那个部分是无限的,将这个部分构成的子列记为\(\Seq{x_{n_k}}\).那么数列\(\Seq{\,f(x_{n_k})}\)自然收敛于\(0\).若有理数的那个部分是无限的,将这个部分构成的子列记为\(\Seq{\bar x_{n_k}}\).只要说明数列\(\Seq{\,f(\bar x_{n_k})}\)也收敛于\(0\),就能说明数列\(\Seq{\,f(x_n)}\)也收敛于\(0\),从而函数在\(x \to x_0\!\)时也收敛于\(0\).

    接下来,我们来证明数列\(\Seq{\,f(\bar x_{n_k})}\)确实收敛于\(0\).只需证明,对于任意的\(ε > 0\),只存在有限个项\(\bar x_{n_k}\)满足\(\,f(\bar x_{n_k}) \ge ε\)即可.因为取\(K = \max\Set[\big]{\, k \,\mid\, f(\bar x_{n_k}) \ge ε\,}\),就能使得当\(k > K\)时都有\(\,f(\bar x_{n_k}) < ε\).因为\(\bar x_{n_k}\)是有理数,必然能写成\(\bar x_{n_k} = p_k/q_k\)的最简分数形式.令
    \begin{equation*}
      P = \Set[\big]{\, p_k \mid \,f(p_k/q_k) \ge ε\,}, \quad
      Q = \Set[\big]{\, q_k \mid \,f(p_k/q_k) \ge ε\,}, \quad
      R = \Set[\big]{\, p_k/q_k \mid \,f(p_k/q_k) \ge ε\,}.
    \end{equation*}
    由于实数的阿基米德性质,易知有\(\card*{\,Q\,} \in \N\)和\(\card*{\,P\,} \in \N\),从而\(\card*{\,R\,} \in \N\).若有无穷多项\(\bar x_{n_k}\)满足\(\,f(\bar x_{n_k}) \ge ε\),则对于任意的正整数\(K\)都存在一个\(k > K\)满足
    \begin{equation*}
      \abs[\big]{\ \bar x_{n_k} - x_0\,} \ge \min\Set[\big]{\,\abs{\ r - x_0\,} \,\mid\, r \in R \,} > 0.
    \end{equation*}
    这与子列\(\Seq{\bar x_{n_k}}\)收敛于\(x_0\)是矛盾的.因此只存在有限个项\(\bar x_{n_k}\)满足\(\,f(\bar x_{n_k}) \ge ε\).
  \end{proof}
\end{example*}

\subpdfbookmark{思考}{B1.2.4.P}
\subsection*{思考}

\begin{enumerate}
\item 用\(ε\)--\(N\)语言给出\(\lim_{x\to\infty} f(x) = A\)的定量描述.

  \ifshowsolp
    用形式语言的来定义,就是
    \begin{equation*}
      \paren[\big]{\forall ε > 0}
      \paren[\big]{\exists N > 0}
      \paren[\big]{\forall \abs{\,x\,} > N}
      \paren[\big]{\abs[\big]{\mkern6mu f(x) - A\,} < ε}.
    \end{equation*}
  \fi

  % https://math.stackexchange.com/q/419909/147999
\item 思考函数极限与数列极限之间的关系,并尝试通过数列极限给出各种函数极限存在的柯西收敛原理.

  \ifshowsolp
    设函数\(\,f\mkern2mu\)在点\(a\)附近有定义.我们称\(\,f\)在点\(a\)处柯西收敛,当且仅当:
    \begin{equation*}
      \paren[\big]{\forall ε > 0}
      \paren[\big]{\exists δ > 0}
      \paren[\big]{\forall \abs[\big]{x - a} < δ}
      \paren[\big]{\forall \abs[\big]{y - a} < δ}
      \paren[\big]{\abs[\big]{\,f(x) - f(y)} < ε}.
    \end{equation*}

    函数的柯西收敛原理无非是说:函数\(\,f\mkern2mu\)在点\(a\)处极限存在的充分必要条件是它在点\(a\)处柯西收敛.

    \begin{proof}
      易证必要性.下面来证一下充分性.对于任意的\(ε > 0\)都存在\(δ > 0\)使得当\(\abs{\,x - a\,} < δ,\ \abs{\,y - a\,} < δ\)时都有\(\abs*{\mkern6mu f(x) - f(y)\,} < ε\).这时,对于任何收敛于\(a\)的非\(a\)数列\(\Seq{x_n}\),自然存在一个正整数\(N\)使得:当\(m\)和\(n\)大于\(N\)时,都有\(\abs*{\,x_m - a\,} < δ\)和\(\abs*{\,x_n - a\,} < δ\).所以有\(\abs*{\mkern6mu f(x_m) - f(x_n)\,} < ε\).这就说明数列\(\Seq{\,f(x_n)}\)是柯西列,从而收敛.与不同的数列\(\Seq{x_n}\)所对应的数列\(\Seq{\,f(x_n)}\)必然收敛于同一个数.因为假设收敛于不同的两个数,将不同的两个数列可以合并成一个新的数列\(\Seq{\bar x_n}\),从而推出\(\Seq{\,f(\bar x_n)}\)发散.这和前面的结论时矛盾的.因此,函数\(\,f\mkern2mu\)在点\(a\)处的极限存在.
    \end{proof}
  \fi
\end{enumerate}

\ifshowex
\currentpdfbookmark{练习}{B1.2.4.E}
\subsection*{练习}

\begin{enumerate}
\item 若\(\lim\limits_{x\to0} \,f(x) = 1\),则必定\uline{\makebox[6em]{}}.
  \begin{itemize}
    \renewcommand{\labelitemi}{\faCircleThin}
  \item \(\,f(0) = 1\)
  \item 函数\(\,f\mkern2mu\)在原点没定义.
    \ifshowsol
    \item[\faCircle]
    \else
    \item
    \fi
    函数\(\,f\mkern2mu\)在原点附近大于\(0\)
  \item 函数\(\,f\mkern2mu\)在原点附近不等于\(1\)
  \end{itemize}

\item 函数\(\,f(x) = \abs{\,x\,} / x\)在原点处\uline{\makebox[6em]{}}.
  \begin{itemize}
    \renewcommand{\labelitemi}{\faCircleThin}
  \item 极限存在且为\(1\)
  \item 极限存在但不为\(1\)
    \ifshowsol
    \item[\faCircle]
    \else
    \item
    \fi
    极限不存在但在该点附近有界
  \item 极限不存在且在该点附近无界
  \end{itemize}

\item 函数
  \begin{equation*}
    f(x) =
    \begin{cases}
      \sin\frac1x, & x > 0, \\
      x \sin\frac1x, & x < 0
    \end{cases}
  \end{equation*}
  在原点处\uline{\makebox[6em]{}}.
  \begin{itemize}
    \renewcommand{\labelitemi}{\faCircleThin}
  \item 左右极限均存在且都为\(0\)
  \item 左右极限均不存在
    \ifshowsol
    \item[\faCircle]
    \else
    \item
    \fi
    左极限存在,但右极限不存在
  \item 左右极限都存在但不相同
  \end{itemize}

\item 函数
  \begin{equation*}
    f(x) =
    \begin{cases}
      2x, & x > 0, \\
      a \cos x + b \sin x, & x < 0
    \end{cases}
  \end{equation*}
  在原点处\uline{\makebox[6em]{}}.
  \begin{itemize}
    \renewcommand{\labelitemi}{\faCircleThin}
  \item 极限存在
  \item 极限不存在
  \item 当且仅当\(a = 0,\ b = 0\)时极限存在
    \ifshowsol
    \item[\faCircle]
    \else
    \item
    \fi
    当且仅当\(a = 0\)时极限存在
  \end{itemize}

\item 关于函数极限,下列结论中,正确的有\uline{\makebox[6em]{%
      \ifshowsol
        \enumparen{2}%
        \enumparen{4}
      \fi}}.
  \begin{enumerate}
    \renewcommand{\labelenumii}{\enumparen{\arabic{enumii}}}
  \item 若\(\lim\limits_{x\to0} \,f(x^2) = A\),则\(\lim_{x\to0} \,f(x) = A\)
  \item 若\(\lim\limits_{x\to0} \,f(x^3) = A\),则\(\lim_{x\to0} \,f(x) = A\)
  \item 若函数\(\,f\mkern2mu\)是周期函数且\(\lim\limits_{x\to0} \,f(x) = A\),则\(\,f(x) \equiv A\)
  \item 若函数\(\,f\mkern2mu\)是周期函数且\(\lim\limits_{x\to\infty} \,f(x) = A\),则\(\,f(x) \equiv A\)
  \end{enumerate}

\item 狄利克雷函数(定义~\ref{defn:dirichlet})\uline{\makebox[6em]{}}.
  \begin{itemize}
    \renewcommand{\labelitemi}{\faCircleThin}
  \item 在任意点处的极限都存在
    \ifshowsol
    \item[\faCircle]
    \else
    \item
    \fi
    在任意点处的极限都不存在
  \item 仅在有理点处的极限存在
  \item 仅在无理点处的极限存在
  \end{itemize}

\item 下列说法中,正确的是\uline{\makebox[10em]{}}.
  \begin{itemize}
    \renewcommand{\labelitemi}{\faCircleThin}
  \item
    \begin{math}
      \paren[\big]{\exists ε > 0}
      \paren[\big]{\forall δ > 0}
      \paren[\big]{\forall 0 < \abs{\,x\,} < δ}
      \paren[\big]{\abs[\big]{\mkern6mu f(x) - A\,} < ε}
      \implies
      \lim\limits_{x\to0} \,f(x) = A
    \end{math}
  \item
    \begin{math}
      \paren[\big]{\exists ε > 0}
      \paren[\big]{\exists δ > 0}
      \paren[\big]{\forall 0 < \abs{\,x\,} < δ}
      \paren[\big]{\abs[\big]{\mkern6mu f(x) - A\,} < ε}
      \implies
      \lim\limits_{x\to0} \,f(x) = A
    \end{math}
  \item
    \begin{math}
      \lim\limits_{x\to0} \,f(x) = A
      \implies
      \paren[\big]{\forall ε > 0}
      \paren[\big]{\forall δ > 0}
      \paren[\big]{\forall 0 < \abs{\,x\,} < δ}
      \paren[\big]{\abs[\big]{\mkern6mu f(x) - A\,} < ε}
    \end{math}
    \ifshowsol
    \item[\faCircle]
    \else
    \item
    \fi
    \begin{math}
      \paren[\big]{\forall ε > 0}
      \paren[\big]{\forall δ > 0}
      \paren[\big]{\forall 0 < \abs{\,x\,} < δ}
      \paren[\big]{\abs[\big]{\mkern6mu f(x) - A\,} < ε}
      \implies
      \lim\limits_{x\to0} \,f(x) = A
    \end{math}
  \end{itemize}
\end{enumerate}
\fi

\section{函数极限的运算}

% TODO: Expatiate on tending to infinity
\begin{theorem}[函数极限的四则运算]
  \label{thm:limfunc4ops}
  若\(\!\lim\limits_{\,x \to x_0\!} \,f(x) = A, \lim\limits_{\,x \to x_0\!} g(x) = B\),则
  \begin{enumerate}
    \renewcommand{\labelenumi}{\enumparen{\arabic{enumi}}}
  \item \(\lim\limits_{\,x \to x_0\!} \paren*{\,f(x) \pm g(x)} = A \pm B\),
  \item \(\lim\limits_{\,x \to x_0\!} \,f(x)\,g(x) = AB\),
  \item \(\lim\limits_{\,x \to x_0\!} {\,f(x)}/{g(x)} = A/B\ (B \ne 0)\).
  \end{enumerate}

  \begin{proof}
    使用定理~\ref{thm:seq4ops}和定理~\ref{thm:limequiv}即可得证.
  \end{proof}
\end{theorem}

\begin{example*}
  求\(\!\lim\limits_{\,x\to+\infty\!} \! \paren{5a^x - 3b^x}/\paren{5a^x + 3b^x}\ (a > 0,\ b > 0)\).\rule{0ex}{4ex}

  \begin{remark}
    当\(a = b\)时,有
    \begin{equation*}
      \lim_{x\to+\infty} \! \frac{5a^x - 3b^x}{5a^x + 3b^x} = \lim_{x\to+\infty} \! \frac{5a^x - 3a^x}{5a^x + 3a^x} = \frac14.
    \end{equation*}
    当\(a > b\)时,有
    \begin{align*}
      \lim_{x\to+\infty} \! \frac{5a^x - 3b^x}{5a^x + 3b^x}
      &= \! \lim_{x\to+\infty} \! \frac{5 - 3(b^x\!/a^x)}{5 + 3(b^x\!/a^x)}
        = \! \lim_{x\to+\infty} \! \frac{5 - 3(b/a)^x}{5 + 3(b/a)^x}
      && \reason{变形} \\
      &= \frac{\! \lim\limits_{x\to+\infty} \! \paren*{5 - 3(b/a)^x}}{\! \lim\limits_{x\to+\infty} \! \paren*{5 + 3(b/a)^x}}
      && \reason{除法法则} \\
      &= \frac{\! \lim\limits_{x\to+\infty} \! 5 - \, \lim\limits_{\mathclap{x\to+\infty}} \ 3(b/a)^x}{\! \lim\limits_{x\to+\infty} \! 5 + \, \lim\limits_{\mathclap{x\to+\infty}} \ 3(b/a)^x}
      && \reason{加减法法则} \\
      &= \frac{5 - 3 \lim\limits_{\mathclap{x\to+\infty}} \ (b/a)^x}{5 + 3 \lim\limits_{\mathclap{x\to+\infty}} \ (b/a)^x}
      && \reason{乘法法则} \\
      &= 1.
    \end{align*}
    同理可知,当\(a < b\)时,有\(\!\lim\limits_{x\to+\infty} \! \paren{5a^x - 3b^x}/\paren{5a^x + 3b^x} = -1\).所以,
    \begin{equation*}
      \lim_{x\to+\infty} \! \frac{5a^x - 3b^x}{5a^x + 3b^x} =
      \begin{cases}
        1/4, & a = b, \\
        1, & a > b, \\
        -1, & a < b.
      \end{cases}
    \end{equation*}
  \end{remark}
\end{example*}

\begin{example*}
  求\(\lim\limits_{x\to2} \, \paren[\big]{x^2 + x - 6}\big/\paren[\big]{\sqrt{x\mathstrut} - \sqrt{4-x\mathstrut}}\).

  \begin{remark}
    分母有理化后,得到
    \begin{equation*}
      \lim_{x\to2} \frac{x^2 + x - 6}{\sqrt{x\mathstrut} - \sqrt{4-x\mathstrut}}
      = \lim_{x\to2} \frac{%
        \paren[\big]{x^2 + x - 6}
        \paren[\big]{\sqrt{x\mathstrut} + \sqrt{4-x\mathstrut}}}{%
        x - 4 + x}
      = \lim_{x\to2} \frac{%
        \paren[\big]{x-2}
        \paren[\big]{x+3}
        \paren[\big]{\sqrt{x\mathstrut} + \sqrt{4-x\mathstrut}}}{%
        2(x-2)}
      = 5\sqrt2\,.
    \end{equation*}
  \end{remark}
\end{example*}

\begin{theorem}[复合函数的极限]
  \label{thm:limfunccomp}
  若函数\(\,f\mkern2mu\)在点\(B\)处的极限是\(A\),函数\(\mkern1mu g\)在点\(x_0\)处的极限是\(B\)且在点\(x_0\)附近不等于\(B\),则复合函数\(\,f \circ g\)在点\(x_0\)的极限是\(A\).

  \begin{proof}
    使用定理~\ref{thm:limequiv}即可得证.
  \end{proof}
\end{theorem}

\begin{example*}
  令\(\,f(u) = \abs{\,\sgn x\,}\,,\ g(x) = x \Fn D(x)\).复合函数\(\,f \circ g\)在原点处的极限存在吗?

  \begin{remark}
    不存在.若“生吞活剥”地套用上述定理,就可能得出极限存在且为\(1\)的结论.实际上,
    \begin{equation*}
      (\,f \circ g)(x)
      = \abs[\Big]{\,\sgn\paren[\big]{x \Fn D(x)}} =
      \begin{cases}
        \Fn D(x), & x \ne 0, \\
        0, & x = 0.
      \end{cases}
    \end{equation*}
    函数的极限在某点处的极限和它在这点的取值无关,狄利克雷函数在实轴上处处极限不存在,因此该复合函数在原点自然也不存在.误用上述定理的原因是:由于无理数是稠密的,函数\(\mkern1mu g\)在原点附近总是能取到\(0\).
  \end{remark}
\end{example*}

\begin{theorem}
  \label{thm:limfuncpowexp}
  在同一个极限过程中,若函数\(\,f\mkern2mu\)的极限是正数\(A\)且函数\(\mkern1mu g\)的极限是\(B\),则函数\(\,f^g\!\)的极限是\(A^B\).

  \begin{proof}
    运用两次定理~\ref{thm:limfunccomp}和一次定理~\ref{thm:limfunc4ops},有
    \begin{align*}
      \lim_{\,x \to x_0\!} \,f(x)^{g(x)}
      &= \smashoperator{\lim_{\,x \to x_0\!}} \expb[\Big]{\ln \,f(x)^{g(x)}}
        = \smashoperator{\lim_{\,x \to x_0\!}} \expb[\Big]{g(x) \ln \,f(x)}
        = \expb[\Big]{\lim_{\,x \to x_0\!} g(x) \ln \,f(x)} \\
      &= \expb[\Big]{\lim_{\,x \to x_0\!} g(x) \smashoperator[r]{\lim_{\,x \to x_0\!}} \ln \,f(x)}
        = \expb[\Big]{B \ln \smashoperator{\lim_{\,x \to x_0\!}} \,f(x)} \\
      &= \expb[\big]{B \ln A} = \expb[\big]{\ln A^B} = A^B.
    \end{align*}
    注意,在使用定理~\ref{thm:limfunccomp}的时候,用到了指数函数和对数函数都是连续函数(见\hyperref[ch:cont]{下章})的事实.
  \end{proof}
\end{theorem}

\begin{theorem}[函数极限的夹逼定理]
  \label{thm:funcsqueeze}
  若函数\(\,f,\ g,\ h\)满足:
  \begin{enumerate}[topsep=0ex,itemsep=0ex]
    \renewcommand{\labelenumi}{\enumparen{\arabic{enumi}}}
  \item 在点\(x_0\)附近有\(\,f \le g \le h\),
  \item \(\lim\limits_{\,x \to x_0\!} \,f(x) = \lim\limits_{\,x \to x_0\!} h(x) = A\),
  \end{enumerate}
  则\(\!\lim\limits_{\,x \to x_0\!} g(x) = A\).

  \begin{proof}
    使用定理~\ref{thm:squeeze}和定理~\ref{thm:limequiv}即可得证.
  \end{proof}
\end{theorem}

\begin{example*}
  求\(\lim\limits_{x\to0} x \, \floor{1/x}\).

  \begin{remark}
    实际上,当\(x > -1\)时,有
    \begin{equation*}
      \Fn H(-x) + \frac{\floor{1/x}}{\floor{1/x}+1} \Fn H(x)
      \le
      x \, \floor[\bigg]{\,\frac1x\,}
      \le
      \frac{\floor{1/x}}{\floor{1/x}+1} \Fn H(-x) + \Fn H(x),
    \end{equation*}
    其中\(\Fn H\)为单位阶跃函数(定义~\ref{defn:heaviside}).所以,\(\lim\limits_{x\to0} x \, \floor{1/x} = 1\).
  \end{remark}
\end{example*}

\begin{theorem*}
  \(\lim\limits_{x\to0} \,\paren{\sin x}/{x} = 1\).

  \begin{proof}
    根据三角函数在几何上的意义,当\(0 < x < \pi/2\)时,有
    \begin{equation*}
      \sin x < x < \tan x
      \iff
      1 < \frac{x}{\sin x} < \sec x
      \iff
      \cos x < \frac{\sin x}{x} < 1.
    \end{equation*}
    使用一次夹逼定理,得
    \begin{equation*}
      \lim_{x\to0^+\!} \! \frac{\sin x}{x} = 1,
    \end{equation*}
    又因为这个函数是偶函数,所以\(\lim\limits_{x\to0} \,\paren{\sin x}/{x} = 1\).
  \end{proof}
\end{theorem*}

\begin{example*}
  常见的极限结论:
  \begin{equation*}
    \lim_{x\to0} \frac{\tan x}{x} = 1,
    \quad
    \lim_{x\to0} \frac{1 - \cos x}{x^2} = \frac12,
    \quad
    \lim_{x\to0} \frac{\arcsin x}{x} = 1,
    \quad
    \lim_{x\to0} \frac{\arctan x}{x} = 1.
  \end{equation*}
\end{example*}

\begin{example*}
  求\(\lim\limits_{x\to0} \,\paren{\sin ax}/\paren{\sin bx}\ (b \ne 0)\).

  \begin{remark}
    当\(a \ne 0\)时,有
    \begin{equation*}
      \lim_{x\to0} \frac{\sin ax}{\sin bx}
      = \lim_{x\to0} \frac{\sin ax}{ax} \cdot \frac{ax}{bx} \cdot \frac{bx}{\sin bx}
      = \frac ab.
    \end{equation*}
    当\(a = 0\)时,有
    \begin{equation*}
      \lim\limits_{x\to0} \frac{\sin ax}{\sin bx} = 0 = \frac ab.
    \end{equation*}
    所以,无论\(a\)的取值,都有\(\lim\limits_{x\to0} \,\paren{\sin ax}/\paren{\sin bx} = a/b\).
  \end{remark}
\end{example*}

\begin{example*}
  求\(\lim\limits_{x\to1} \,\paren[\big]{\sinp{1-x}}/\paren{\sqrt x - 1}\). % \rule{0ex}{3.5ex}

  \begin{equation*}
    \lim_{x\to1} \frac{\sinp{1-x}}{\sqrt x - 1}
    = \lim_{x\to1} \frac{-\paren{\sqrt x + 1} \sinp{x-1}}{x - 1}
    = -2.
  \end{equation*}
\end{example*}

\begin{example*}
  求\(\ \lim\limits_{\mathclap{x\to\pi/2}} \ \paren{\cos x}/\paren{\pi/2 - x}\). % \rule{0ex}{3.5ex}

  \begin{equation*}
    \lim_{x\to\pi/2} \frac{\cos x}{\pi/2 - x}
    = \,\smashoperator{\lim_{x\to\pi/2}} \frac{\cosp{\pi/2 - x - \pi/2}}{\pi/2 - x}
    = \,\smashoperator{\lim_{x\to\pi/2}} \frac{\sinp{\pi/2 - x}}{\pi/2 - x}
    = 1.
  \end{equation*}
\end{example*}

\begin{theorem*}
  \(\lim\limits_{x\to\infty} \paren{1 + 1/x}^x = e\). % \rule{0ex}{3.5ex}

  \begin{proof}
    使用定理~\ref{thm:seqe}和定理~\ref{thm:funcsqueeze}即可得证.
  \end{proof}
\end{theorem*}

\begin{example*}
  求\(\lim\limits_{x\to0} \,\paren[\big]{\ln(1+x)}/{x} = 1\). % \rule{0ex}{3.5ex}

  \begin{remark}
    令\(t = 1/x\),有
    \begin{equation*}
      \lim_{x\to0} \frac{\ln(1+x)}{x}
      = \lim_{x\to0} \ln(1+x)^{1/x}
      = \lim_{t\to\infty} \lnp[\bigg]{1 + \frac1t}^t
      = \ln e
      = 1.
    \end{equation*}
  \end{remark}
\end{example*}

\begin{example*}
  求\(\lim\limits_{x\to0} \,\paren{e^x-1}/{x}\).

  \begin{remark}
    令\(t = e^x - 1\),有
    \begin{equation*}
      \lim_{x\to0} \frac{e^x-1}{x}
      = \lim_{t\to0} \frac{t}{\ln(1+t)}
      = 1.
    \end{equation*}
  \end{remark}
\end{example*}

\begin{example*}
  求\(\lim\limits_{x\to0} \,\paren{a^x-1}/{x}\ (a > 0)\).

  \begin{remark}
    当\(a \ne 1\)时,有
    \begin{equation*}
      \lim_{x\to0} \frac{a^x-1}{x}
      = \lim_{x\to0} \frac{e^{x \ln a}-1}{x \ln a} \ln a
      = \ln a.
    \end{equation*}
    当\(a = 1\)时,有
    \begin{equation*}
      \lim_{x\to0} \frac{a^x-1}{x}
      = \lim_{x\to0} \frac{0}{x}
      = 0 = \ln a.
    \end{equation*}
    所以,\(\lim\limits_{x\to0} \,\paren{a^x-1}/{x} = \ln a\).
  \end{remark}
\end{example*}

\begin{example*}
  求\(\lim\limits_{x\to\infty} \paren[\big]{\paren{x+5}/\paren{x+2}}^{x+3}\).

  \begin{remark}
    稍作变形,有
    \begin{align*}
      \lim_{x\to\infty} \paren[\bigg]{\frac{x+5}{x+2}}^{x+3}\negthickspace
      &= \lim_{x\to\infty} \paren[\bigg]{1 + \frac{3}{x+2}}^{x+2} \paren[\bigg]{%
        1 + \frac{3}{x+2}} \\
      &= \lim_{x\to\infty} \paren[\bigg]{1 + \frac{1}{(x+2)/3}}^{(x+2)/3\cdot3} \paren[\bigg]{%
        1 + \frac{3}{x+2}} \\
      &= e^3.
    \end{align*}
  \end{remark}
\end{example*}

\begin{example*}
  求\(\lim\limits_{x\to0} \,(\cos x)^{1/\!\sin^2 x}\).

  \begin{remark}
    稍作变形,有
    \begin{align*}
      \lim_{x\to0} (\cos x)^{1/\!\sin^2 x}
      &= \lim_{x\to0} \expb[\Bigg]{\frac{\ln \cos x}{\sin^2 x}} \\
      &= \lim_{x\to0} \expb[\Bigg]{
        \frac{\lnp{1 - 2 \sin^2 \frac x2}}{
        \paren{2 \sin\frac x2 \cos\frac x2}^2}} \\
      &= \lim_{x\to0} \expb[\Bigg]{
        \frac{\lnp{1 - 2 \sin^2 \frac x2}}{-2 \sin^2 \frac x2}
        \cdot \frac{1}{-2 \cos^2 \frac x2}} \\
      &= e^{-1/2} = \frac1{\!\sqrt e\,}.
    \end{align*}
  \end{remark}
\end{example*}

\subpdfbookmark{思考}{B1.2.5.P}
\subsection*{思考}

复合函数求极限的条件是什么?若不满足该条件,会出现什么问题?

\ifshowsolp
  \pskip
  要在自变量趋向的点附近存在一个去心邻域,使得内部函数在此邻域上不等于所趋向的极限值.若不满足,则可能所求的极限不存在,或者等于其他值.
\fi

\ifshowex
\currentpdfbookmark{练习}{B1.2.5.E}
\subsection*{练习}

\begin{enumerate}
\item 函数
  \begin{equation*}
    f(x) = \frac{2+e^{1/x}}{1+e^{4/x}} + \frac{\sin x}{\abs{\,x\,}}
  \end{equation*}
  在原点处\uline{\makebox[6em]{}}.
  \begin{itemize}
    \renewcommand{\labelitemi}{\faCircleThin}
    \ifshowsol
    \item[\faCircle]
    \else
    \item
    \fi
    极限存在
  \item 左极限存在,右极限不存在
  \item 左极限不存在,右极限存在
  \item 左右极限都存在但不相等
  \end{itemize}

  \ifshowsol
    实际上,有
    \begin{gather*}
      \lim_{\,x\to0^+} \,f(x)
      = {}\smashoperator[l]{\lim_{\,x\to0^+}} \paren[\bigg]{\frac{2/e^{1/x} + 1}{1/e^{1/x} + e^{3/x}} + \frac{\sin x}{x}}
      = 1
      \siand
      \lim_{\,x\to0^-} \,f(x)
      = {}\smashoperator[l]{\lim_{\,x\to0^-}} \paren[\bigg]{\frac{2+e^{1/x}}{1+e^{4/x}} - \frac{\sin x}{x}}
      = 1.
    \end{gather*}
    所以函数\(\,f\mkern2mu\)在原点处的极限是\(1\).
  \fi

\item 若\(\lim\limits_{\,x \to x_0\!} \,f(x)\)存在且\(\lim\limits_{\,x \to x_0\!} g(x)\)不存在,则\uline{\makebox[10em]{}}.
  \begin{itemize}[itemsep=1ex]
    \renewcommand{\labelitemi}{\faCircleThin}
  \item \(\lim\limits_{\,x \to x_0\!} \,f(x)\,g(x)\)和\(\lim\limits_{\,x \to x_0\!} {g(x)}/{\,f(x)}\)一定都不存在
  \item \(\lim\limits_{\,x \to x_0\!} \,f(x)\,g(x)\)和\(\lim\limits_{\,x \to x_0\!} {g(x)}/{\,f(x)}\)一定都存在
  \item 在\(\lim\limits_{\,x \to x_0\!} \,f(x)\,g(x)\)和\(\lim\limits_{\,x \to x_0\!} {g(x)}/{\,f(x)}\)中恰有一个存在
    \ifshowsol
    \item[\faCircle]
    \else
    \item
    \fi
    \(\lim\limits_{\,x \to x_0\!} \paren*{\,f(x)+g(x)}\)和\(\lim\limits_{\,x \to x_0\!} \paren*{\,f(x)-g(x)}\)一定都不存在
  \end{itemize}

  \ifshowsol
    令\(\,f(x) = x\)和\(g(x) = 1/x\),可以证伪选项~A和~B.在此基础上,令\(g(x) = 1/x^2\),可以证伪选项~C.实际上,在题干的条件下,\(\!\lim\limits_{\,x \to x_0\!} \paren*{\,f(x)+g(x)}\)、\(\!\lim\limits_{\,x \to x_0\!} \paren*{\,f(x)-g(x)}\)和\(\!\lim\limits_{\,x \to x_0\!} {g(x)}/{\,f(x)}\)一定都不存在.当\(\!\lim\limits_{\,x \to x_0\!} \,f(x) \ne 0\)时,\(\!\lim\limits_{\,x \to x_0\!} \,f(x)\,g(x)\)一定不存在.当\(\!\lim\limits_{\,x \to x_0\!} \,f(x) = 0\)时,\(\!\lim\limits_{\,x \to x_0\!} \,f(x)\,g(x)\)可能存在也可能不存在.
  \fi

\item 若\(\lim\limits_{x\to\infty} \paren[\big]{\paren{x^2+1}/\paren{x+1} - ax -b} = 0\),则\(a\)和\(b\)的值分别为\uline{\makebox[3em]{%
      \ifshowsol
        \(1\)
      \fi}}和\uline{\makebox[3em]{%
      \ifshowsol
        \(-1\)
      \fi}}.

\item 求\(\lim\limits_{n\to\infty} \sin^2 \paren[\big]{\pi\sqrt{n^2+1}}\).

  \ifshowsol
    稍作变形,有
    \begin{align*}
      \lim_{n \to \infty} \sin^2 \pi\sqrt{n^2+1}
      &= \lim_{n \to \infty} \sin^2 \brce[\big]{\paren[\big]{\sqrt{n^2+1} - n + n} \pi} \\
      &= \lim_{n \to \infty}
        \Bigl\lbrace
        \sin\brkt[\big]{\paren[\big]{\sqrt{n^2+1} - n}\pi}
        \cos n\pi \\
      &\hphantom{= \lim_{n \to \infty} \lbrace} +
        \cos\brkt[\big]{\paren[\big]{\sqrt{n^2+1} - n}\pi}
        \sin n\pi
        \Bigr\rbrace^2 \\
      &= \lim_{n \to \infty}
        \sin^2\brkt[\big]{\paren[\big]{\sqrt{n^2+1} - n}\pi}
        \cos^2 n\pi \\
      &= \lim_{n \to \infty} \sin^2\brkt[\big]{π/\paren[\big]{\sqrt{n^2+1} + n}} \\
      &= 0.
    \end{align*}
  \fi

\item 若\(\!\lim\limits_{x\to-\infty} \paren[\big]{\sqrt{x^2 - x + 1} - ax - b} = 0\),则\(a\)和\(b\)的值分别为\uline{\makebox[3em]{%
      \ifshowsol
        \(-1\)
      \fi}}和\uline{\makebox[3em]{%
      \ifshowsol
        \(1/2\)
      \fi}}.

  \ifshowsol
    令\(t = -x\),有
    \begin{equation*}
      \begin{split}
        \lim_{x\to-\infty} \paren[\big]{\sqrt{x^2 - x + 1} - ax - b}
        &= \lim_{t\to+\infty} \paren[\big]{\sqrt{t^2 + t + 1} + at - b} \\
        &= \lim_{t\to+\infty} \frac{t^2 + t + 1 - (at - b)^2}{\sqrt{t^2 + t + 1} - at + b} \\
        &= \lim_{t\to+\infty} \frac{(1-a^2)t^2 + (1+2ab)t + 1 - b^2}{\sqrt{t^2 + t + 1} - at + b} \\
        &= \lim_{t\to+\infty} \frac{(1-a^2)t + (1+2ab) + (1 - b^2)/t}{\sqrt{1 + 1/t + 1/t^2} - a + b/t}.
      \end{split}
    \end{equation*}
    此时,若\(1 - a^2 \ne 0\),则其发散.矛盾,所以有\(a^2 = 1\).若\(a = 1\)且\(1 + 2ab \ne 0\),则其亦发散.矛盾.若\(a = 1\)且\(1 + 2ab = 0\),则其亦发散.矛盾,所以有\(a = -1\).此时,其收敛于\((1 - 2b)/2\),从而\(b = 1/2\).
  \fi

\item 求\(\lim\limits_{x\to0} \,\paren{2 \sin x + \cos x}^{1/x}\).

  \ifshowsol
    稍作变形,有
    \begin{align*}
      \lim_{x\to0} \paren{2 \sin x + \cos x}^{1/x}
      &= \lim_{x\to0} \expb[\bigg]{
        \frac{\lnp{1 + 2 \sin x + \cos x - 1}}{2 \sin x + \cos x - 1}
        \cdot
        \frac{2 \sin x + \cos x - 1}{x}} \\
      &= e^2.
    \end{align*}
  \fi

\item 若函数
  \begin{equation*}
    f(x) =
    \begin{dcases}
      \frac{\sin x}{x}, & x \ne 0, \\
      0, & x = 0
    \end{dcases}
    \txt{且}
    g(t) = t \sin\frac1t,
    \qquad
  \end{equation*}
  则\(\lim\limits_{t\to0} \,f\,\paren*{g(t)}\)\uline{\makebox[6em]{}}.
  \begin{itemize}
    \renewcommand{\labelitemi}{\faCircleThin}
  \item 等于\(1\)
  \item 等于\(0\)
  \item 等于\(-1\)
    \ifshowsol
    \item[\faCircle]
    \else
    \item
    \fi
    不存在
  \end{itemize}

  \ifshowsol
    若令\(t_n = 1/(nπ)\),则\(\lim\limits_{n\to\infty} \,f\,\paren*{g(t_n)} = 0\).若令\(t_n = 1/(n\pi+1)\),则\(\lim\limits_{n\to\infty} \,f\,\paren*{g(t_n)} = 1\).根据定理~\ref{thm:limequiv},所以\(\lim\limits_{t\to0} \,f\,\paren*{g(t)}\)不存在.
  \fi

\item 求\(\!\lim\limits_{\,x\to0^+\!} \paren{\cos\sqrt x}^{\pi/x}\).

  \ifshowsol
    稍作变形,有
    \begin{equation*}
      \begin{split}
        \lim_{\,x\to0^+\!} \paren{\cos\sqrt x}^{\pi/x}
        &= \lim_{\,x\to0^+\!} \expb[\bigg]{\frac{\pi \ln\cos\sqrt x}{x}}
          = \lim_{\,x\to0^+\!} \expb[\bigg]{
          \frac{\pi \lnp{1 + \cos\sqrt x - 1}}{\cos\sqrt x - 1}
          \cdot
          \frac{\cos\sqrt x - 1}{x}} \\
        &= e^{-\pi/2}.
      \end{split}
    \end{equation*}
  \fi

\item 求\(\lim\limits_{x\to1} \,\paren[\big]{\sinp{x-1}}/\paren{\sqrt x - 1}\).
  \ifshowsol
    \begin{equation*}
      \lim_{x\to1} \frac{\sinp{x-1}}{\sqrt x - 1}
      = \lim_{x\to1} \frac{\sinp{x-1}}{x-1} \cdot \frac{x-1}{\sqrt x - 1}
      = 2\,.
    \end{equation*}
  \fi
\end{enumerate}
\fi

\section{无穷小量及其(阶的)比较}

\begin{definition*}
  若\(\!\lim\limits_{\,x \to x_0\!} \,f(x) = 0\),则称函数\(\,f\mkern2mu\)在\(x \to x_0\)时是无穷小量,记作\(\,f(x) = \littleo(1)\ (x \to x_0)\).
\end{definition*}

\begin{definition*}
  函数\(\,f\mkern2mu\)在\(x \to x_0\)时是无穷大量,记作\(\!\lim\limits_{\,x \to x_0\!} \,f(x) = \infty\),当且仅当:对于任意的\(M > 0\)都存在\(δ > 0\)使得当\(0 < \abs*{\,x - x_0\,} < δ\)时都有\(\abs*{\mkern6mu f(x)\,} > M\).
\end{definition*}

\begin{theorem*}
  无穷大量的倒数是无穷小量.
\end{theorem*}

\begin{theorem*}
  非零无穷小量的倒数是无穷大量.
\end{theorem*}

\begin{theorem*}
  无穷小量与有界变量的乘积还是无穷小量.
\end{theorem*}

\begin{theorem*}
  函数\(\,f\mkern2mu\)极限是\(A\),当且仅当它可以写成\(A\)与某个无穷小量之和的形式.
\end{theorem*}

% https://portal.tpu.ru/SHARED/k/KONVAL/Textbooks/Tab1/Konev-Limits_of_Sequences_and_Functions_Textbook.pdf
\begin{definition*}
  设在同一个极限过程中,函数\(\,f\mkern2mu\)和\(\mkern1mu g\)都是无穷小量.
  \begin{enumerate}[topsep=3pt,itemsep=0ex]
    \renewcommand{\labelenumi}{\enumparen{\arabic{enumi}}}
  \item 函数\(\,f\mkern2mu\)和\(\mkern1mu g\)是\kenten{同阶无穷小量},当且仅当:\(\,f\mkern2mu/g\)收敛于某个非零常数;特别地,函数\(\,f\mkern2mu\)和\(\mkern1mu g\)是\mbox{\kenten{等价无穷小量}},记作\(\,f \sim g\),当且仅当:\(\,f\mkern2mu/g\)收敛于\(1\);
  \item 函数\(\,f\mkern2mu\)是函数\(\mkern1mu g\)的\kenten{高阶无穷小量},函数\(\mkern2mu g\)是函数\(\,f\mkern2mu\)的\kenten{低阶无穷小量},当且仅当:\(\,f\mkern2mu/g\)是无穷小量;特别地,函数\(\,f\mkern2mu\)是函数\(\mkern2mu g\)的\(n\)阶无穷小量,当且仅当:\(\,f/g^n\!\)收敛于某个非零常数.
  \end{enumerate}

  % https://mathworld.wolfram.com/Little-ONotation.html
  \begin{remark}
    我们经常用小\(o\)关系\footnote{关于小o和大O关系,最常见的定义,除了还要求\(\mkern1mu g\)是一个正函数之外,其实并不关心\(\,f\mkern2mu\)和\(\mkern1mu g\)是否收敛或者为无穷大量,只要它们的商满足条件即可.若\(\,f\mkern2mu\)是\(\mkern1mu g\)的高阶无穷小量,则一定满足\(\,f = \littleo(\abs{\mkern1mu g})\);反之不然.这里的小o和大O记号,其实表达的是一个函数的集合,更严谨的用法应该是\(\,f \in \littleop{\mkern1mu g}\)和\(\,f \in \bigOp{\mkern1mu g}\).}的记号把“函数\(\,f\mkern2mu\)是函数\(\mkern1mu g\)的高阶无穷小量”记作\(\,f= \littleo(\mkern1mu g)\).
  \end{remark}

  \begin{remark}
    我们简称函数\(x-x_0\)在点\(x_0\)处和函数\(1/x\)在无穷处的\(k\)阶无穷小量为\(k\)阶无穷小量.也就是说,函数\(\,f\mkern2mu\)在点\(x_0\)处是\(k\)阶无穷小量,当且仅当\(0 < \abs[\big]{\,\lim\limits_{x \to x_0\!} {\,f(x)}/{(x-x_0)^k}\,} < +\infty\);函数\(\,f\mkern2mu\)在无穷处是\(k\)阶无穷小量,当且仅当\(0 < \abs[\big]{\,\lim\limits_{x\to\infty\!} x^k\,f(x)\,} < +\infty\).
  \end{remark}
\end{definition*}

% https://mathworld.wolfram.com/Big-ONotation.html
\begin{definition*}
  在同一个极限过程中,函数\(\,f\mkern2mu\)和\(\mkern1mu g\)都是无穷小量.若\(\,f\mkern2mu/g\)有界,则称\(\,f\mkern2mu\)和\(\mkern1mu g\)有大O关系\footnote{同上.},记作\(\,f = \bigO(g)\).
\end{definition*}

\begin{example*}
  证明\(\sqrt[\leftroot{-2}\uproot{2}k]{1+x} - 1 \sim x/k\ (x \to 0)\).

  \begin{proof}
    利用公式\(a^n-b^n = (a-b)(a^{n-1} + a^{n-2}b + \dots + b^{n-1})\),稍作变形,有
    \begin{align*}
      \lim_{x\to0} \frac{\sqrt[\leftroot{-2}\uproot{2}k]{1+x} - 1}{x/k}
      &= \lim_{x\to0} \frac{k(1+x-1)}{
        x \sum\limits_{j=0}^{\mathclap{k-1}} (1+x)^{j/k}}
      && \reason{分子分母同乘以\(\Sigma\)} \\
      &= k\bigg/\!\lim_{x\to0} \sum_{j=0}^{k-1} (1+x)^{j/k}
      && \reason{除法法则} \\
      &= k\bigg/\!\sum_{j=0}^{k-1} \lim_{x\to0} (1+x)^{j/k}
      && \reason{加法法则} \\
      &= k/k = 1.
      && \reason{复合法则}
         \qedhere
    \end{align*}
  \end{proof}
\end{example*}

\begin{theorem*}
  若\(\,f \sim α,\ g \sim β \),则\(\lim \,f\mkern2mu/g = \lim α/β\).

  \begin{proof}
    \begin{equation*}
      \lim \frac{\,f}{g}
      = \lim \frac{\,f}{α} \cdot \frac{α}{β} \cdot \frac{β}{g}
      = \lim \frac{α}{β}
      \qedhere
    \end{equation*}
  \end{proof}
\end{theorem*}

\begin{example*}
  求\(\lim\limits_{x\to0} \,\paren{\tan x - \sin x}/{x^3}\). % \rule[-2ex]{0ex}{0ex}

  \begin{remark}
    稍作变形,有
    \begin{equation*}
      \lim_{x\to0} \frac{\tan x - \sin x}{x^3}
      = \lim_{x\to0} \frac{1}{\cos x} \cdot \frac{1 - \cos x}{x^2} \cdot \frac{\sin x}{x}
      = \frac12.
    \end{equation*}
  \end{remark}
\end{example*}

\begin{example*}
  求\(\lim\limits_{x\to0} \,\paren{e^{1-\cos x}-1}/{x^2}\). % \rule[-2ex]{0ex}{3.5ex}

  \begin{remark}
    使用等价无穷小量替换和复合法则,有
    \begin{equation*}
      \lim_{x\to0} \frac{e^{1-\cos x}-1}{x^2}
      = \lim_{x\to0} \frac{e^{1-\cos x}-1}{2(1-\cos x)}
      = \frac12.
    \end{equation*}
  \end{remark}
\end{example*}

% report
\begin{example*}
  求\(\lim\limits_{x\to0} \,\paren[\big]{\ln(1+ax^m)}/\paren[\big]{1-\cos(1-\cos x)}\ (a \ne 0, m \ge 1)\). % \rule[-2ex]{0ex}{3.5ex}

  \begin{remark}
    使用等价无穷小量替换和复合法则,有
    \begin{equation*}
      \lim_{x\to0} \frac{\ln(1+ax^m)}{1-\cos(1-\cos x)}
      = \lim_{x\to0} \frac{ax^m}{(1-\cos x)^2\!/2}
      = \lim_{x\to0} \frac{ax^m}{x^4\!/8}
      = \lim_{x\to0} 8\,ax^{m-4} =
      \begin{cases}
        0, & m > 4, \\
        8\,a, & m = 4, \\
        \infty, & 1 \le m < 4.
      \end{cases}
    \end{equation*}
  \end{remark}
\end{example*}

\subpdfbookmark{思考}{B1.2.6.P}
\subsection*{思考}

\begin{enumerate}
\item 两个无穷小量之和是不是无穷小量?两个无穷大量之和是不是无穷大量?

  \ifshowsolp
    是.不是.前者可由极限的加法法则得出.后者可以构造反例,函数\(\,f(x) = x,\ g(x) = -x\).这里的\(\,f\mkern1mu\)和\(\mkern1mu g\)在无穷处都是无穷大量,但是\(\,f+g\)是常函数.
  \fi

\item 任意两个无穷小量是否都可以比阶?试举例说明.

  \ifshowsolp
    不一定.例如,函数\(x \sin\frac1x\)和\(x\)在原点处都是无穷小量,但是它们的商发散.
  \fi
\end{enumerate}

\ifshowex
\currentpdfbookmark{练习}{B1.2.6.E}
\subsection*{练习}

\begin{enumerate}
\item 下列说法中,正确的是\uline{\makebox[6em]{}}.
  \begin{itemize}
    \renewcommand{\labelitemi}{\faCircleThin}
    \ifshowsol
    \item[\faCircle]
    \else
    \item
    \fi
    无穷小量与无穷大量之和为无穷大量
  \item 无穷小量与无穷大量之积为无穷大量
  \item 无穷小量与无穷大量之差为无穷小量
  \item 无穷小量与无穷大量之积为无穷小量
  \end{itemize}

  \ifshowsol
    令\(\,f(x) = \sin x,\ g(x) = 1/x\),则在原点处\(\,f\mkern2mu\)和\(\mkern1mu g\)分别是无穷小量和无穷大量.但是\(\,f \cdot g\)即不是无穷小量,也不是无穷大量.所以选项~B和~D都错了.设\(\,f\mkern2mu\)和\(\mkern2mu g\)在点\(a\)处分别是无穷小量和无穷大量.对于任意的\(M > 0\)都存在一个去心邻域使得当\(x\)在此邻域上时都有
    \begin{gather*}
      \abs*{\mkern6mu f(x)\,} < M
      \txt{且}
      2\,M < \abs*{\,g(x)\,} \\
      \shortintertext{即}
      \abs*{\mkern6mu f(x) \pm g(x)\,}
      \ge \abs[\Big]{\,\abs*{\,g(x)\,} - \abs*{\mkern6mu f(x)\,}\,}
      \ge \abs*{\,g(x)\,} - \abs*{\mkern6mu f(x)\,}
      > M.
    \end{gather*}
    所以选项~A对了而选项~C错了.
  \fi

\item 当\(x \to 0\)时,下列函数中不是无穷小量的是\uline{\makebox[6em]{}}.
  \begin{itemize}
    \renewcommand{\labelitemi}{\faCircleThin}
  \item \(\sin(\tan x^2)\)
  \item \(x \cos(1/x)\)
    \ifshowsol
    \item[\faCircle]
    \else
    \item
    \fi
    \(\sin\paren{\cos x}\)
  \item \(\lnp{1 + \sin x}\)
  \end{itemize}

\item 当\(x \to 0^+\)时,下列函数中不是无穷大量的是\uline{\makebox[6em]{}}.
  \begin{itemize}
    \renewcommand{\labelitemi}{\faCircleThin}
  \item \(x - \ln x\)
  \item \(x + \ln x\)
    \ifshowsol
    \item[\faCircle]
    \else
    \item
    \fi
    \(x \ln x\)
  \item \(\paren{\ln x}/{x}\)   % \rule{0ex}{3.5ex}
  \end{itemize}

  \ifshowsol
    对于选项~C,令\(t = -\ln x\),实际上有
    \begin{equation*}
      \lim_{\,x \to 0^+\!} \!x \ln x
      = - \smashoperator{\lim_{\,t \to +\infty\!}} te^{-t}
      = 0.
    \end{equation*}
  \fi

\item 当\(x \to 0^+\)时,下列无穷小量按照其阶由低到高排列正确的是\uline{\makebox[10em]{}}.
  \begin{itemize}
    \renewcommand{\labelitemi}{\faCircleThin}
  \item \(\sin x^2,\ \sin(\tan x),\ e^{x^3}-1,\ \lnp{1+\sqrt x}\)
  \item \(\lnp{1+\sqrt x},\ \sin x^2,\ \sin(\tan x),\ e^{x^3}-1\)
  \item \(\sin(\tan x),\ \lnp{1+\sqrt x},\ \sin x^2,\ e^{x^3}-1\)
    \ifshowsol
    \item[\faCircle]
    \else
    \item
    \fi
    \(\lnp{1+\sqrt x},\ \sin(\tan x),\ \sin x^2,\ e^{x^3}-1\)
  \end{itemize}

  \ifshowsol
    实际上,有
    \begin{equation*}
      \lnp{1+\sqrt x} \sim \sqrt x, \quad
      \sin(\tan x) \sim x, \quad
      \sin x^2 \sim x^2, \quad
      e^{x^3}-1 \sim x^3.
    \end{equation*}
  \fi

\item 当\(n \to \infty\)时,下列无穷大量按照其阶由低到高排列正确的是\uline{\makebox[6em]{}}.
  \begin{itemize}
    \renewcommand{\labelitemi}{\faCircleThin}
    \ifshowsol
    \item[\faCircle]
    \else
    \item
    \fi
    \(\sqrt n,\ n^2,\ e^n,\ n!,\ n^n\)
  \item \(n^2,\ e^n,\ n!,\ \sqrt n,\ n^n\)
  \item \(\sqrt n,\ n^2,\ n!,\ e^n,\ n^n\)
  \item \(\sqrt n,\ n^2,\ n!,\ n^n,\ e^n\)
  \end{itemize}

  \ifshowsol
    关于\(e^n = \littleop{n!}\),参见例~\ref{eg:factexp}.下面证明一下,对于给定的数\(a > 1\)和正整数\(k\),都有\(n^k = \littleop{a^n}\).

    \begin{proof}
      令\(α = a - 1, c = α^{k+1}\!/(k+1)!\),则
      \begin{gather*}
        a^n = (1+α)^n = \sum_{j=0}^n \binom nj α^j > \binom{n}{k+1} α^{k+1}
        = c n^{k+1} + \bigOp{n^k}, \\
        \shortintertext{所以有}
        0 < \frac{n^k}{a^n} < \frac{n^k}{c n^{k+1} + \bigOp{n^k}}
        = \frac{1}{cn + \bigOp{1}}.
        \qedhere
      \end{gather*}
    \end{proof}
  \fi

\item 当\(x \to 0^+\)时,与\(\sqrt x\)等价的无穷小量是\uline{\makebox[6em]{}}.
  \begin{itemize}
    \renewcommand{\labelitemi}{\faCircleThin}
  \item \(1 - e^{\sqrt x}\)
    \ifshowsol
    \item[\faCircle]
    \else
    \item
    \fi
    \(\ln\brkt{\paren{1+x}/\paren{1-\sqrt x}}\) % \rule[-2ex]{0ex}{5ex}
  \item \(\sqrt{1 + \sqrt x} - 1\)
  \item \(1 - \cos\sqrt x\)
  \end{itemize}

  \ifshowsol
    实际上,有
    \begin{equation*}
      1 - e^{\sqrt x} \sim -\sqrt x, \quad
      \ln \frac{1+x}{1-\sqrt x} \sim \sqrt x, \quad
      \sqrt{1 + \sqrt x} - 1 \sim \frac{\sqrt x}{2}, \quad
      1 - \cos\sqrt x \sim \frac x2.
    \end{equation*}
  \fi

\item 当\(x \to 0\)时,有\(\paren{1-ax^2}^{1/4} - 1 \sim x \sin x\).求\(a\)的值.

  \ifshowsol
    因为\(x \sin x \sim x^2\)和\(\paren{1-ax^2}^{1/4} - 1 \sim -ax^2\!/4\),所以有\(a = -4\).
  \fi
\end{enumerate}
\fi

% Local Variables:
% TeX-engine: luatex
% TeX-master: "微积分B"
% LaTeX-indent-begin-exceptions-list: ("ifshowex")
% End:
